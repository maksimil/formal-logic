\section{Правила вывода}

\newcommand\figsep{0cm}
\begin{marginfigure}
	\[
		\begin{nd}
			\have{1}{B}
			\have{2}{A}
			\have{3}{A\land B} \ai{1,2}
		\end{nd}
	\]

	\caption{Пример использования правила $A,B\vdash A\land B$.}\label{fig:ex_ai}
\end{marginfigure}

\begin{marginfigure}[\figsep]
	\[
		\begin{nd}
			\open
			\hypo{1}{A}
			\have{}{...}
			\have{2}{C}
			\close
			\have{}{A\implies C}\ii{1-2}
		\end{nd}
	\]

	\caption{Пример использования правила ${[A\vdash C]\vdash (A\implies C)}$.}
	\label{fig:ex_ii}
\end{marginfigure}

\begin{marginfigure}[\figsep]
	\[
		\begin{nd}
			\have{1}{\exists x:P(x)}
			\open[u]
			\hypo{2}{P(u)}
			\have{}{...}
			\have{3}{C}
			\close
			\have{4}{C}\Ee{1,2-3}
		\end{nd}
	\]

	\caption{Пример использования правила ${[\exists x:P(x)],[t;P(t)\vdash C]\vdash C}$.}
	\label{fig:ex_Ee}
\end{marginfigure}

\begin{marginfigure}[\figsep]
	\[
		\begin{nd}
			\have{1}{S}	\by{AI}{}
		\end{nd}
	\]

	\caption{Пример использования правила $\vdash S$, где $S$ --- аксиома.}
	\label{fig:ex_xi}
\end{marginfigure}

Начнём построение формальной системы. Алфавит уже определён, но также введём
в него способ построения диаграм.

Если $A$, $B$ --- формулы, $\gamma$ --- переменная,
а $K$ --- квантор, то выражения ${A\implies B}$, $A\land B$, $A\lor B$, $\lnot A$,
$K\gamma A$, $A=B$ --- формулы.
Диаграммы Фитча и $\bot$ тоже являются формулами.

Правило вывода: $A\vdash T$, если можно построить диаграмму,
предполагающую $A$ и заключающую $T$.

Пусть $\mathcal A$ --- совокупность формул. Если $\mathcal A$ находится
в одной области видимости и $\mathcal A\vdash B$,
то можно добавить строку $B$ в эту область видимости.
Если $T$ выводимо без предположений
(обозначается как $\vdash T$),
то в диаграмму можно добавить строку $T$ в произвольную область видимости.

Введём остальные правила вывода.
Для символов будут введены правила введения и исключения.
При использовании правил справа пишется
название правила и какие строки использованы. Для краткости обозначим с помощью
$[A\vdash B]$ диаграмму из $A$ в $B$ (рис.~\ref{fig:ex_ii}),
а с помощью ${[t;A\vdash B]}$ ---
диаграмму из $A$ в $B$, вводящее переменную $t$ (рис.~\ref{fig:ex_Ee}).
Мы можем использовать $\vdash$, чтобы обозначить выводимость и
существование диаграммы, потому что в данной формальной ситеме это достаточно
близкие понятия.

Ниже предоставлен полный список правил. Примеры их использования
можно увидеть на рис.~\ref{fig:ex_ai},\ref{fig:ex_ii},\ref{fig:ex_Ee},\ref{fig:ex_xi}.

\hspace{-0.7cm}
\begin{tabular}{rl|rl}
	($\Rightarrow$I) & $[A\vdash B]\vdash (A\implies B)$           &
	($\Rightarrow$E) & $A,(A\implies B)\vdash B$                     \\
	($\land$I)       & $A,B\vdash A\land B$                        &
	($\land$E)       & $A\land B\vdash A$ или $A\land B\vdash B$     \\
	($\lor$I)        & $A\vdash A\lor B$ или $B\vdash A\lor B$     &
	($\lor$E)        & $[A\lor B],[A\vdash C],[B\vdash C]\vdash C$   \\
	($\lnot$I)       & $[A\vdash\bot]\vdash \lnot A$               &
	($\lnot$E)       & $[\lnot A\vdash\bot]\vdash A$                 \\
	($\bot$I)        & $A,\lnot A\vdash \bot$                      &
	($\bot$E)        & $\bot\vdash A$                                \\
	($\forall$I)     & $[t;\vdash P(t)]\vdash (\forall x)P(x)$     &
	($\forall$E)     & $(\forall x)P(x)\vdash P(t)$                  \\
	($\exists$I)     & $P(t)\vdash [\exists x:P(x)]$               &
	($\exists$E)     & $[\exists x:P(x)],[t;P(t)\vdash C]\vdash C$   \\
	($=$I)           & $\vdash V=V$                                &
	($=$E)           & $[u=v],P(u)\vdash P(v)$                       \\
	(AI)             & $\vdash S$                                  &
	                 & или $[u=v], P(v)\vdash P(u)$                  \\
\end{tabular}

\newcommand\bi[1]{\by{$\bot$I}{#1}}
где $A$, $B$ --- формулы, $P$ --- шаблон формулы\footnote{
	Шаблон формулы определяется следующим образом:
	В формуле $P$ определяется какой-то символ, который будет заменён
	аргументом. В формуле $P(x)$ вместо него стоит $x$.

	Например, возьмём формулу
	\[
		Q=\xi\land t
	\]
	и выберем в ней символ $\xi$,
	тогда $Q(\alpha)=\alpha\land t$.
}, $S$ --- аксиома,
$t$, $u$, $v$, $x$ --- переменные, $V$ --- такое выражение, что ${V=V}$ --- формула.
Под ``или'' имеется в виду, чт
два правила обозначаются одинаково.
В качестве примера докажем правило R: $A\vdash A$\footnote{
	R --- Reiteration, повторение.
} и применим его
для доказательства $A\vdash (B\implies A)$.
\[
	\begin{array}{l|l}
		\begin{nd}
			\hypo{1}{A}
			\open
			\hypo{2}{\lnot A}
			\have{3}{\bot}\bi{1,2}
			\close
			\have{4}{A}\ne{2-3}
		\end{nd} &
		\begin{nd}
			\hypo{1}{A}
			\open
			\hypo{2}{B}
			\have{3}{A}\r{1}
			\close
			\have{4}{B\implies A}\ii{2-3}
		\end{nd}
	\end{array}
\]

Докажем закон исключённого третьего (EM): $\vdash A\lor \lnot A$.
\[
	\begin{nd}
		\hypo{}{\qquad\qquad}
		\open
		\hypo{1}{\lnot(A\lor\lnot A)}
		\open
		\hypo{2}{A}
		\have{3}{A\lor \lnot A}\oi{2}
		\have{4}{\bot}\bi{1,3}
		\close
		\have{5}{\lnot A}\ni{2-4}
		\have{6}{A\lor\lnot A}\oi{5}
		\have{7}{\bot}\bi{1,6}
		\close
		\have{8}{A\lor\lnot A}\ne{1-7}
	\end{nd}
\]

Для завершения построения формальной системы
введём аксиому: $\lnot\bot$. Если $\vdash T$, то и $\lnot\bot\vdash T$,
тогда $T$ --- теорема. Заметим, что благодаря правилу AI для всякой
теоремы $T$ справедливо $\vdash T$.

Докажем
$(\forall x)P(x)\vdash\lnot[\exists x:\lnot P(x)]$\footnote{Вспомните
	первое доказательство в этой книге и сравните словесные рассуждения с
	диаграммой Фитча.

	Из диаграммы можно получить и сами словесные рассуждения:
	\begin{enumerate}[label=(\arabic*)]
		\item{}Предположим $(\forall x)P(x)$.\label{enum:1}
		\item{}Предположим $\exists x:\lnot P(x)$.\label{enum:2}
		\item{}Для произвольного $t$ предположим $\lnot P(t)$.\label{enum:3}
		\item{}По $\forall$E, \ref{enum:1}: $P(t)$.\label{enum:4}
		\item{}По $\bot$I, \ref{enum:3} и \ref{enum:4}: $\bot$.\label{enum:5}
		\item{}По $\exists$E, \ref{enum:2}, \ref{enum:3}-\ref{enum:5}: $\bot$.\label{enum:6}
		\item{}По $\lnot$I, \ref{enum:2}-\ref{enum:6}: $\lnot[\exists x:\lnot P(x)]$.\qed
	\end{enumerate}

	Их можно привести к более человеческому виду и получить доказательство на
	с.~\pageref{wordproof}.
}.
\[
	\begin{nd}
		\hypo{1}{(\forall x)P(x)}
		\open
		\hypo{2}{\exists x:\lnot P(x)}
		\open[t]
		\hypo{3}{\lnot P(t)}
		\have{4}{P(t)}\Ae{1}
		\have{5}{\bot}\bi{3,4}
		\close
		\have{6}{\bot}\Ee{2,3-5}
		\close
		\have{7}{\lnot[\exists x:\lnot P(x)]}\ni{2-6}
	\end{nd}
\]

{\it Упражнения:}
\begin{enumerate}
	\item{}Обосновать словесно правила вывода. Почему такая система называется
	естественной дедукцией?
	\item{}Какие формулы будут теоремами, если $\bot$ --- аксиома?
	Какие формулы будут теоремами, если введены аксиомы, противоречащие
	друг другу?
	\item{}Определить области видимости для формул $A$, $\lnot A$, $A\lor\lnot A$
	и $\lnot(A\lor\lnot A)$ в доказательстве $\vdash A\lor\lnot A$.
	\item{}*Доказать $\bot$E, используя остальные правила.\label{ex:be_proof}
	\pagebreak
	\item{}*Доказать законы:\label{ex:neg_rules_proof}
	\begin{multicols}{2}
		\begin{enumerate}
			\item[($\lnot\land$)]{}$\lnot(A\land B)\vdash \lnot A\lor \lnot B$
			\item[($\lnot\lor$)]{}$\lnot(A\lor B)\vdash \lnot A\land \lnot B$
			\item[($\land\lnot$)]{}$\lnot A\land\lnot B\vdash \lnot(A\lor B)$
			\item[($\lor\lnot$)]{}$\lnot A\lor\lnot B\vdash \lnot(A\land B)$
			\item[($\lnot\forall$)]{}$\lnot[(\forall x)P(x)]\vdash\exists x:\lnot P(x)$
			\item[($\lnot\exists$)]{}$\lnot[\exists x:P(x)]\vdash (\forall x)\lnot P(x)$
			\item[($\forall\lnot$)]{}$(\forall x)\lnot P(x)\vdash \lnot[\exists x:P(x)]$
			\item[($\exists\lnot$)]{}$\exists x:\lnot P(x)\vdash \lnot[(\forall x)P(x)]$
		\end{enumerate}
	\end{multicols}
\end{enumerate}

{\it Решение упражнения \ref{ex:neg_rules_proof}:}
\[
	\begin{array}{l|l}
		\lnot(A\land B)\vdash \lnot A\lor\lnot B      &
		\lnot(A\lor B)\vdash \lnot A\land \lnot B       \\
		\begin{nd}
			\hypo{1}{\lnot(A\land B)}
			\have{2}{A\lor \lnot A}\by{EM}{}
			\open
			\hypo{3}{\lnot A}
			\have{4}{\lnot A\lor\lnot B}\oi{3}
			\close
			\open
			\hypo{5}{A}
			\open
			\hypo{6}{B}
			\have{7}{A\land B}\ai{5,6}
			\have{8}{\bot}\bi{1,7}
			\close
			\have{9}{\lnot B}\ni{6-8}
			\have{10}{\lnot A\lor \lnot B}\oi{9}
			\close
			\have{11}{\lnot A\lor \lnot B}\oe{2,3-4,5-10}
		\end{nd} &
		\begin{nd}
			\hypo{1}{\lnot(A\lor B)}
			\open
			\hypo{2}{A}
			\have{3}{A\lor B}\oi{2}
			\have{4}{\bot}\bi{1,3}
			\close
			\have{5}{\lnot A}\ni{2-4}
			\open
			\hypo{6}{B}
			\have{7}{A\lor B}\oi{6}
			\have{8}{\bot}\bi{1,7}
			\close
			\have{9}{\lnot B}\ni{6-8}
			\have{10}{\lnot A\land \lnot B}\ai{5,9}
		\end{nd}          \\\hline
		\lnot A\land\lnot B\vdash \lnot(A\lor B)      &
		\lnot A\lor\lnot B\vdash \lnot(A\land B)        \\
		\begin{nd}
			\hypo{1}{\lnot A\land\lnot B}
			\open
			\hypo{2}{A\lor B}
			\open
			\hypo{3}{A}
			\have{4}{\lnot A}\ae{1}
			\have{5}{\bot}\bi{3,4}
			\close
			\open
			\hypo{6}{B}
			\have{7}{\lnot B}\ae{1}
			\have{8}{\bot}\bi{6,7}
			\close
			\have{9}{\bot}\oe{2,3-5,6-8}
			\close
			\have{10}{\lnot(A\lor B)}\ni{2-9}
		\end{nd}             &
		\begin{nd}
			\hypo{1}{\lnot A\lor\lnot B}
			\open
			\hypo{2}{A\land B}
			\open
			\hypo{3}{\lnot A}
			\have{4}{A}\ae{2}
			\have{5}{\bot}\bi{3,4}
			\close
			\open
			\hypo{6}{\lnot B}
			\have{7}{B}\ae{2}
			\have{8}{\bot}\bi{6,7}
			\close
			\have{9}{\bot}\oe{1,3-5,6-8}
			\close
			\have{10}{\lnot (A\land B)}\ni{2-9}
		\end{nd}
	\end{array}
\]
\[
	\begin{array}{l|l}
		\lnot[(\forall x)P(x)]\vdash \exists x:\lnot P(x) &
		\lnot[\exists x:P(x)]\vdash (\forall x)\lnot P(x)   \\
		\begin{nd}
			\hypo{1}{\lnot[(\forall x)P(x)]}
			\open
			\hypo{2}{\lnot[\exists x:\lnot P(x)]}
			\open[t]
			\hypo{3}{}
			\open
			\hypo{7}{\lnot P(t)}
			\have{8}{\exists x:\lnot P(x)}\Ei{7}
			\have{9}{\bot}\bi{2,8}
			\close
			\have{11}{P(t)}\ne{7-9}
			\close
			\have{12}{(\forall x)P(x)}\Ai{3-11}
			\have{13}{\bot}\bi{1,12}
			\close
			\have{14}{\exists x:\lnot P(x)}\ne{2-13}
		\end{nd}          &
		\begin{nd}
			\hypo{1}{\lnot[\exists x:P(x)]}
			\open[t]
			\hypo{2}{}
			\open
			\hypo{3}{P(t)}
			\have{4}{\exists x:P(x)}\Ei{3}
			\have{5}{\bot}\bi{1,4}
			\close
			\have{6}{\lnot P(t)}\ni{3-5}
			\close
			\have{6}{(\forall x)\lnot P(x)}\Ai{2-6}
		\end{nd}              \\\hline
		(\forall x)\lnot P(x)\vdash \lnot[\exists x:P(x)] &
		\exists x:\lnot P(x)\vdash \lnot[(\forall x)P(x)]   \\
		\begin{nd}
			\hypo{1}{(\forall x)\lnot P(x)}
			\open
			\hypo{2}{\exists x:P(x)}
			\open[t]
			\hypo{3}{P(t)}
			\have{4}{\lnot P(t)}\Ae{1}
			\have{5}{\bot}\ni{3,4}
			\close
			\have{6}{\bot}\Ee{2,3-5}
			\close
			\have{7}{\lnot[\exists x:P(x)]}\ni{2-6}
		\end{nd}         &
		\begin{nd}
			\hypo{1}{\exists x:\lnot P(x)}
			\open
			\hypo{2}{(\forall x)P(x)}
			\open[t]
			\hypo{3}{\lnot P(t)}
			\have{4}{P(t)}\Ae{2}
			\have{5}{\bot}\bi{3,4}
			\close
			\have{6}{\bot}\Ee{1,3-5}
			\close
			\have{7}{\lnot[(\forall x)P(x)]}\ni{2-6}
		\end{nd}
	\end{array}
\]

{\it Решение упражнения \ref{ex:be_proof}:}
\[
	\begin{array}{l}
		\bot\vdash A \\
		\begin{nd}
			\hypo{1}{\bot}
			\open
			\hypo{2}{\lnot A}
			\have{3}{\bot}\r{1}
			\close
			\have{4}{A}\ne{2-3}
		\end{nd}
	\end{array}
\]

\pagebreak
