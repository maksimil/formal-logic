\section{Теория множеств}

Введём новый символ --- $\in$. ${x\in S}$ означает, что $x$ лежит в множестве $S$.
Отметим, что такие словесные интерпретации выражений не обязательны и не имеют
значения для формальной системы. Для неё формулы это просто строки,
которые можно преобразовывать по определённым правилам.

До этого мы обращались только с одним {\it родом} переменных --- формулой.
То есть всякая переменная обозначала формулу, но теперь мы вводим новый род
переменных --- {\it множество}.

Расширим понятие формулы:
\begin{enumerate}
	\item{}Если $\gamma$ --- переменная, а $S$ --- множество\footnote{
		То есть род переменной $S$ --- множество.
	},
	то $\gamma\in S$ --- формула.
	\item{}Если $\alpha$ и $\beta$ --- переменные одного рода,
	то $\alpha=\beta$ --- формула.
\end{enumerate}

Введём обозначения:
\newcommand\eset{\varnothing}
\begin{enumerate}
	\item{}$\eset$ --- {\it пустое множество}, для него выполняется аксиома
	${(\forall x)x\notin\eset}$, из которой выводится закон
	$\eset_1$: $\vdash t\notin\eset$.

	\item{}$A\subseteq B:=(\forall x)(x\in A\implies x\in B)$,
	из чего доказуемы законы

	\begin{tabular}{rl|rl}
		($\subseteq_1$) & $t\in A,A\subseteq B\vdash t\in B$                      &
		($\subseteq_2$) & $(\forall x)(x\in A\implies x\in B)\vdash A\subseteq B$
	\end{tabular}

	\item{}$A\cap B$ --- {\it пересечение} множеств $A$ и $B$. Введём аксиому:
	\[
		(\forall x)[x\in A\cap B\iff (x\in A\land x\in B)]
	\]

	Из которой доказуем закон $\cap_1$: $t\in A\cap B\dashv~\vdash t\in A\land t\in B$.

	\item{}$A\cup B$ --- {\it объединение} множеств $A$ и $B$. Введём аксиому:
	\[
		(\forall x)[x\in A\cup B\iff (x\in A\lor x\in B)]
	\]

	Из которой доказуем закон $\cup_1$: $t\in A\cap B\dashv~\vdash t\in A\lor t\in B$.
\end{enumerate}

Наконец, введём понятие равенства {\it первой аксиомой ZFC:}
\[
	(\forall X)(\forall Y)[(\forall z)(z\in X\iff  z\in Y)\implies X=Y]
\]

Тогда для множеств $A$ и $B$ справелив закон $=_{S 1}$:\\
$(\forall z)(z\in A\iff z\in B)\vdash A=B$:
\[
	\begin{nd}
		\hypo{1}{(\forall z)(z\in A\iff z\in B)}
		\have{2}{(\forall X)(\forall Y)[(\forall z)(z\in X\iff  z\in Y)\implies X=Y]}
		\by{AI}{}
		\have{3}{(\forall Y)[(\forall z)(z\in A\iff  z\in Y)\implies A=Y]}\Ae{2}
		\have{4}{(\forall z)(z\in A\iff z\in B)\implies A=B}\Ae{3}
		\have{5}{A=B}\ie{1,4}
	\end{nd}
\]

Докажем закон $=_{S 2}$: $A\subseteq B, B\subseteq A\vdash A=B$.
\[
	\begin{nd}
		\hypo{1}{A\subseteq B}
		\hypo{2}{B\subseteq A}
		\open[t]
		\hypo{6}{}
		\open
		\hypo{7}{t\in A}
		\have{8}{t\in B}\by{$\subseteq_1$}{1,7}
		\close
		\have{9}{t\in A\implies t\in B}
		\open
		\hypo{10}{t\in B}
		\have{11}{t\in A}\by{$\subseteq_1$}{2,10}
		\close
		\have{12}{t\in B\implies t\in A}
		\have{13}{t\in A\iff t\in B}\by{$\ruleEquiv_2$}{9,12}
		\close
		\have{14}{(\forall z)(z\in A\iff z\in B)}\Ai{6-13}
		\have{15}{A=B}\by{$=_{S 1}$}{14}
	\end{nd}
\]

{\it Теорема:}
Пусть $S$ --- множество, тогда $S=S\cap S$.

	{\it Доказательство:}
\[
	\begin{nd}
		\hypo{1}{}
		\open[t]
		\hypo{3}{}
		\open
		\hypo{4}{t\in S}
		\have{5}{t\in S\land t\in S}\ai{4}
		\have{6}{t\in S\cap S}\by{$\cap_1$}{5}
		\close
		\have{7}{t\in S\implies t\in S\cap S}\ii{4-6}
		\open
		\hypo{8}{t\in S\cap S}
		\have{9}{t\in S\land t\in S}\by{$\cap_1$}{7}
		\have{10}{t\in S}\ae{9}
		\close
		\have{11}{t\in S\cap S\implies t\in S}
		\have{12}{t\in S\iff t\in S\cap S}\by{$\ruleEquiv_2$}{7,11}
		\close
		\have{13}{(\forall x)(x\in S\iff x\in S\cap S)}\Ai{3-12}
		\have{14}{S=S\cap S}\by{$=_{S 1}$}{13}
	\end{nd}
\]

\pagebreak

Докажем теорему, обещанную в самом начале главы.

{\it Теорема:} Пусть $A$, $B$ --- множества, причём
\[
	(\forall x)(x\in A\iff x=a),\qquad B\subseteq A,
\]
тогда $B=\eset\lor B=A$.

Для начала докажем {\it леммы}\footnote{Лемма --- теорема, созданная
	для доказательства теоремы. Часто большие доказательства разбиваются
	на леммы для понятности.}:
\[
	\begin{array}{l}
		\text{(LM1)}~\lnot(\exists x:x\in B)\vdash B=\eset \\
		\begin{nd}
			\hypo{4}{\lnot(\exists x:x\in B)}
			\have{5}{(\forall x)~x\notin B}\by{$\lnot\forall$}{4}
			\open[t]
			\hypo{6}{}
			\open
			\hypo{7}{t\in B}
			\have{8}{t\notin B}\Ae{5}
			\have{9}{\bot}\bi{7,8}
			\have{10}{t\in \eset}\be{9}
			\close
			\have{11}{t\in B\implies t\in \eset}\ii{7-10}
			\open
			\hypo{12}{t\in\eset}
			\have{13}{t\notin\eset}\by{$\eset_1$}{}
			\have{14}{\bot}\bi{12,13}
			\have{15}{t\in B}\be{14}
			\close
			\have{16}{t\in \eset\implies t\in B}\ii{12-15}
			\have{17}{t\in B\iff t\in \eset}\by{$\ruleEquiv_2$}{11,16}
			\close
			\have{18}{(\forall x)(x\in B\iff x\in \eset)}\Ai{6-17}
			\have{19}{B=\eset}\by{$=_{S 1}$}{18}
		\end{nd}
	\end{array}
\]
\[
	\begin{array}{l}
		\text{(LM2)}~[(\forall x)(x\in A\iff x=a)],[t\in A]\vdash t=a \\
		\begin{nd}
			\hypo{1}{(\forall x)(x\in A\iff x=a)}
			\hypo{2}{t\in A}
			\have{3}{t\in A\iff t=a}\Ai{1}
			\have{4}{t=a}\by{$\ruleEquiv_3$}{2,3}
		\end{nd}
	\end{array}
\]

\pagebreak

Наконец, докажем теорему:
\[
	\begin{nd}
		\hypo{1}{(\forall x)(x\in A\iff x=a)}
		\hypo{2}{B\subseteq A}
		\have{3}{(\exists x:x\in B)\lor\lnot(\exists x:x\in B)}\by{EM}{}
		\open
		\hypo{4}{\lnot(\exists x:x\in B)}
		\have{5}{B=\eset}\by{LM1}{4}
		\have{6}{B=\eset\lor B=A}\oi{5}
		\close
		\open
		\hypo{7}{\exists x:x\in B}
		\open[t]
		\hypo{8}{t\in B}
		\have{9}{t\in A}\by{$\subseteq_2$}{2,8}
		\have{11}{t=a}\by{LM2}{1,9}
		\have{12}{a\in B}\by{$=$E}{8,11}
		\close
		\have{13}{a\in B}\Ee{7,8-12}
		\open[t]
		\hypo{14}{}
		\open
		\hypo{15}{t\in A}
		\have{16}{t=a}\by{LM2}{1,15}
		\have{17}{t\in B}\by{$=$E}{13,16}
		\close
		\have{18}{t\in A\implies t\in B}\ii{15-17}
		\close
		\have{19}{(\forall x)(x\in A\implies x\in B)}\Ai{14-18}
		\have{20}{A\subseteq B}\by{$\subseteq_2$}{19}
		\have{21}{B=A}\by{$=_{S 2}$}{2,20}
		\have{22}{B=\eset\lor B=A}\oi{21}
		\close
		\have{23}{B=\eset\lor B=A}\oe{3,4-6,7-22}
	\end{nd}
\]

Обычно в математике в качестве доказательств используются рассуждения
из цепочек импликаций\footnote{
	Например, рассуждения
	\[
		\begin{aligned}
			x\in A
			 & \implies x\in A\lor x\in B\implies \\
			 & \implies x\in A\cup B
		\end{aligned}
	\]

	доказывают $A\subseteq A\cup B$.
} (их можно перевести в диаграммы Фитча),
потому что невероятный факт, что единичное множество имеет два подмножества доказывается
с помощью диаграмм Фитча в 2 страницы.

{\it Упражнения:}
\begin{enumerate}
	\item{}Получить словесные рассуждение из последнего доказательства и привести их
	к человеческому виду.
	\item{}Почему условие ${(\forall x)(x\in A\iff x=a)}$ означает $A=\{a\}$?
\end{enumerate}
