\part{Элементы теории множеств}

\section{Род множества и элемента множества}

 {\it Множество} --- неопределимое понятие в рамках формальной системы,
обычно оно интерпретируется как совокупность объектов,
называемых {\it элементами множества}\footnote{
	Новые роды выражений, как и значения в первой главе,
	являются неопределимыми понятиями в рамках
	формальной системы.

	Их смысл проявляется через аксиомы и правила вывода.
}.

Введём в алфавит знак $\in$.
Введём новые роды выражений: множество и элемент множества $S$.
Вспомним, что выражение может принадлежать к нескольким родам.

Правила определения рода выражения:
\begin{enumerate}
	\item{}Если род выражения $A$ --- множество, а выражение $x$
	имеет род, то $x\in A$ --- формула.
	$x\in A$ читают как ``$x$ --- элемент $A$''.
	\item{}Если $x\in A$, то род выражения $x$ --- элемент множества $A$.
\end{enumerate}

\newcommand\N{\mathbb N}
Эти правила сильно упрощают дальнейшее определение понятия рода.
Пусть $\N$ --- множество
натуральных чисел, причём $1,2,...\in\N$ и выполняется
аксиома $(\forall a,b\in\N)~a+b\in\N$.
Тогда, определив род ``натуральное число'' как род ``элемент $\N$'', получим
\begin{enumerate}
	\item{}Род выражений $1,2,...$ --- натуральное число.
	\item{}Если выражения $a$, $b$ --- натуральные числа,
	то род выражения $a+b$ --- натуральное число.
\end{enumerate}

Пусть $P$ --- утверждение о $x$ и род $x$ --- элемент $S$, тогда
формулы с кванторами часто записывают так:
\[
	(\forall x\in S)~P\qquad \exists x\in S:P
\]
То есть род переменной указывается, а не подразумевается.
Они читаются как ``$P$ для произвольного $x$ из $S$'' и
``существует $x$ в $S$ такой, что $P$'' соответственно.

\section{Равенство множеств и операции на множествах}

Введём операции на множествах.
\begin{enumerate}
	\item{}Пересечение множеств $A$ и $B$ --- такое множество $A\cap B$, что
	\[
		(\forall x)~x\in A\cap B\iff x\in A\land x\in B,
	\]

	\item{}Объединение множеств $A$ и $B$ --- такое множество $A\cup B$, что
	\[
		(\forall x)~x\in A\cup B\iff x\in A\lor x\in B
	\]

	\item{}Для множеств $A$ и $B$ определена формула $A\subseteq B$, причём
	\[
		A\subseteq B\iff (\forall x)(x\in A\implies x\in B)
	\]
\end{enumerate}

Более формально их можно ввести следующим образом:
\begin{enumerate}
	\item{}Введём в алфавит символы $\cap$, $\cup$ и $\subseteq$.
	\item{}Расширим понятие формулы и рода: Если $A$ и $B$ --- множества,
	то  $A\cap B$ и $A\cup B$ --- множества, а $A\subseteq B$ --- формула.
	\item{}Введём следующие аксиомы (схемы аксиом) для множеств $A,B$.
	\begin{enumerate}
		\item{}$(\forall x)(x\in A\cap B\iff x\in A\land x\in B)$
		\item{}$(\forall x)(x\in A\cup B\iff x\in A\lor x\in B)$
		\item{}$A\subseteq B \iff (\forall x)(x\in A\implies x\in B)$
	\end{enumerate}
\end{enumerate}

Введём понятие равенства множеств первой аксиомой ZF\footnote{
	ZF (Zermelo-Fraenkel) --- набор аксиом Цермело-Френкеля, обычно с ними также
	используется аксиома выбора. Аксиомы ZF с аксиомой выбора
	обозначаются как ZFC (ZF, axiom of Choice).
}:
\[
	(\forall x)(\forall y)[(\forall z)(z\in x\iff z\in y)\implies x=y]
\]
или используя другие обозначения и как схему аксиом:
\[
	A\subseteq B\land B\subseteq A\implies A=B,
\]
где $A,B$ --- множества. Часто аксиомы формулируются как формулы без
свободных переменных, чтобы избежать схем аксиом.
Формулы без свободных переменных называются {\it замкнутыми}.

{\it Теорема:} пусть $S$ --- множество, тогда $S=S\cap S$

{\it Доказательство:}
\[
	t\in S\implies t\in S\land t\in S\implies t\in S\cap S
\]

Тогда $(\forall x)(x\in S\implies x\in S\cap S)$ и $S\subseteq S\cap S$.
\[
	t\in S\cap S\implies t\in S\land t\in S\implies t\in S
\]

Тогда $(\forall x)(x\in S\cap S\implies x\in S)$ и $S\cap S\subseteq S$.

Тогда $S=S\cap S$ по первой аксиоме ZFC.\qed

{\it Теорема:} пусть $R,S$ --- множества и $R\subseteq S$, тогда
\[
	R\cap S= R,\qquad R\cup S= S
\]

{\it Доказательство:}

Пусть $R\subseteq S$.
\[
	t\in R\cap S\iff t\in R\land t\in S\xLeftrightarrow{t\in R\implies t\in S} t\in R
\]

Тогда $(\forall x)(x\in R\cap S\iff x\in R)$ и $R\cap S=R$.
\[
	t\in R\cup S\iff t\in R\lor t\in S\iff t\in S
\]

Тогда $R\cap S=R$.\qed

Последнее доказательство использует цепочки эквивалентности:
цепочка $A_1\iff...\iff A_{n}$ означает, что $A_1\iff A_2$, ... и
$A_{n-1}\iff A_{n}$, и доказывает $A_1\iff A_{n}$. Иначе пришлось
бы составлять две цепочки импликаций.

\pagebreak

\newcommand\eset{\varnothing}
{\it Упражнения:}
\begin{enumerate}
	\item{}Доказать следующие утверждения для множеств $R,S,T$
	\begin{fullwidth}
		\begin{multicols}{2}
			\begin{enumerate}
				\item{}$S\cup S=S$
				\item{}$R\cap (S\cup T)=(R\cap S)\cup (R\cap T)$
				\item{}$S\cap (S\cup T)=S\cup (S\cap T)=S$
				\item{}$R\subseteq T\implies R\cup (S\cap T)=(R\cup S)\cap T$
			\end{enumerate}
		\end{multicols}
	\end{fullwidth}

	\item{}Доказать, что для множеств $A$ и $B$ следующие утверждения эквивалентны
	(любые два утверждения эквивалентны).
	\[
		A\subseteq B\qquad A\cup B=B\qquad A\cap B=A
		\qquad (\forall x)~\lnot(x\in A\setminus B)
	\]
\end{enumerate}

\section{Конструкция из множеств}

Введём новые обозначения. Пусть $a_1,...,a_{n}$ --- выражения с родом,
тогда $\{a_{1},...,a_{n}\}$ --- такое множество, что
\[
	(\forall x)~x\in \{a_1,...,a_{n}\}\iff (x=a_1\lor...\lor x=a_{n})
\]

{\it Пустое множество} $\eset$ --- такое множество, что
\[
	(\forall x)~\lnot(x\in \eset)
\]

На основе теории множеств можно построить многие другие области математики.
Например, натуральные числа можно определить следующим образом:
\[
	\begin{array}{llll}
		0  :=\eset\qquad & 1  :=\{\eset\}\qquad & 2  :=\{\eset,\{\eset\}\}\qquad & ... \\
		                 & 1:=\{0\}\qquad       & 2:=\{0,1\}
		                 & n:=\{0,1,...,n-1\}
	\end{array}
\]

Другими словами, $0:=\eset$ и $\sigma(x):=x\cup \{x\}$ --- функция следования,
выдающая следующее натуральное число после $x$. Заметим, что количество элементов
в множестве $n$ равно $n$.

Часто в математике нужно понятие {\it упорядоченного множества}:
множества, в котором порядок элементов имеет значение. Его элементы записываются
в круглых скобках. С помощью множеств их можно определить следующим образом:
\[
	(x,y):=\{x,\{x,y\}\}
\]
Очевидно, выполняется неравенство $(x,y)\neq (y,x)$\footnote{
	$(a\nprec b):=\lnot(a\prec b)$, где вместо $\prec$ могут быть знаки
	$<$, $>$, $\leq$, $\in$ и прочие.
}.

{\it Упражнения:}
\begin{enumerate}
	\item{}Как можно определить упорядоченные множества из трёх элементов?
	Из одного элемента? Из $n$ элементов?

	\item{}Обосновать определения пустого множества, множества $\{a_1,...,a_{n}\}$.

	\item{}*Возьмём совокупность множеств $U=\{U_{1},U_{2},...\}$.
	Как можно определить объединение, пересечение всех множеств в совокупности?
	Как это сделать, если $U$ бесконечно?
\end{enumerate}

\pagebreak

\part{В защиту интуиции}

Нельзя забывать и о ценности интуитивного понимания формул.
Оно легче укладывается в голове\footnote{Например, диаграммы
	Эйлера-Венна помогают понять и запомнить операции и теоремы, связанные с
	множествами, но их нельзя приводить как доказательства.},
оно может натолкнуть на формальное доказательство. Поэтому я привожу пример
некоторых словесных интерпретаций формул:
\begin{table}
	\centering
	\begin{tabular}{r|l}
		$\exists k:(\forall n>k)~P$     & $P$ начиная с какого-то $n$       \\
		                                & $P$ для достаточно больших $n$    \\[1em]
		$\exists \delta:\forall x~
		\big(|x|<\delta\implies P\big)$ & $P$ для достаточно малых $x$      \\[1em]
		$\forall \varepsilon~
		\exists \delta:P$               & Для всякого $\varepsilon$ можно   \\
		                                & подобрать такой $\delta$, что $P$
	\end{tabular}
	\caption{Интерпретации формул}\label{table:formula_interp}
\end{table}

{\it Упражнения:}
\begin{enumerate}
	\item{}Обосновать интерпретации в таблице~\ref{table:formula_interp}.
	\item{}Взять лист бумаги и записать
	все правила, составляющую формальную систему, то есть построить её с нуля.
\end{enumerate}

% Важно понимать, что \textsc{математические объекты абстрактны
%   и не имеют связи с действительностью.}
% ${1+1=2}$ не потому что ``если взять одно и одно яблоко, то будет два яблока'', а
% потому что мы определили $1$, $2$, $+$ и $=$ таким образом, что $1+1=2$\footnote{
%   Вспомните, почему истина в формальной системе называется тавтологией.
% }.
