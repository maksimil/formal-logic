\part{Элементы теории множеств}

{\it Множество} --- неопределимое понятие в рамках формальной системы,
обычно оно интерпретируется как совокупность объектов,
называемых {\it элементами множества}.
В теории множеств каждый терм --- множество.

Введём в алфавит бинарный предикат $\in$. Формула ${x\in S}$ читается
как ``$x$ --- элемент $S$'' или ``$x$ {\it лежит} в $S$''.

Пусть $P$ --- утверждение о $x$. Можно сказать, что $P$ описывает какое-то
свойство $x$.
Формулы, описывающие свойство каждого элемента множества или
существование элемента с каким-то свойством обычно записывают в сокращённом виде:
\[
	(\forall x\in S)~P\qquad \exists x\in S:P\qquad \exists!x\in S:P
\]
Они читаются как ``$P$ для произвольного $x$ из $S$'' и
``существует (единственный) $x$ в $S$ такой, что $P$'' соответственно.

Например,
\[
	(\forall n\in\N)~\exists m\in\N:n+1=m
\]
читается как ``для любого натурального числа $n$ существует натуральное число
$m$ такое, что $n+1=m$''.

Важно понимать, что единственный новый знак, который вводится в теории множеств
это $\in$. Операции пересечения и объединения, понятие подмножества, пустое множество,
множество $\{a\}$ нужно будет сформулировать через него.

ZF (Zermelo-Fraenkel) --- набор аксиом Цермело-Френкеля, обычно с ними также
используется Аксиома Выбора. Аксиомы ZF с Аксиомой Выбора
обозначаются как ZFC (ZF, axiom of Choice).
ZFC --- одна из наиболее широко используемых систем аксиом теории множеств.

\section{Равенство множеств и операции на множествах}

Введём новый бинарный предикатный знак $\subseteq$.
\[
	a\subseteq b\equiv \forall x(x\in a\implies x\in b)
\]
${a\subseteq b}$ читается как $a$ --- подмножество $b$, то есть все
элементы множества $a$ лежат в $b$.

Введём первую аксиому ZFC:
\[
	(\forall x)(\forall y)[(\forall z)(z\in x\iff z\in y)\implies x=y]
\]
или другими словами
\[
	\forall x\forall y[x\subseteq y\land y\subseteq x\implies x=y]
\]

\textsc{Такие аксиомы формализуют, что мы понимаем под равенством объектов.}
Например, для векторов равенство можно ввести аксиомой
$(\forall \vec{a},\vec{b})~
	\big[\vec{a}\upuparrows\vec{b}\land|\vec{a}|=|\vec{b}|\big]
	\implies \vec{a}=\vec{b}$.

Из аксиом ZFC следуют\footnote{
	Данная книга не будет углубляться в сами аксиомы ZFC, но я сильно реккомендую
	ознакомиться с ними самостоятельно.
} формулы
\[
	\forall x\forall y\exists z:\forall w(w\in z\iff w\in x\land w\in y)
\]
\[
	\forall x\forall y\exists z:\forall w(w\in z\iff w\in x\lor w\in y)
\]

Единственность можно доказать следующим образом:
\[
	w\in z_1\iff w\in x\land w\in y\iff w\in z_2,
\]
тогда $z_1=z_2$ по первой аксиоме.

Можем ввести бинарные функциональные знаки $\cap$, $\cup$ и аксиомы
\[
	\forall x\forall y\forall w(w\in x\cap y\iff w\in x\land w\in y)
\]
\[
	\forall x\forall y\forall w(w\in x\cup y\iff w\in x\lor w\in y)
\]
$A\cap B$ называется {\it пересечением} $A$ и $B$,
$A\cup B$ --- их {\it объединением}.

\vspace{1em}
{\it Теорема:} ${\forall R\forall S(R\subseteq S\implies R\cap S=R\land R\cup S=S)}$.
Обычно формулировки таких теорем расписывают более подробно:
``Пусть $R,S$ --- множества и $R\subseteq S$, тогда $R\cap S=R$ и $R\cup S=S$''.

{\it Доказательство:}

Пусть $R\subseteq S$.
\[
	t\in R\cap S\iff t\in R\land t\in S\xLeftrightarrow{t\in R\implies t\in S} t\in R
\]

Тогда $(\forall x)(x\in R\cap S\iff x\in R)$ и $R\cap S=R$.
\[
	t\in R\cup S\iff t\in R\lor t\in S\iff t\in S
\]

Тогда $R\cup S=S$.\qed

% \pagebreak

\newcommand\eset{\varnothing}
\vspace{1em}
{\it Упражнения:}
\begin{enumerate}
	\item{}Доказать эквивалентность двух формулировок первой аксиомы:
	\[
		\forall x\forall y(\forall z(z\in x\iff z\in y)\implies x=y)
	\]
	\[
		\forall x\forall y(x\subseteq y\land y\subseteq x\implies x=y)
	\]

	\item{}Доказать теоремы
	\begin{fullwidth}
		\begin{multicols}{2}
			\begin{enumerate}
				\item{}$\forall S(S\cup S=S)$
				\item{}$\forall R\forall S\forall T[R\cap (S\cup T)=(R\cap S)\cup (R\cap T)]$
				\item{}$\forall S\forall T(S\cap (S\cup T)=S\cup (S\cap T)=S)$
				\item{}$\forall R\forall S\forall T[R\subseteq T
							\implies R\cup (S\cap T)=(R\cup S)\cap T]$
			\end{enumerate}
		\end{multicols}
	\end{fullwidth}

	\item{}Доказать, что для множеств $A$ и $B$ следующие утверждения эквивалентны
	(любые два утверждения эквивалентны).
	\[
		A\subseteq B\qquad A\cup B=B\qquad A\cap B=A
		\qquad (\forall x)~\lnot(x\in A\setminus B)
	\]
\end{enumerate}

\section{Конструкции из множеств}

Из аксиом ZFC следует, что для $n>0$ справедливо
\[
	\forall a_1...\forall a_{n}\exists !A:\forall w
		[w\in A\iff (w=a_1\lor...\lor w=a_{n})],
\]
тогда введём $n$-арный функциональный знак\footnote{
	То есть мы вводим функциональный знак для каждого натурального числа.
} $\{,\}$ и аксиому
\[
	\forall a_1...\forall a_{n}\forall w
	[w\in \{a_1,a_2,...,a_{n}\}\iff (w=a_1\lor ...\lor w=a_{n})]
\]

Из ZFC следует ${\exists !e:\forall x(x\notin e)}$\footnote{
	см. упражнение~\ref{ex:eset_only}.
}, можем ввести константу $\eset$ и аксиому $\forall x(x\notin \eset)$.
$\eset$ называется {\it пустым множеством}.

На основе теории множеств можно построить многие другие области математики.
Например, натуральные числа можно определить следующим образом:
\[ \begin{array}{llll}
		0 :=\eset\qquad & 1  :=\{\eset\}\qquad & 2  :=\{\eset,\{\eset\}\}\qquad & ... \\
		                & 1:=\{0\}\qquad       & 2:=\{0,1\}
		                & n:=\{0,1,...,n-1\}
	\end{array}
\]

Другими словами, ${0:=\eset}$ и $\sigma(x):=x\cup \{x\}$ --- функция следования,
возвращающая следующее натуральное число после $x$. Заметим, что количество элементов
в множестве $n$ равно $n$. Из аксиом ZFC следует как существование множества таких
натуральных чисел, так и принцип индукции.

Часто в математике нужно понятие {\it упорядоченного множества}~---~множества,
в котором определён порядок элементов. Порядок элементов
конечного упорядоченного множества можно
определить, просто записав его элементы в каком-то порядке.
Упорядоченное множество из двух элементов (упорядоченную пару)
можно определить так:
\[
	(x,y):=\{x,\{x,y\}\}
\]

Рассмотрим задачу по-интереснее: как определить пересечение и объединение
всех множеств в множестве $U=\{u_1,u_2,...\}$, где $U$ может быть бесконечно?
Введём (из аксиом ZFC следуют необходимые формулы) унарные $\cup$, $\cap$
и аксиомы
\[
	\forall U\forall x(x\in \cap U\iff \forall u(u\in U\implies x\in u))
\]
\[
	\forall U\forall x(x\in \cup U\iff \exists u(u\in U\land x\in u))
\]

\vspace{1em}
{\it Упражнения:}
\begin{enumerate}
	\item{}Как можно определить упорядоченные множества из трёх элементов?
	Из одного элемента? Из $n$ элементов?

	\item{}Доказать теоремы
	\begin{enumerate}
		\item{}${\exists e:\forall x(x\notin e)\implies
					\exists!e:\forall x(x\notin e)}$ \label{ex:eset_only}
		\item{}$\forall x\forall y[(x,y)\neq (y,x)]$
	\end{enumerate}
	\item{}Объяснить определения унарных $\cup$ и $\cap$.
\end{enumerate}

\pagebreak
