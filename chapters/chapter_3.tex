\part{Элементы теории множеств}

\section{Род множества и элемента множества}

 {\it Множество} --- неопределимое понятие в рамках формальной системы,
обычно оно интерпретируется как совокупность объектов,
называемых {\it элементами множества}\footnote{
	Новые роды выражений, как и значения в первой главе,
	являются неопределимыми понятиями в рамках
	формальной системы.

	Их смысл проявляется через аксиомы и правила вывода.
}.

Введём в алфавит символ $\in$.
Введём новые роды выражений: множество и элемент множества $S$.
Вспомним, что выражение может принадлежать к нескольким родам.

Правила определения рода выражения:
\begin{enumerate}
	\item{}Если род выражения $A$ --- множество, а выражение $x$
	имеет род, то $x\in A$ --- формула.
	$x\in A$ читают как ``$x$ --- элемент $A$''.
	\item{}Если $x\in A$, то род выражения $x$ --- элемент множества $A$.
\end{enumerate}

\newcommand\N{\mathbb N}
Эти правила сильно упрощают дальнейшее определение понятия рода.
Пусть $\N$ --- множество
натуральных чисел, $+$ --- операция сложения, причём выполняется
аксиома: $(\forall a,b\in\N)~a+b\in\N$
и $1,2,...\in\N$. Тогда, определив род ``натуральное число'' как род ``элемент $\N$'',
получим
\begin{enumerate}
	\item{}Род выражений $1,2,...$ --- натуральное число.
	\item{}Если выражения $a$, $b$ --- натуральные числа,
	то род выражения $a+b$ --- натуральное число.
\end{enumerate}

\section{Равенство множеств и операции на множествах}

Введём операции на множествах.
\begin{enumerate}
	\item{}Пересечение множеств $A$ и $B$ --- такое множество $A\cap B$, что
	\[
		(\forall x)~x\in A\cap B\iff x\in A\land x\in B,
	\]

	где $[p\iff q]=[(p\implies q)\land (q\implies p)]$.

	\item{}Объединение множеств $A$ и $B$ --- такое множество $A\cup B$, что
	\[
		(\forall x)~x\in A\cup B\iff x\in A\lor x\in B
	\]

	\item{}Для множеств $A$ и $B$ определена формула $A\subseteq B$, причём
	\[
		A\subseteq B\iff (\forall x)(x\in A\implies x\in B)
	\]
\end{enumerate}

Более формально новые обозначения можно ввести аксиомами.

\pagebreak

Для начала введём $\iff$.

\begin{enumerate}
	\item{}Введём в алфавит символ $\iff$.
	\item{}Расширим понятие формулы: если $A,B$ --- формулы, то\\${A\iff B}$ --- формула.
		\item{}Введём аксиому:
		\[
			[A\iff B]=[(A\implies B)\land (B\implies A)],
		\]

		где $A$, $B$ --- формулы.
		Такие аксиомы называются {\it схемами аксиом},
		потому что они содерждат аксиому для каждой возможной формулы $A$ и $B$.
\end{enumerate}

Теперь введём $\cap$, $\cup$ и $\subseteq$.

\begin{enumerate}
	\item{}Введём в алфавит символы $\cap$, $\cup$ и $\subseteq$.
	\item{}Расширим понятие формулы и рода.
	\begin{enumerate}
		\item{}Если $A$, $B$ --- множества, то $A\cap B$, $A\cup B$ --- множества.
		\item{}Если $A$, $B$ --- множества, то $A\subseteq B$ --- формула.
	\end{enumerate}
	\item{}Введём следующие аксиомы для множеств $A$ и $B$.
	\begin{enumerate}
		\item{}$(\forall x)(x\in A\cap B\iff x\in A\land x\in B)$
		\item{}$(\forall x)(x\in A\cup B\iff x\in A\lor x\in B)$
		\item{}$A\subseteq B \iff (\forall x)(x\in A\implies x\in B)$
	\end{enumerate}
\end{enumerate}

Введём понятие равенства множеств первой аксиомой ZF\footnote{
	ZF (Zermelo-Fraenkel) --- набор аксиом Цермело-Френкеля, обычно с ними также
	используется аксиома выбора. Аксиомы ZF с аксиомой выбора
	обозначаются как ZFC (ZF, axiom of Choice).
}:
\[
	(\forall x)(\forall y)[(\forall z)(z\in x\iff z\in y)\implies x=y]
\]

или используя другие обозначения:
\[
	A\subseteq B\land B\subseteq A\implies A=B,
\]

где $A,B$ --- множества.

{\it Теорема:} пусть $S$ --- множество, тогда $S=S\cap S$

{\it Доказательство:}
\[
	t\in S\implies t\in S\land t\in S\implies t\in S\cap S
\]

Тогда $(\forall x)(x\in S\implies x\in S\cap S)$ и $S\subseteq S\cap S$.
\[
	t\in S\cap S\implies t\in S\land t\in S\implies t\in S
\]

Тогда $(\forall x)(x\in S\cap S\implies x\in S)$ и $S\cap S\subseteq S$.

Тогда $S=S\cap S$ по первой аксиоме ZFC.\qed

{\it Теорема:} пусть $R,S$ --- множества и $R\subseteq S$, тогда
\[
	R\cap S= R,\qquad R\cup S= S
\]

{\it Доказательство:}

Пусть $R\subseteq S$.
\[
	t\in R\cap S\iff t\in R\land t\in S\iff t\in R
\]

Тогда $(\forall x)(x\in R\cap S\iff x\in R)$ и $R\cap S=R$.
\[
	t\in R\cup S\iff t\in R\lor t\in S\iff t\in S
\]

Тогда $R\cap S=R$.\qed

{\it Упражнения:}
\begin{enumerate}
	\item{}Доказать следующие утверждения для всяких множеств $R,S,T$
	\begin{enumerate}
		\item{}$S\cup S=S$
		\item{}$R\cap (S\cup T)=(R\cap S)\cup (R\cap T)$
		\item{}$S\cap (S\cup T)=S\cup (S\cap T)=S$
		\item{}$R\subseteq T\implies R\cup (S\cap T)=(R\cup S)\cap T$
	\end{enumerate}

	\item{}Доказать, что для множеств $A$ и $B$ следующие утверждения эквивалентны,
	то есть любые два из этих утверждений эквивалентны.
	\[
		A\subseteq B\qquad A\cup B=B\qquad A\cap B=A
		\qquad (\forall x)~\lnot(x\in A\setminus B)
	\]
	\item{}Обосновать определение пустого множества ($\varnothing$):
	\[
		(\forall x)\lnot(x\in \varnothing)
	\]

	Как можно схожим образом определить множества $\{a\}$, $\{a,b\}$,
	$\{a_1,a_2,...,a_{n}\}$?
	\item{}*Возьмём совокупность множеств $U=\{U_{1},U_{2},...\}$.
	Как можно определить объединение, пересечение всех множеств в совокупности?
	Как это сделать, если $U$ бесконечно?
\end{enumerate}

\pagebreak

\part{В защиту интуиции}

Нельзя забывать и о ценности интуитивного понимания формул.
Оно легче укладывается в голове\footnote{Например, диаграммы
	Эйлера-Венна помогают понять и запомнить операции и теоремы, связанные с
	множествами, но их нельзя приводить как доказательства.},
оно может натолкнуть на формальное доказательство. Поэтому я привожу пример
некоторых словесных интерпретаций логических утверждений\footnote{В качестве
	упражнения попробуйте обосновать эти интерпретации.}:
\begin{enumerate}
	\item{}$\exists k:(\forall n>k)~P(n)$ --- $P(n)$ начиная с какого-то $n$
	(для достаточно больших $n$).
	\item{}$\exists \delta:(\forall x)~|x|<\delta\implies P(x)$ --- $P(x)$ для
	достаточно малых $x$.
	\item{}$\forall \varepsilon~\exists \delta:P(\varepsilon,\delta)$ ---
	для всякого $\varepsilon$ можно подобрать такой $\delta$,\\что $P(\varepsilon,\delta)$.
\end{enumerate}

В качестве последнего упражнения я предлагаю вам взять лист бумаги и записать
все правила, составляющую формальную систему, то есть построить её с нуля.

% Важно понимать, что \textsc{математические объекты абстрактны
%   и не имеют связи с действительностью.}
% ${1+1=2}$ не потому что ``если взять одно и одно яблоко, то будет два яблока'', а
% потому что мы определили $1$, $2$, $+$ и $=$ таким образом, что $1+1=2$\footnote{
%   Вспомните, почему истина в формальной системе называется тавтологией.
% }.
