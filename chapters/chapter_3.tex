\part{Элементы теории множеств}

\begin{fullwidth}
  \begin{multicols}{2}
    \index{множество}\index{элемент!множества}\index{элемент}
    Нестрого {\it множество} можно определить как совокупность объектов,
    называемых {\it элементами множества}.
    Более строго можно сначала определить, что такое теория множеств, то есть
    определить формальную систему, после чего определить множество как
    объект теории множеств. При интерпретации формул множество интерпретируют
    как совокупность объектов.

    Рассуждения выше подчёркивают, что, определяя часть целого, часто
    необходимо сначала определить это целое. Например, можем определить целое число
    как элемент множества целых чисел, а множество целых чисел как множество
    с определённой структурой: операции сложения, умножения,
    отношение порядка и так далее. \textsc{Целое число нельзя вырвать
      из контекста струтктуры целых чисел, иначе оно потеряет весь смысл.}
    Аналогично бессмысленно определять, что такое кровеносная система,
    не определяя, что такое живой организм.
  \end{multicols}
\end{fullwidth}

\section{Основные понятия}

\index{лежать в множестве, $\in$}
Введём бинарный предикатный знак $\in$. Формула ${x\in S}$
соответствует высказыванию ``$x$ --- элемент $S$'' или
``$x$ {\it лежит} в $S$''.

Пусть $P(x)$ --- формула о $x$. Можно сказать, что $P(x)$ описывает какое-то
свойство $x$.
Формулы, описывающие свойство каждого элемента множества или
существование элемента с каким-то свойством обычно записывают в сокращённом виде:
\[
  (\forall x\in S)~P(x)\qquad
  \exists x\in S:P(x)\qquad \exists!x\in S:P(x)
\]
Их читают как ``$P(x)$ для произвольного $x$ из $S$'' и
``существует (единственный) $x$ в $S$ такой, что $P(x)$'' соответственно.

Например, пусть $\N$ --- множество натуральных чисел, тогда
\[
  \forall n\in\N~\exists m\in\N:n+1=m
\]
читают как ``для любого натурального числа $n$ существует натуральное число
$m$ такое, что $n+1=m$''.

\index{подмножество, $\subseteq$}
\index{включение, $\subseteq$}
Введём бинарный предикатный знак $\subseteq$.
\[
  a\subseteq b\iff [(\forall x)~x\in a\implies x\in b]
\]
Формулу ${a\subseteq b}$ читают как ``$a$ --- {\it подмножество} $b$'', то есть все
элементы множества $a$ лежат в $b$. Отношение $\subseteq$ также называют
{\it отношением включения}.

\index{ZF (Zermelo-Fraenkel)}\index{ZFC (ZF, axiom of Choice)}
Введём первую аксиому ZFC\footnote{
  ZF (Zermelo-Fraenkel) --- набор аксиом
  Цермело-Френкеля, обычно с ними также
  используется Аксиома Выбора. Аксиомы ZF с Аксиомой Выбора
  обозначают как ZFC (ZF, axiom of Choice).
  ZFC --- одна из наиболее широко используемых систем аксиом теории множеств.
} --- {\it Аксиому Объёмности}\index{Аксиома!Объёмности}:
\begin{equation}\label{eq:ax_ext_1}
  (\forall x,y)~[(\forall z)~z\in x\iff z\in y]\implies x=y
\end{equation}
или, другими словами
\begin{equation}\label{eq:ax_ext_2}
  (\forall x,y)~x\subseteq y\land y\subseteq x\implies x=y
\end{equation}

\textsc{Такие аксиомы формализуют, что мы понимаем под равенством объектов.}
Например, для векторов равенство можно ввести аксиомой
$(\forall \vec{a},\vec{b})~
  \vec{a}\upuparrows\vec{b}\land|\vec{a}|=|\vec{b}|
  \implies \vec{a}=\vec{b}$.

Можем сделать важнейшее наблюдение теории множеств: \textsc{Множество $S$
  определяется только необходимым и достаточным условием формулы $x\in S$,
  которое называют условием множества $S$.
  Определив такой $P(x)$, что $P(x)\iff x\in S$, мы определим множество $S$.
  Множество полностью определяется своим условием.}\index{условие множества}

Введём остальные аксиомы ZFC. Из них следуют\footnote{
  Данная книга не будет углубляться в сами аксиомы ZFC, но я сильно реккомендую
  ознакомиться с ними самостоятельно.}
формулы
\[
  \forall x,y~\exists z:(\forall w)~w\in z\iff w\in x\land w\in y
\]
\[
  \forall x,y~\exists z:(\forall w)~w\in z\iff w\in x\lor w\in y
\]
Единственность таких $z$ можно доказать следующим образом:
\[
  w\in z_1\iff w\in x\land w\in y\iff w\in z_2,
\]
тогда $z_1=z_2$ по Аксиоме Объёмности.

\index{пересечение множеств, $\cap$}
\index{объединение множеств, $\cup$}
Можем ввести бинарные функциональные знаки $\cap$, $\cup$ и аксиомы
\[
  (\forall x,y)(\forall w)~w\in x\cap y\iff w\in x\land w\in y
\]
\[
  (\forall x,y)(\forall w)~w\in x\cup y\iff w\in x\lor w\in y
\]
Множество $A\cap B$ называют {\it пересечением} множеств $A$ и $B$,
а множество $A\cup B$ --- их {\it объединением}.

Очевидно, если $P_{A}(x)$ --- условие $A$, а $P_{B}(x)$ --- условие $B$,
то $P_{A}(x)\land P_{B}(x)$ --- условие $A\cap B$,
а $P_{A}(x)\lor P_{B}(x)$ --- условие $A\cup B$.

\vspace{1em}
{\it Теорема:}
\[
  (\forall R,S)~R\subseteq S\implies R\cap S=R
\]

{\it Доказательство:}
Пусть $R\subseteq S$. Для произвольного $t$ имеем
\[
  t\in R\cap S\iff t\in R\land t\in S\xLeftrightarrow{t\in R\implies t\in S} t\in R
\]
Тогда $(\forall x)(x\in R\cap S\iff x\in R)$ и $R\cap S=R$.\qed

\newcommand\eset{\varnothing}
\vspace{1em}
{\it Упражнения:}
\begin{enumerate}
  \item{}Доказать эквивалентность \eqref{eq:ax_ext_1} и \eqref{eq:ax_ext_2}.

  \item{}Пусть $C$ --- множество, содержащее все элементы $A$, не лежащие в $B$.
  Записать необходимое и достаточное условие $x\in C$.

  \item{}Пусть $P_{A}(x), P_{B}(x),P_{C}(x)$ --- условия множеств
  $A,B,C$ соответственно. Выразить\footnote{Например,
  \[
    A\subseteq B\iff
    [(\forall x)~P_{A}(x)\implies P_{B}(x)]
  \]} через них формулы
  \[
    A=B\qquad C=A\cap B\qquad C=A\cup B\qquad C\cap A=C\cap B
  \]

  \item{}Доказать теоремы
  \begin{enumerate}
    \item{}$(\forall S)~S\cup S=S\cap S=S$
    \item{}$(\forall R,S)~R\subseteq S\implies R\cup S=S$
    \item{}$(\forall R,S,T)~R\cap (S\cup T)=(R\cap S)\cup (R\cap T)$
    \item{}$(\forall S, T)~S\cap (S\cup T)=S\cup (S\cap T)=S$
    \item{}$(\forall R,S, T)~R\subseteq T\implies R\cup (S\cap T)=(R\cup S)\cap T$
  \end{enumerate}


  \item{}Доказать, что для множеств $A$ и $B$ следующие формулы попарно эквивалентны
  (любые две формулы эквивалентны).
  \[
    A\subseteq B\qquad A\cup B=B\qquad A\cap B=A
  \]
\end{enumerate}

\section{Конструкции из множеств}

Из аксиом ZFC следует, что для положительного $n$ справедливо
\[
  \forall a_1,...,a_{n}~\exists !A:(\forall w)~
  w\in A\iff (w=a_1\lor...\lor w=a_{n}),
\]
Введём $n$-арный функциональный знак\footnote{
  То есть мы вводим функциональный знак для каждого натурального числа.}
$\{,\}$ и аксиому
\[
  (\forall a_1,...,a_{n})(\forall w)~
  w\in \{a_1,a_2,...,a_{n}\}\iff (w=a_1\lor ...\lor w=a_{n})
\]

\index{пустое множество}\index{множество!пустое}
Из ZFC следует ${\exists !e:(\forall x)~x\notin e}$,
можем ввести константу $\eset$ и аксиому $(\forall x)~x\notin \eset$.
$\eset$ называют {\it пустым множеством}.

На основе теории множеств можно построить многие другие области математики.
Например, натуральные числа можно определить следующим образом:
\[
  \begin{array}{llll}
    0 :=\eset\qquad & 1  :=\{\eset\}\qquad & 2  :=\{\eset,\{\eset\}\}\qquad & ... \\
                    & 1:=\{0\}\qquad       & 2:=\{0,1\}
                    & n:=\{0,1,...,n-1\}
  \end{array}
\]
Другими словами $\sigma(x):=x\cup \{x\}$ --- функция следования,
возвращающая следующее натуральное число. Заметим, что ``количество''\footnote{
  Понятие количества элементов конечного множества определяют
  через такие натуральные числа.} элементов
в множестве $n$ равно $n$. Из аксиом ZFC следует как существование множества
всех таких натуральных чисел (Аксиома Бесконечности), так и принцип индукции.
Множество натуральных чисел с $0$ обычно обозначают как $\N_0$,
без $0$ --- $\N$.
\index{множество!натуральных чисел!без 0, $\N$}
\index{множество!натуральных чисел!с 0, $\N_0$}

\index{множество!упорядоченное}\index{упорядоченное!множество}
\index{упорядоченное!пара}
% Часто в математике нужно понятие {\it упорядоченного множества}~---~множества,
% в котором определён порядок элементов. Порядок элементов
% конечного упорядоченного множества можно
% определить, просто записав его элементы в строку.
% Упорядоченное множество из двух элементов (упорядоченную пару)
% можно определить так:
Заметим, что $\{x,y\}=\{y,x\}$. Определим способ задания
{\it упорядоченного множества} --- множества, в котором определён порядок элементов.
Начнём с {\it упорядоченной пары}.
\[
  (x,y):=\{\{x\},\{x,y\}\}
\]
Множество
всех таких упорядоченных пар, где первый элемент лежит в $X$, а второй элемент
лежит в $Y$ называют {\it декартовым произведением}
\index{декартово произведение, $\times$} множеств $X$ и $Y$ и
обозначают $X\times Y$, то есть
\[
  (\forall z)~z\in X\times Y\iff \exists x\in X,y\in Y:z=(x,y)
\]
Декартово произведение $A\times A$ также называют {\it декартовым квадратом}
\index{декартов квадрат}
и обозначают $A^{2}:=A\times A$. Существование декартова произведения следует из ZFC.

Упорядоченную тройку и другие конечные упорядоченные множества
можно определить как
\[
  (x,y,z):=((x,y),z)\qquad (x_1,...,x_{n}):=((x_1,...,x_{n-1}),x_{n})
\]
Тогда $X\times Y\times Z=(X\times Y)\times Z$\footnote{
  Обычно в терме
  \[
    x_1\circ x_2\circ x_3\circ...\circ x_{n},
  \]
  где $\circ$ --- бинарный функциональный знак,
  скобки расставляют следующим образом:
  \[
    (((x_1\circ x_2)\circ x_3)\circ...)\circ x_{n}
  \]
  То есть разложение идёт по самому правому $\circ$.
} содержит все такие тройки.
Аналогично определяют $X_1\times X_2\times ...\times X_{n}$ и $X^{n}$.
Можем также ввести функциональные знаки $\pi_{k}^{n}$ и аксиомы
\[
  (\forall z)~[\exists x_1,...,x_{n}:z=(x_1,...,x_{n})]\implies \pi_{k}^{n}(z)=x_{k}
\]
для каждых $n\geq k\geq 1$. Так, для $\pi_{2}^{3}$ аксиома
будет выглядеть как
\[
  (\forall z)~[\exists x_1,x_2,x_3:z=(x_1,x_2,x_3)]\implies \pi_2^{3}(z)=x_2
\]

Определим теперь пересечение и объединение элементов произвольного множества $U$.
Определить это пересечение через бинарные $\cup$ и $\cap$
возмножно только для конечных\footnote{Здесь используется
  интуитивное понятие конечности.} $U$.
Определим\footnote{Из аксиом ZFC следуют необходимые формулы.} унарные $\cup$, $\cap$
через аксиомы
\[
  (\forall U)(\forall x)~x\in\cap U\iff (\forall u\in U)~x\in u
\]
\[
  (\forall U)(\forall x)~x\in\cup U\iff \exists u\in U:x\in u
\]
Заметим, что $\cup \{A,B\}=A\cup B$ и $\cap \{A,B\}=A\cap B$.
Также заметим связь между $\cap,\land$ и $\forall$ и между $\cup,\lor$ и $\exists$.

Из аксиом ZFC следует существование множества всех подмножеств множества $A$.
Его обозначают как $2^{A}$ или $\mathcal P(A)$ и определяют аксиомой
\[
  (\forall A)(\forall \alpha)~\alpha\in 2^{A}\iff \alpha\subseteq A
\]

\index{Аксиома!Выделения}\index{выделение подмножества}
\index{схема аксиом!Выделения}
Определим способ выделения подмножеств множества.
Из Аксиомы Выделения (Схемы Аксиом Выделения)
ZFC следует, что для каждой формулы $\varphi$, в которой свободны
только $x_1,...,x_{n},w,A$, справедливо
\[
  \forall x_1...., x_{n}~\forall A~\exists !B:
  (\forall w)~w\in B\iff w\in A\land \varphi,
\]
то есть существует $B$, содержащее все $w\in A$ такие, что $\varphi$.
Обычно такое построение множества $B$ записывают как
\[
  B=\left\{w\in A\;\big|\;\varphi\right\}
  % \forall w(w\in B\iff w\in A\land \varphi)
\]
и читают как ``$w\in A$ такие, что $\varphi$''.

\newcommand\Z{\mathbb Z}
Множество целых чисел обозначают как $\Z$\index{множество!целых чисел, $\Z$}.
Пустым местом будем обозначать произведение. Тогда
множество всех чётных целых чисел можно определить как
\[
  \Z_{E}:=\left\{x\in\Z\;\big|\;\exists z\in\Z:x=2z\right\}
\]

Особый случай выделения подмножества --- {\it разность множеств} $A$ и $B$,
его обозначают $A\setminus B$.
\[
  A\setminus B:=\left\{a\in A\;\big|\; a\notin B\right\}
\]
Тогда множество нечётных целых чисел можно определить как
\[
  \Z'_{E}:=\Z\setminus \Z_E
\]


Таким образом, мы определили основные способы построения множеств:
объединение, пересечение, выделение подмножества, разность.
Важно помнить их условия (см. таблицу~\ref{table:set_def}).

\vspace{1em}
{\it Упражнения:}
\begin{enumerate}
  \item{}Пусть $A=\{a,c\}$ и $B=\{b,c\}$. Какие элементы лежат в множествах
  $A\cup B$, $A\cap B$, $A\setminus B$ и $A\times B$?
  \item{}Доказать теоремы
  \begin{enumerate}
    \item{}$(\forall A,B)~\cup \{A,B\}=A\cup B$
    \item{}$(\forall A,B)~\cap \{A,B\}=A\cap B$
    \item{}$[\exists e:(\forall x)~x\notin e]\implies
      [\exists!e:(\forall x)~x\notin e]$
    \item{}$(\forall x)~\{x,x\}=\{x\}$
    \item{}$(\forall x,y)~x\neq y\iff (x,y)\neq (y,x)$
    \item{}$(\forall x_1,y_1,x_2,y_2)~
      (x_1,y_1)=(x_2,y_2)\iff (x_1=x_2\land y_1=y_2)$
    \item{}$(\forall A,B)~A\setminus B=\eset\iff A\subseteq B$
    \item{}$(\forall A,B)(\forall a,b)~
      (a,b)\in A\times B\iff a\in A\land b\in B$
    \item{}$(\forall A,B,C,D)~
      (A\times B)\cap (C\times D)=(A\cap C)\times (B\cap D)$
  \end{enumerate}
  \item{}Объяснить определения унарных $\cup$ и $\cap$.
  В чём заключается связь между союзами $\land$, $\forall$, $\lor$, $\exists$?
  В чём заключается связь между ``и'', ``для каждого'', ``или'', ``существует''?
  \item{}{\it Симметричную разность}\index{симметричная разность}
  множеств $A$ и $B$ определяют как
  \[
    A\Delta B:=(A\setminus B)\cup (B\setminus A)
  \]
  \begin{enumerate}
    \item{}Найти условие $A\Delta B$.
    \item{}Доказать, что для произвольных $A$ и $B$ справедливы
    \begin{enumerate}
      \item{}$A\Delta B=(A\cup B)\setminus (A\cap B)$
      \item{}$A\Delta B=B\Delta A$
    \end{enumerate}
    \item{}Пусть $A=\{a,c\}$ и $B=\{b,c\}$. Какие элементы лежат в $A\Delta B$?
  \end{enumerate}
\end{enumerate}
