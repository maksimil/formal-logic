\part{Примеры формальных систем}

\section{Элементы теории множеств}

Рассмотрим примеры доказательств в теории множеств.

{\it Множество} --- совокупность значений, называемых {\it элементами множества}.
Логическое утверждение $x\in S$ означает ``$x$ --- элемент множества $S$''.

Введём новые значения переменных: все множества $S$, для которых для
всякого значения $x$ логическое утверждение ${x\in S}$ имеет определённое
значение\footnote{Существуют и другие способы определить
	понятие множества, например аксиомы ZFC. Я сильно реккомендую
	хотя бы ознакомиться с ними после прочтения книги.}.
Если $S$ содержит только элементы $a_1,a_2,...$, то пишут $S=\{a_1,a_2,...\}$.
$\varnothing:=\{\}$.

Определим следующие понятия для множеств\footnote{Заметьте схожесть между операциями
	над множествами и логическими операциями.}:
\begin{enumerate}
	\item{}$A$ --- подмножество $B$ (${A\subseteq B}$ или $B\supseteq A$) означает
	\[
		(\forall x)~x\in A\implies x\in B
	\]

	\item{}Множества $A$ и $B$ равны ($A=B$) тогда и только тогда, когда
	$A\subseteq B\land B\subseteq A$.
	То есть \textsc{множество определяется его элементами и ничем более}\footnote{
		В целом, если для двух значений определено понятие равненства,
		то имеется в виду, что значение такого рода определяется только этими
		свойствами.

		Например, следующее утверждение для векторов:
		\[
			\vec{a}=\vec{b}\iff\left[|\vec{a}|=|\vec{b}|\land
				\vec{a}\upuparrows\vec{b}\right]
		\]

		Означает, что вектор определяется длиной (модулем) и направлением.
	}.

	\item{}Пересечение множеств $A$ и $B$ --- такое множество $A\cap B$, что
	\[
		(\forall x)~x\in A\cap B\iff(x\in A\land x\in B)
	\]

	\item{}Объединение множеств $A$ и $B$ --- такое множество $A\cup B$, что
	\[
		(\forall x)~x\in A\cup B\iff (x\in A\lor x\in B)
	\]

	\item{}Разность множеств $A$ и $B$ --- такое множество $A\setminus B$, что
	\[
		(\forall x)~x\in A\setminus B\iff (x\in A\land \lnot(x\in B))
	\]
\end{enumerate}

{\it Теорема:} если $S$ --- множество, то $S\cap S=S$\footnote{
	Технически, ``если $S$ --- множество'' можно опустить, но если
	мы введём $\cap$ например, для чисел, то это уточнение станет необходимым.
}.

{\it Доказательство:}

$S\cap S=S$ тогда и только
тогда, когда ${(\forall x)(x\in S\cap S\iff x\in S)}$,
значит это утверждение и нужно доказать.

Возьмём произвольное $t$. Предположим $t\in S\cap S$.
\[
	t\in S\cap S\implies (t\in S\land t\in S)\implies t\in S
\]

То есть $t\in S\cap S\implies t\in S$\footnote{Используется
	тавтология
	\[
		[p\implies q]\land[q\implies t]\implies [p\implies t]
	\]

	Думаю вы уже со школы знакомы с цепочками импликаций.
}.

Предположим $t\in S$.
\[
	t\in S\implies (t\in S\land t\in S)\implies t\in S\cap S
\]

То есть $t\in S\implies t\in S\cap S$ и $t\in S\iff t\in S\cap S$.

По \Aii{} можем вывести $(\forall x)(x\in S\iff x\in S\cap S)$.\qed

\pagebreak

У большинства доказательств следующие шаги:
\begin{enumerate}
	\item{}Раскрыть определения, чтобы выяснить, какое утверждение нужно доказать.
	\item{}Применить логические операции и преобразования, чтобы вывести это утверждение.
	\item{}Если застряли, попробовать доказать от обратного.
\end{enumerate}

{\it Упражнения:}
\begin{enumerate}
	\item{}Доказать следующие утверждения для всяких множеств $R,S,T$
	\begin{enumerate}
		\item{}$S\cup S=S$
		\item{}$R\cap (S\cup T)=(R\cap S)\cup (R\cap T)$
		\item{}$S\cap (S\cup T)=S\cup (S\cap T)=S$
		\item{}$R\subseteq T\implies R\cup (S\cap T)=(R\cup S)\cap T$
	\end{enumerate}

	\item{}Доказать, что для множеств $A$ и $B$ следующие утверждения эквивалентны:
	\[
		A\subseteq B\qquad A\cup B=B\qquad A\cap B=A
		\qquad (\forall x)~\lnot(x\in A\setminus B)
	\]
	\item{}Обосновать альтернативное определение пустого множества:
	\[
		(\forall x)\lnot(x\in \varnothing)
	\]

	Как можно схожим образом определить множества $\{a\}$, $\{a,b\}$,
	$\{a_1,a_2,...,a_{n}\}$?
	\item{}*Возьмём совокупность множеств $U=\{U_{1},U_{2},...\}$.
	Как можно определить объединение, пересечение всех множеств в совокупности?
	Как это сделать, если $U$ бесконечно?
\end{enumerate}

\section{Аксиомы Пеано}

Рассмотрим пример системы\footnote{
	Система, совокупность, класс и набор --- синонимы, а множество --- более формальное
	понятие, обычно определённое аксиомами ZFC.
} аксиом, определяющей натуральные числа: {\it аксиомы Пеано}.

\newcommand\N{\mathbb{N}}
Множество $\N$ с функцией следования $\sigma$\footnote{$\sigma(x)$ нельзя определить
	как $x+1$, потому что операция сложения не была введена.},
определённой для всякого $x\in\N$
называется {\it множеством натуральных чисел}, если для него
выполняются {\it Аксиомы Пеано}:
\begin{enumerate}
	\item{}$1\in \N$
	\item{}$x\in\N\implies \sigma(x)\in\N$
	\item{}$(\forall x\in\N)~\sigma(x)\neq 1$
	\item{}$(\forall a,b\in\N)(\sigma(a)=\sigma(b)\implies a=b)$
	\item{}$P(1)\land [(\forall n\in\N)~(P(n)\implies P(\sigma(n)))]
		\implies(\forall n\in\N)~P(n)$\footnote{На этой аксиоме основаны
		доказательства по индукции.}
\end{enumerate}

\pagebreak

Введём существование такого множества как аксиому, а
все его элементы как
возможные значения переменных\footnote{Множество натуральных чисел можно
	получить и из множеств:
	\[
		1:=\{\varnothing\}\qquad\sigma(x):=x\cup \{x\}
	\]
	\[
		\N:=\{1,\sigma(1),...\}
	\]

	Существование такого множества следует из
	аксиомы бесконечности ZFC (см. в интернете).}.

Обозначим за $F(n)$ следующее утверждение: ${n\in\N}$ можно представить как
$n=\sigma(\sigma(...\sigma(1)))$, причём это выражение конечно.

{\it Теорема:} $(\forall n\in\N)~F(n)$.

	{\it Доказательство:}

$1\in\N$ можно представить как $1$, значит $F(1)$.

Возьмём произвольное $n$ и предположим $F(n)$, тогда $\sigma(n)$ тоже
представимо в виде $\sigma(\sigma(...\sigma(1)))$, то есть $F(\sigma(n))$.

По \Aii{} можем вывести $(\forall n\in\N)(F(n)\implies F(\sigma(n)))$.

Применим аксиому $4$ и выведем $(\forall n\in\N)~F(n)$.\qed

{\it Упражнения:}
\begin{enumerate}
	\item{}Доказать $(\forall n\in\N)~1<n\lor n=1$, если $<$ определено как
	\begin{enumerate}
		\item{}$(\forall a\in\N)~a<\sigma(a)$
		\item{}$(\forall a,b,c\in\N)~(a<b\land b<c)\implies a<c$
	\end{enumerate}
	\item{}Доказать $a<b\land b<a\implies a<a$\footnote{
		Заметьте, что в данном случае $<$ не соответствует вам известному понятию $<$,
		введённому в школе.
	}.
	\item{}*Определить операцию сложения двух натуральных чисел.
	\item{}*Определить операцию умножения двух натуральных чисел.
\end{enumerate}

\part{В защиту интуиции}

Всякая теорема в математике должна быть доказана формально,
но в интуитивном понимании доказательств и логических утверждений есть ценность.
Интуитивное понимание легче укладывается в голове\footnote{Например, диаграммы
	Эйлера-Венна помогают понять и запомнить операции на множествах, но они
	не являются доказательствами.},
оно может натолкнуть на формальное доказательство. Поэтому я привожу пример
некоторых словесных интерпретаций логических утверждений\footnote{В качестве
	упражнения попробуйте обосновать эти интерпретации.}:
\begin{enumerate}
	\item{}$\exists k:(\forall n>k)~P(n)$ --- $P(n)$ начиная с какого-то $n$,
	для достаточно больших $n$.
	\item{}$\exists \delta:(\forall x)~|x|<\delta\implies P(x)$ --- $P(x)$ для
	достаточно малых $x$.
	\item{}$\forall \varepsilon~\exists \delta:P(\varepsilon,\delta)$ ---
	для всякого $\varepsilon$ можно подобрать такой $\delta$,\\что $P(\varepsilon,\delta)$.
\end{enumerate}

Важно понимать, что \textsc{математические объекты абстрактны
	и не имеют связи с действительностью.}
${1+1=2}$ не потому что ``если взять одно и одно яблоко, то будет два яблока'', а
потому что мы определили $1$, $2$, $+$ и $=$ таким образом, что $1+1=2$\footnote{
	Вспомните, почему истина в формальной системе называется тавтологией.
}.
