\part{В защиту интуиции}

Всякая теорема в математике должна быть доказана формально,
но в интуитивном понимании доказательств и логических утверждений есть ценность.
Интуитивное понимание легче укладывается в голове\footnote{Например, диаграммы
	Эйлера-Венна помогают понять и запомнить операции на множествах, но они
	не являются доказательствами.},
оно может натолкнуть на формальное доказательство. Поэтому я привожу пример
некоторых словесных интерпретаций логических утверждений\footnote{В качестве
	упражнения попробуйте обосновать эти интерпретации.}:
\begin{enumerate}
	\item{}$\exists k:(\forall n>k)~P(n)$ --- $P(n)$ начиная с какого-то $n$,
	для достаточно больших $n$.
	\item{}$\exists \delta:(\forall x)~|x|<\delta\implies P(x)$ --- $P(x)$ для
	достаточно малых $x$.
	\item{}$\forall \varepsilon~\exists \delta:P(\varepsilon,\delta)$ ---
	для всякого $\varepsilon$ можно подобрать такой $\delta$,\\что $P(\varepsilon,\delta)$.
\end{enumerate}

Важно понимать, что математические объекты не имеют связи с действительностью.
$1+1=2$ не потому что если взять одно и одно яблоко, то будет два яблока, а
потому что мы определили $1$, $2$, $+$ и $=$ таким образом, что $1+1=2$ (тавтология).
