\part{Элементы теории множеств}

{\it Множество} --- неопределимое понятие в рамках формальной системы,
обычно оно интерпретируется как совокупность объектов,
называемых {\it элементами множества}.
В теории множеств каждый терм --- множество.

Введём в алфавит бинарный предикат $\in$. Выражение $x\in S$ читается
как $x$ --- элемент $S$.

Пусть $P$ --- утверждение о $x$. Можно сказать, что $P$ описывает какое-то
свойство $x$.

Формулы, описывающие свойство каждого элемента множества или
существование элемента с каким-то свойством обычно записывают в сокращённом виде:
\[
	(\forall x\in S)~P\qquad \exists x\in S:P\qquad \exists!x\in S:P
\]
Они читаются как ``$P$ для произвольного $x$ из $S$'' и
``существует (единственный) $x$ в $S$ такой, что $P$'' соответственно.

\newcommand\N{\mathbb N}
Например,
\[
	(\forall n\in\N)~\exists m\in\N:n+1=m
\]
читается как ``для любого натурального числа $n$ существует натуральное число
$m$ такое, что $n+1=m$''.

\section{Равенство множеств и операции на множествах}

Введём понятие равенства множеств первой аксиомой ZFC\footnote{
	ZF (Zermelo-Fraenkel) --- набор аксиом Цермело-Френкеля, обычно с ними также
	используется Аксиома Выбора. Аксиомы ZF с Аксиомой Выбора
	обозначаются как ZFC (ZF, axiom of Choice).

	ZFC --- одна из наиболее широко используемых систем аксиом теории множеств.
}:
\[
	(\forall x)(\forall y)[(\forall z)(z\in x\iff z\in y)\implies x=y]
\]
$A\subseteq B$ читается как $A$ --- подмножество $B$, то есть все
элементы множества $A$ лежат в $B$ (являются элементами $B$).

Введём новый бинарный предикатный знак $\subseteq$ и аксиому
\[
	\forall x\forall y[(x\subseteq y)\iff \forall z(z\in x\implies z\in y)]
\]

Тогда первую аксиому ZFC можно записать проще:
\[
	\forall x\forall y[x\subseteq y\land y\subseteq x\implies x=y]
\]

Из аксиом ZFC\footnote{
	Данная книга не будет углубляться в сами аксиомы ZFC, но я сильно реккомендую
	ознакомиться с ними самостоятельно.
} следуют формулы
\[
	\forall x\forall y\exists! z:\forall w(w\in z\iff w\in x\land w\in y)
\]
\[
	\forall x\forall y\exists! z:\forall w(w\in z\iff w\in x\lor w\in y)
\]
Поэтому можем ввести функциональные знаки $\cap$ и $\cup$ и аксиомы
\[
	\forall x\forall y\forall w(w\in x\cap y\iff w\in x\land w\in y)
\]
\[
	\forall x\forall y\forall w(w\in x\cup y\iff w\in x\lor w\in y)
\]

$A\cap B$ называется {\it пересечением} $A$ и $B$,
$A\cup B$ --- их {\it объединением}.

\vspace{1em}
{\it Теорема:} Пусть $S$ --- множество, тогда $S=S\cap S$.

	{\it Доказательство:}
\[
	t\in S\implies t\in S\land t\in S\implies t\in S\cap S
\]

Тогда $(\forall x)(x\in S\implies x\in S\cap S)$ и $S\subseteq S\cap S$.
\[
	t\in S\cap S\implies t\in S\land t\in S\implies t\in S
\]

Тогда $(\forall x)(x\in S\cap S\implies x\in S)$ и $S\cap S\subseteq S$.

Тогда $S=S\cap S$ по первой аксиоме ZFC.\qed

\vspace{1em}
{\it Теорема:} Пусть $R,S$ --- множества и $R\subseteq S$, тогда
\[
	R\cap S= R,\qquad R\cup S= S
\]

{\it Доказательство:}

Пусть $R\subseteq S$.
\[
	t\in R\cap S\iff t\in R\land t\in S\xLeftrightarrow{t\in R\implies t\in S} t\in R
\]

Тогда $(\forall x)(x\in R\cap S\iff x\in R)$ и $R\cap S=R$.
\[
	t\in R\cup S\iff t\in R\lor t\in S\iff t\in S
\]

Тогда $R\cup S=S$.\qed

Последнее доказательство использует цепочки эквивалентности:
цепочка $A_1\iff...\iff A_{n}$ означает, что $A_1\iff A_2$, ... и
$A_{n-1}\iff A_{n}$, и доказывает $A_1\iff A_{n}$. Иначе пришлось
бы составлять по две цепочки импликаций на цепочку эквивалентности.

\newcommand\eset{\varnothing}
{\it Упражнения:}
\begin{enumerate}
	\item{}Доказать следующие утверждения для множеств $R,S,T$
	\begin{fullwidth}
		\begin{multicols}{2}
			\begin{enumerate}
				\item{}$S\cup S=S$
				\item{}$R\cap (S\cup T)=(R\cap S)\cup (R\cap T)$
				\item{}$S\cap (S\cup T)=S\cup (S\cap T)=S$
				\item{}$R\subseteq T\implies R\cup (S\cap T)=(R\cup S)\cap T$
			\end{enumerate}
		\end{multicols}
	\end{fullwidth}

	\item{}Доказать, что для множеств $A$ и $B$ следующие утверждения эквивалентны
	(любые два утверждения эквивалентны).
	\[
		A\subseteq B\qquad A\cup B=B\qquad A\cap B=A
		\qquad (\forall x)~\lnot(x\in A\setminus B)
	\]
\end{enumerate}

\section{Конструкция из множеств}

Из аксиом ZFC для произвольного $n\geq 1$ следует
\[
	\forall a_1\forall a_2...\forall a_{n}\exists !A:\forall w
		[w\in A\iff (w=a_1\lor w=a_2\lor...\lor w=a_{n})],
\]
тогда введём $n$-арный функциональный знак $\{,\}$ и аксиому
\[
	\forall a_1\forall a_2...\forall a_{n}\forall w
	[w\in \{a_1,a_2,...,a_{n}\}\iff (w=a_1\lor w=a_2\lor...\lor w=a_{n})],
\]

Также из ZFC следует $\exists !e:\forall x(x\notin e)$\footnote{
	см. упражнение~\ref{ex:eset_only}.
}, можем ввести константу ($0$-арный функциональный знак) $\eset$
и аксиому $\forall x(x\notin \eset)$. $\eset$ называется {\it пустым множеством}.

На основе теории множеств можно построить многие другие области математики.
Например, натуральные числа можно определить следующим образом:
\[
	\begin{array}{llll}
		0  :=\eset\qquad & 1  :=\{\eset\}\qquad & 2  :=\{\eset,\{\eset\}\}\qquad & ... \\
		                 & 1:=\{0\}\qquad       & 2:=\{0,1\}
		                 & n:=\{0,1,...,n-1\}
	\end{array}
\]

Другими словами, ${0:=\eset}$ и $\sigma(x):=x\cup \{x\}$ --- функция следования,
возвращающая следующее натуральное число после $x$. Заметим, что количество элементов
в множестве $n$ равно $n$.

Часто в математике нужно понятие {\it упорядоченного множества}~---~множества,
в котором порядок элементов имеет значение. Его элементы записываются
в круглых скобках. Упорядоченное множество из двух элементов (упорядоченную пару)
можно определить следующим образом:
\[
	(x,y):=\{x,\{x,y\}\}
\]

\vspace{1em}
{\it Упражнения:}
\begin{enumerate}
	\item{}Доказать теоремы
	\begin{enumerate}
		\item{}$\exists e:\forall x(x\notin e)\implies
			\exists!e:\forall x(x\notin e)$\label{ex:eset_only}
		\item{}$\forall x\forall y[(x,y)\neq (y,x)]$
	\end{enumerate}

	\item{}Как можно определить упорядоченные множества из трёх элементов?
	Из одного элемента? Из $n$ элементов?

	\item{}*Возьмём совокупность множеств $U=\{U_{1},U_{2},...\}$.
	Как можно определить объединение, пересечение всех множеств в совокупности?
	Как это сделать, если $U$ бесконечно?
\end{enumerate}

\pagebreak
