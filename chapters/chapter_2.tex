\part{Вывод теорем}

\section{Формальная система}

Совокупность правил называется {\it формальной системой}, если выполняются следующие
условия:
\begin{enumerate}
	\item{}Задан {\it алфавит}.
	\item{}Определено понятия {\it формулы}.
	\item{}Задана совокупность формул, называемых {\it аксиомами}.
	\item{}Заданы {\it правила вывода} одних формул из других.
\end{enumerate}

Если формула $T$ выводима из формул $A_1,A_2,...,A_{n}$ за один шаг (одно применение
правил вывода), то пишут $A_1,A_2,...,A_{n}\vdash T$.

Слева от $\vdash$ порядок формул не имеет значения и могут быть излишние формулы:
если $A,B\vdash T$, то $B,A\vdash T$ и $B,A,C\vdash T$.

\newcommand\Sx{\mathcal S}
Возьмём совокупность формул $\Sx$. Пусть
\[
	\Sx\vdash A_1\qquad
	\Sx,A_1\vdash A_2\qquad...\qquad
	\Sx,A_1,A_2,...,A_{n-1}\vdash A_{n}
\]

Тогда пишут $\Sx\to A_{n}$ и говорят, что $A_{n}$ {\it доказуемо} из $\Sx$.
Очевидно,	$A\vdash B$ подразумевает $A\to B$, слева от $\to$ порядок формул
не имеет значения и могут быть излишние формулы.
Рассуждения, показывающие доказуемость называются {\it доказательством}.
Если $\Sx$~---~совокупность аксиом и $\Sx\to T$, то $T$ называют
{\it теоремой}.

\newcommand\ruleR{\mathbf{R}}
\newcommand\ruleC{\mathbf{C}}

Рассмотрим следующий пример формальной системы:
\begin{enumerate}
	\item{}Алфавит: $a$, $b$, $c$.
	\item{}Всякое непустое выражение является формулой.
	\item{}Аксиомы: $aab$, $c$.
	\item{}Правила вывода:
	\begin{multicols}{2}
		\begin{enumerate}
			\item[($\ruleR$)]{}$A\vdash A'$, если $A'$ можно получить из $A$,
			убрав один символ. Тогда $aab\vdash ab$.
			\columnbreak
			\item[($\ruleC$)]{}$A,B\vdash AB$, если у формул $A$ и $B$ нет
			общих символов. Тогда $b,c\vdash bc$ и $ac,bb\vdash acbb$.
		\end{enumerate}
	\end{multicols}
\end{enumerate}

{\it Теорема:} $abc$

{\it Доказательство:}
$aab\vdash ab$ по правилу $\ruleR$, значит $aab,c\vdash ab$.
$ab,c\vdash abc$ по правилу $\ruleC$, значит $aab,c,ab\vdash abc$.
Тогда $aab,c\to abc$ и $abc$ --- теорема, потому что $aab$ и $c$ --- аксиомы.
\qed\footnote{$\qed$ означает ``что и требовалось доказать''.}

Доказуемость можно показать с помощью рассуждений в {\it нормальном виде}:
Чтобы показать $\Sx\to T$ можно предположить формулы из $\Sx$ и последовательно
увеличивать список выведенных формул, пока не дойдём до $T$.

\pagebreak

Доказательство $abc$ в нормальном виде:
\begin{enumerate}[label=(\arabic*)]
	\item{}\label{1}Предположим $aab$.
	\item{}\label{2}Предположим $c$.
	\item{}\label{3}$ab$ по $\ruleR$, \ref{1}.
	\item{}\label{4}$abc$ по $\ruleC$, \ref{2}, \ref{3}.\qed
\end{enumerate}

\textsc{Формальная система является просто набором правил и её формулы
	могут не иметь смысла. Они и не будут иметь смысла, пока мы не начнём их
	интерпретировать, то есть придавать им этот смысл.}

В общем случае формальные системы не обращаются с понятиями ``истины'' и ``лжи'',
они обращаются с понятиями ``выводимости''. Но для удобства определим формулу
как {\it истинную}, если она является аксиомой или теоремой.

\section{Правила вывода}

Начнём описывать формальную систему, в которой мы будем работать.
Оставим тот же самый алфавит, понятия терма, переменной и формулы:
\begin{fullwidth}
	\begin{multicols}{2}
		\begin{enumerate}
			\item{}Любой знак переменной является термом.
			\item{}Если $f$ --- функциональный знак арности $n$, и $t_1,...,t_{n}$ --- термы,
			то $f(t_1,...,t_{n})$ --- терм.
			\item{}Если $\phi$ --- предикатный знак арности $n$, и $t_1,...,t_{n}$ --- термы,
			то $\phi(t_1,...,t_{n})$ --- формула.

			\columnbreak

			\item{}Пусть $F_1,F_2$ --- формулы, а $F$ --- формула о $x$. Выражения
			\[
				\lnot F_1\qquad F_1\land F_2\qquad F_1\lor F_2\qquad
				F_1\implies F_2
			\]
			\[
				F_1\iff F_2\qquad (\forall x)~F\qquad
				\exists x:F\qquad \exists! x:F
			\]
			являются формулами.
		\end{enumerate}
	\end{multicols}
\end{fullwidth}

\newcommand\taut{$\mathcal T$}
\newcommand\axiom{$\mathcal A$}
\newcommand\implic{$\mathcal I$}
\newcommand\Px{\mathcal P}
Возьмём формулу $\top$ как аксиому и начнём вводить правила вывода.
Введём правило \axiom{} для более удобного использования аксиом в доказательствах.
\begin{enumerate}
	\item[(\axiom)]{}$S\vdash A$, где $A$ --- аксиома, $S$ --- любая формула.
\end{enumerate}

Если $S\to T$ или $S\vdash T$ для произвольной формулы $S$,
то пишут $\to T$ и $\vdash T$ соответственно. \axiom{} можно переписать:
\begin{enumerate}
	\item[(\axiom)]{}$\vdash A$, где $A$ --- аксиома.
\end{enumerate}

Если $T$ выводимо из $A$ и $B$, то из $A$ должна быть выводима формула $B\implies T$,
поэтому введём правило \implic{}.
\begin{enumerate}
	\item[(\implic)]{}${\Gamma\vdash (S\implies T)}$, если $\Gamma,S\to T$,
	где $\Gamma$ --- совокупность формул\footnote[][-2cm]{
		Три ``стрелочки'' имеют схожие значения:
		\begin{enumerate}
			\item{}$\vdash$ означает выводимость за один шаг.
			\item{}$\to$ означает существование доказательства.
			\item{}$\implies$ --- знак из алфавита формальной системы.
		\end{enumerate}
	}.
\end{enumerate}

Для использования законов логики (modus ponens, следствие от обратного, закон
исключённого тертьего) введём правило \taut{}.
\begin{enumerate}
	\item[(\taut)]{}$F_1(\Sx/\Px),...,F_n(\Sx/\Px)\vdash T(\Sx/\Px)$, если
	\begin{enumerate}
		\item{}Выражение ${F_1\land...\land F_n\implies T}$ --- простая тавтология.
		\item{}$\Px$ --- совокупность всех знаков переменных тавтологии.
		\item{}$\Sx$ --- совокупность формул, причём каждому знаку из $\Px$
		поставлена в соотвествие единственная формула из $\Sx$.
	\end{enumerate}
\end{enumerate}

Например, возьмём простую тавтологию $(p\land (p\implies q))\implies q$.
\[
	F_1\equiv p\qquad F_2\equiv p\implies q\qquad T\equiv q
\]
Совокупность $\Px$ содержит $p$ и $q$.
Совокупность $\Sx$ содержит произвольные формулы $A$ и $B$.
По \taut{} имеем правило ${A,[A\implies B]\vdash B}$
для произвольных формул $A,B$.
Исходя из той же тавтологии также правило ${A\land[A\implies B]\vdash B}$
для произвольных формул $A,B$.

Докажем два занимательных факта:
\begin{enumerate}
	\item{}
	      {\it Теорема:}
	Если $T$ --- простая тавтология, то
	для произвольного набора формул $\Sx$ справедливо\footnote{
		Справедливо --- истинно, доказуемо.
	} $T(\Sx/\Px)$
	($\Px$ и $\Sx$ определены как в правиле \taut{}).

		{\it Доказательство:}
	$T$ --- простая тавтология, значит и
	${\top\implies T}$ --- простая тавтология. Тогда по правилу \taut{} имеем
	$\top\vdash T(\Sx/\Px)$.\qed

	\item{}Если $T$ --- теорема, то $\to T$.

		{\it Доказательство:}
	Пусть теорема $T$ выводима из аксиом $A_1,...,A_{n}$.
	\begin{enumerate}[label=(\roman*)]
		\item{}\label{1p}Предположим произвольную формулу $S$.
		\item{}\label{2p}По \axiom{} из \ref{1p} можем вывести
		формулы $A_1,...,A_{n}$.
		\item{}\label{3p}$A_1,...,A_{n}\to T$, значит из \ref{2p} можем вывести $T$.\qed
	\end{enumerate}

	Таким образом, $T$ истинна ттк $\to T$, значит для доказательства $T$
	достаточно показать $\to T$.

	Строку \ref{1p} можно опустить, но подразумевать.
	Для доказательства $T$ сначала подразумеваем предположение $S$, потом
	выводим $T$. В ходе доказательства необходимые
	аксиомы и теоремы можно вывести из $S$.

	$\to A$ для любой аксиомы или теоремы $A$,
	значит в доказательство в нормальном виде
	можно добавить строку $A$, подразумевая, что она была выведена из $S$.

	\textsc{Понятия выводимости из ниоткуда не существует,
		самое близкое к нему --- выводимость из произвольной формулы.}
\end{enumerate}

\section{Кванторы}

\newcommand\Aii{$\forall$I}
\newcommand\Aee{$\forall$E}
\newcommand\Eii{$\exists$I}
\newcommand\Eee{$\exists$E}

Новый знак можно ввести набором из двух правил:
$\lambda$I\footnote{I --- Introduction, введение} --- {\it правило введения}
и $\lambda$E\footnote{E --- Elimination, исключение} --- {\it правило
исключения (использования)}, где $\lambda$ --- знак, который мы хотим ввести.

Начнём введение правил с квантора всеобщности. $(\forall x)P$ означает, что $P$ для
всякого $x$, значит оно доказуемо для произвольного~$x$. Это можно сформулировать
в правилах
\begin{enumerate}
	\item[(\Aii{})]{}$\Gamma\vdash(\forall x)P(t/_{a}x)$, если $\Gamma\to P$, где
	$P$ --- формула о $t$ и ни одна из формул в совокупности $\Gamma$
	не содержит $t$.

	\item[(\Aee{})]{}$(\forall x)P\vdash P(x/_{a}t)$, где $t$ --- терм.
\end{enumerate}

Для квантора существования вводятся следующие правила:
\begin{enumerate}
	\item[(\Eii{})]$P\vdash [\exists x:P(t/x)]$, где $P$ --- формула о $t$.
	\item[(\Eee{})]$[\exists x:P],[(\forall x)(P\implies C)]\vdash C$,
	где $C$ не содержит $x$.
\end{enumerate}

Заметим, что при введении правил мы
сначала выбираем, какой смысл мы придаём обозначениям, а потом вводим
правила для формализации этого смысла.
\textsc{С помощью правил мы формализуем смысл, придаваемый обозначениям.}

{\it Теорема:} $\lnot [\exists x:P]\implies  (\forall x)\lnot P$

{\it Доказательство:}
\begin{enumerate}[label=(\arabic*)]
	\item{}\label{1}Предположим $\top$.
	\item{}\label{2}Предположим $\lnot [\exists x:P]$.
	\item{}\label{3}Введём переменную $t$, предполагая, что $P$ не содержит $t$.
	\item{}\label{4}Предположим $P(x/_{a}t)$.
	\item{}\label{5}По \Eii{}, \ref{4} имеем $\exists x:P$.
	\item{}\label{6}По \implic{}, \ref{4}-\ref{5} имеем
	$P(x/_{a}t)\implies \exists x:P$.
	\item{}\label{7}${[p\implies q],\lnot q\vdash \lnot p}$ по \taut{}\footnote{
		Опираясь на тавтологию
		\[
			[(p\implies q)\land\lnot q]\implies\lnot p
		\]
	}, из
	\ref{2} и \ref{6} можем вывести $\lnot P(x/_{a}t)$.
	\item{}\label{8}Рассуждения \ref{2}-\ref{7} показывают
	$\lnot[\exists x:P]\to\lnot P(x/_{a}t)$, значит по \Aii{} из \ref{2} можем
	вывести $(\forall x)\lnot P$.
	\item{}\label{9}Рассуждения выше показывают
	$\top,\lnot[\exists x:P]\to (\forall x)\lnot P$, значит по \implic{}
	имеем $\top\vdash \lnot [\exists x:P]\implies  (\forall x)\lnot P$.\qed
\end{enumerate}

Первая строка нужна, потому что для доказательства $T$ нужно показать $\top\to T$.
Далее будем опускать её:
Для доказательства формулы $T$ подразумеваем строку ``предположим $\top$'',
и выводим~$T$, из $\top$ по необходимости можно выводить теоремы и аксиомы\footnote{
	Вспомним, что, если $T$ --- теорема или аксиома, то $\to T$.
}.
\textsc{Понятия выводимости из ниоткуда не существует,
	самое близкое к нему --- выводимость из произвольной формулы.}

Введём квантор существования и единственности $\exists !$ аксиомой
\[
	[\exists! x:P]\iff [\exists x:P]\land[(\forall x,y)~(P\land P(x/_{a}y)\implies x=y)],
\]
где $P$ --- формула о $x$, не содержащая $y$. Эту аксиому можно скорее называть
{\it схемой аксиом}, потому что она включает в себя по аксиоме для
каждой подходящей формулы $P$.

\vspace{1em}
{\it Упражнения:}
\begin{enumerate}
	\item{}Обосновать правила \Eii{} и \Eee{}.
	\item{}Обосновать ограничения на связанность и свободу переменных в правилах
	\Aii{}, \Aee{}, \Eii{} и \Eee{}.

	\pagebreak
	\item{}\label{ex:obv_thm}Доказать теоремы:
	\begin{enumerate}
		\item{}$[(\forall x)P]\land[(\forall x)(P\implies Q)]
			\implies [(\forall x)Q]$\label{thm:obv_forall}
		\item{}$[\exists x:P]\land[(\forall x)(P\implies Q)]
			\implies [\exists x:Q]$
		\item{}$(\forall x)T(p/F)$, где $T$ --- простая тавтология,
		в которой $p$ --- единственный символ переменной, $F$ --- формула о $x$.
		\item{}$(\forall x)T(\Px/\Sx)$, где $T$ --- простая тавтология,
		$\Px$ --- символы её переменных, $\Sx$ --- формулы о $x$.\label{thm:obv_taut}
	\end{enumerate}
\end{enumerate}

\section{Цепочки импликаций}

Часто доказательства в математике записываются в виде {\it цепочек импликаций}.
Например, пусть мы уже вывели совокупность формул $\Gamma$ и пусть
\[
	\Gamma,A_1\to A_2\qquad \Gamma,A_2\to A_3
	\qquad  ... \qquad \Gamma,A_{n-1}\to A_{n}
\]
Тогда по правилу \implic{} из $\Gamma$ можем вывести импликации,
которые кратко можно записать в виде цепочки
\[
	A_1\implies A_2\implies ...\implies A_{n}
\]
По тавтологии
\[
	[(p\implies q)\land (q\implies r)]\implies (p\implies r)
\]
построение такой цепочки доказывает $A_1\implies A_{n}$ из $\Gamma$.

Цепочки импликаций также можно называть ``человеческим видом'', потому что они более
понятны при чтении.

В человеческом виде часто используются обороты из таблицы~\ref{table:human_form}.
\begin{table}
	\centering
	\begin{tabular}{l|l}
		Пусть $A$.                 & Пусть $A$.                                   \\
		$A\implies ... \implies B$ & $B\implies ...\implies \bot
		\text{ (или $\lnot A$)}$                                                  \\
		Тогда $B$.                 & Тогда $\lnot B$.                             \\\hline
		Пусть $A$.                 & Пусть  $A$. Пусть $B$.                       \\
		...                        & ...                                          \\
		Тогда $B$.                 & Тогда $\bot$ (или $\lnot A$, или $\lnot B$). \\
		Тогда $A\implies B$.       & Тогда $A\implies \lnot B$
		(или $B\implies\lnot A$).
	\end{tabular}
	\caption{Обороты в человеческом виде.}\label{table:human_form}
\end{table}

\vspace{1em}
{\it Теорема:} $\lnot[\exists x:P]\implies (\forall x)\lnot P$

{\it Доказательство:}

Пусть\footnote{
	``Пусть'' --- синоним слова предположим. Некоторые авторы также
	используют значок $\sqsupset$.
} $\lnot[\exists x:P]$. Возьмём произвольное $t$. Это и следующие доказательства
предполагают, что $P$ не содержит $t$.
\[
	P(x/_{a}t)\xRightarrow{\text{$\exists$I}} \exists x:P
	\xRightarrow{p,\lnot p\vdash\bot} \bot
\]

Тогда $\lnot P(x/_{a}t)$ и $(\forall x)\lnot P$
по $\forall$I.\qed

\vspace{1em}
{\it Теорема:} $\lnot[(\forall x)P]\implies \exists x:\lnot P$

{\it Доказательство:}

Пусть $\lnot[(\forall x)P]$.
\[
	\lnot[\exists x:\lnot P]\implies (\forall x)\lnot\lnot P
	\xRightarrow{\lnot\lnot q\implies q} (\forall x)P\implies\bot
\]

Тогда $\exists x:\lnot P$.\qed

\vspace{1em}
{\it Теорема:} $(\forall x)\lnot P\implies \lnot[\exists x:P]$

{\it Доказательство:}

Пусть $(\forall x)\lnot P$. Возьмём произвольное $t$. $\lnot P(x/_{a}t)$ по $\forall$E.
\[
	P(x/_{a}t)\implies \bot
\]

Тогда $(\forall x)(P\implies \bot)$.
\[
	[\exists x:P]\xRightarrow{\text{$\exists$E}} \bot
\]

Тогда $\lnot[\exists x:P]$.\qed

\vspace{1em}
{\it Теорема:} $\exists x:\lnot P\implies\lnot[(\forall x)P]$

{\it Доказательство:}

Пусть $\exists x:\lnot P$. Возьмём произвольное $t$. Пусть $\lnot P(x/_{a}t)$.
\[
	(\forall x)P\implies P(x/_{a}t)\implies \bot
\]

Тогда ${\lnot P(x/_{a}t)\implies \lnot[(\forall x)P]}$ и по $\forall$I, $\exists$E
имеем $\lnot[(\forall x)P]$.\qed

\vspace{1em}
{\it Упражнения:}
\begin{enumerate}
	\item{}Обосновать каждую импликацию в доказательствах законов отрицания кванторов.
	\item{}*Доказать в нормальном виде законы отрицания кванторов.
\end{enumerate}

\section{Равенство}

Так как мы абстрагировались от понятия значения выражения,
нам придётся заново ввести понятие равенства, то есть ввести правила, по которым
мы будем манипулировать формулами с знаком $=$.

Введём новый бинарный предикатный знак $=$\footnote{
	Тогда, если $a,b$	 --- термы, то $a=b$ --- формула.
}.
Основное свойство равенства в том, что если два выражения равны, то их можно
заменять друг другом в формулах. Введём правило $=$E.
\begin{enumerate}
	\item[($=$E)]{}$P,[A=B]\vdash P(A/B)$,
	где $P$ --- формула, $A,B$ --- термы.
\end{enumerate}

Любой терм равен сам себе, введём аксиому
\[
	(\forall x)~x=x
\]

Часто аксиомы формулируются как формулы без
свободных переменных, чтобы избежать схем аксиом.
Формулы без свободных переменных называются {\it замкнутыми}.


{\it Теорема:} Пусть $A,B,P$ --- термы, тогда
\[
	A=B\implies P=P(A/B)
\]

{\it Доказательство:}

Пусть $A=B$. По аксиоме выше имеем $P=P$. По правилу $=$E имеем $P=P(A/B)$.\qed

% \textsc{Обычно понятие равенства для различных родов выражений
%   вводится через такие аксиомы.}
% Например для векторов равенство вводится следующей аксиомой:
% \[
%   (\forall \vec{a},\vec{b})~
%   \big[\vec{a}\upuparrows\vec{b}\land|\vec{a}|=|\vec{b}|\big]
%   \implies \vec{a}=\vec{b}
% \]

\vspace{1em}
{\it Упражнения:}
\begin{enumerate}
	\item{}Доказать теоремы
	\begin{enumerate}
		\item{}$(\forall a)~\exists! b:a=b$
		\item{}$(\forall x,y)(x=y\iff y=x)$
		\item{}$\forall x,y,z[(x=y)\land (y=z)\implies (x=z)]$
	\end{enumerate}
\end{enumerate}

\section{Введение новых знаков}

Введём правила для введения сокращений, то есть правила введения новых функциональных
и предикатных знаков.

Пусть $T,T'$ --- система до и после введения нового сокарщения. Тогда
если $\psi$ --- формула в $T$ и доказуема в $T'$, то $\psi$ доказуема в $T$.

Поэтому внутри доказательства можно для удобства расширить систему
новыми знаками (например, ввести новые функции).

Начнём с правила введения новых предикатных символов. Пусть $\varphi$ --- формула,
$t_1,...,t_{n}$ --- все её свободные переменные, тогда можем ввести
$n$-арный предикатный символ $p$ и аксиому
\[
	\forall t_1\forall t_2...\forall t_{n}[p(t_1,...,t_{n})\iff\varphi]
\]

Например, пусть $<$ --- бинарный предикатный знак, тогда можем сократить формулу
$(x=y)\lor (x<y)$ до $x\leq y$, введя бинарный предикатный знак $\leq$ и аксиому
\[
	\forall x\forall y[(x\leq y)\iff (x=y)\lor (x<y)]
\]

Для каждого бинарного предикатного знака $\prec$ можем ввести знак $\nprec$ и аксиому
\[
	\forall x\forall y[(x\nprec y)\iff \lnot(x\prec y)]
\]

Перейдём к функциональным знакам. Функция --- правило, по которому совокупности
аргументов сопоставляется единственное значение. Поэтому можем ввести следующее
правило: Пусть $\varphi$ --- формула, $x_1,...,x_{n},y$ --- все её свободные переменные
и доказуемо
\[
	\forall x_1\forall x_2...\forall x_{n}\exists !y:\varphi,
\]
тогда можем ввести $n$-арный функциональный знак $f$ и аксиому
\[
	\forall x_1\forall x_2...\forall x_{n} [\varphi(y/_{a}f(x_1,...,x_{n}))]
\]

Например, пусть $+$ --- бинарный функциональный знак. Возьмём формулу $y=x+a$, очевидно,
\[
	(\forall x,a)~\exists !y:y=x+a
\]
Можем ввести бинарный функциональный знак $g$ и аксиому
\[
	(\forall x,a)~g(x,a)=x+a
\]

Часто в доказательствах некоторые аргументы опускают.
Например, $g(x,a)$ сокращают до $g(x)$.
Иногда все эти рассуждения выше сокращают до $g(x):=x+a$.

Пусть $\varphi$ --- формула, $x_1,...,x_{n},y$ --- все её
свободные переменные, $P$ --- формула только о $x_1,...,x_{n}$ и
\begin{equation}\label{eq:partial_fn_def}
	\forall x_1\forall x_2...\forall x_{n}(P\implies \exists!y:\varphi),
\end{equation}
тогда функциональный знак можно определить только для
таких $x_1,...,x_{n}$, что $P$. Введём $0$-арный функциональный знак $\bot'$,
функция будет равна ему при $\lnot P$.

	{\it Теорема:} Из \eqref{eq:partial_fn_def} следует
\[
	\forall x_1\forall x_2...\forall x_{n}\exists !y:
	[(P\implies\varphi)\land(\lnot P\implies y=\bot')]
\]

{\it Доказательство:}

Предположим \eqref{eq:partial_fn_def}.
Возьмём произвольные $x_1,...,x_{n}$.

\begin{enumerate}
	\item{}Пусть $P$. Из \eqref{eq:partial_fn_def} следует
	\[
		P\implies \exists !y:\varphi\implies \exists y:\varphi
	\]

	Возьмём произвольное $c$\footnote{
		Предполагается, что $\varphi$ и $P$ не содержат $c$.
	} и предположим $\varphi(y/_{a}c)$.

	Тогда $P\implies\varphi(y/_{a}c)$. Из $P$ следует $\lnot P\implies c=\bot'$.

	Тогда из $\exists y:\varphi$ можем вывести
	\begin{equation}\label{eq:part_fn_def_exists}
		\exists y:[(P\implies\varphi)\land(\lnot P\implies y=\bot')]
	\end{equation}

	\item{}Пусть $\lnot P$.

	Возьмём произвольное $c$ и предположим $c=\bot'$.

	Тогда $\lnot P\implies c=\bot'$.
	Из $\lnot P$ следует $P\implies \varphi$.

	Тогда из $\exists y:y=\bot'$ можем вывести \eqref{eq:part_fn_def_exists}.
\end{enumerate}

По тавтолгии $r\lor\lnot r$ имеем $P\lor \lnot P$.

По тавтологии $[(p\lor q)\land (p\implies r)\land (q\implies r)]\implies r$\footnote{
	На ней основаны доказательства с оборотом ``рассмотрим следующие случаи''.
}
имеем
\[
	[(P\lor \lnot P)\land (P\implies\eqref{eq:part_fn_def_exists})\land
			(\lnot P\implies \eqref{eq:part_fn_def_exists})]
	\implies\eqref{eq:part_fn_def_exists}
\]

Таким образом, мы доказали \eqref{eq:part_fn_def_exists}.
Теперь нужно доказать единственность такого $y$.

Возьмём $y,y'$ и предположим
\[
	(P\implies \varphi)\land (\lnot P\implies y=\bot')
\]
\[
	(P\implies \varphi(y/_{a}y'))\land (\lnot P\implies y'=\bot'),
\]

Тогда
\[
	P\implies \varphi\land \varphi(y/_{a}y')\land\exists !y:\varphi\implies y=y'
\]
\[
	\lnot P\implies y=\bot'\land y'=\bot'\implies y=y'
\]

Таким образом,
\[
	\forall x_1\forall x_2...\forall x_{n}\exists !y:
	[(P\implies \varphi)\land(\lnot P\implies y=\bot')]\qed
\]

Например, пусть $\cdot$ --- бинарный функциональный знак,
а $0,1$ --- $0$-арные функциональные знаки и справедлива формула
\[
	\forall a(a\neq 0\implies \exists !b:a\cdot b=1)
\]

По теореме выше имеем
\[
	\forall a\exists !b:[(a\neq 0\implies a\cdot b=1)\land (a=0\implies b=\bot')],
\]
можем ввести унарный функциональный знак $f$ и аксиому
\[
	\forall a[(a\neq 0\implies a\cdot f(a)=1)\land (a=0\implies f(a)=\bot')],
\]
тогда выполняется
\[
	(\forall a\neq 0)~a\cdot f(a)=1
\]
