\part{Вывод теорем}

\section{Формальная система}

Совокупность правил называют {\it формальной системой}\index{формальная система},
если выполняются следующие условия:
\begin{enumerate}
  \item{}Задан {\it алфавит}.
  \item{}Определено понятия {\it формулы}.
  \item{}Задана совокупность формул, называемых {\it аксиомами}\index{Аксиома}.
  \item{}Заданы {\it правила вывода}\index{правило!вывода} одних формул из других.
\end{enumerate}

Если формула $T$ выводима из формул $A_1,A_2,...,A_{n}$ за один шаг (одно применение
правил вывода), то пишут $A_1,A_2,...,A_{n}\vdash T$.

Слева от $\vdash$ порядок формул не имеет значения и могут быть излишние формулы:
если $A,B\vdash T$, то $B,A\vdash T$ и $B,A,C\vdash T$.

\newcommand\Sx{\mathcal S}
Возьмём совокупность формул $\Sx$. Пусть
\[
  \Sx\vdash A_1\qquad
  \Sx,A_1\vdash A_2\qquad...\qquad
  \Sx,A_1,A_2,...,A_{n-1}\vdash A_{n}
\]

Тогда пишут $\Sx\to A_{n}$ и говорят, что $A_{n}$
{\it доказуемо из}\index{доказуемость} $\Sx$.
Очевидно,	$A\vdash B$ подразумевает $A\to B$. Слева от $\to$ порядок формул
не имеет значения и могут быть излишние формулы.
Рассуждения, показывающие доказуемость называются
{\it доказательством}\index{доказательство}.
Если $\Sx$~---~совокупность аксиом и $\Sx\to T$, то $T$ называют
{\it теоремой}\index{теорема}.

\newcommand\ruleR{\mathbf{R}}
\newcommand\ruleC{\mathbf{C}}

Рассмотрим следующий пример формальной системы:
\begin{enumerate}
  \item{}Алфавит: $a$, $b$, $c$.
  \item{}Всякое непустое выражение является формулой.
  \item{}Аксиомы: $aab$, $c$.
  \item{}Правила вывода:
  \begin{multicols}{2}
    \begin{enumerate}
      \item[($\ruleR$)]{}$A\vdash A'$, если $A'$ можно получить из $A$,
      убрав один символ. Тогда $aab\vdash ab$.
      \columnbreak
      \item[($\ruleC$)]{}$A,B\vdash AB$, если у формул $A$ и $B$ нет
      общих символов. Тогда $b,c\vdash bc$ и $ac,bb\vdash acbb$.
    \end{enumerate}
  \end{multicols}
\end{enumerate}

{\it Теорема:} $abc$

{\it Доказательство:}
$aab\vdash ab$ по правилу $\ruleR$, значит $aab,c\vdash ab$.
$ab,c\vdash abc$ по правилу $\ruleC$, значит $aab,c,ab\vdash abc$.
Тогда $aab,c\to abc$ и $abc$ --- теорема, поскольку $aab$ и $c$ --- аксиомы.
\qed\footnote{$\qed$ означает ``что и требовалось доказать''.}

Более наглядно
доказуемость можно показать с помощью рассуждений в
{\it нормальном виде}\index{нормальный вид}, где рассуждение
разбивают на определённые шаги, которые потом нумеруют.

В рассуждениях, которые мы будем рассматривать, каждый шаг является предположением
или выводом формулы. Предположение вводит новую формулу в рассуждение.
Вывод с помощью правил вывода получает новую формулу из предыдущих.

Пусть последний шаг в рассуждении в нормальном виде является выводом формулы $T$,
а $\Sx$ --- формулы, которые были введены предположением.
Не влияя на справедливость рассуждения,
можем переставить шаги так, чтобы все предположения были в начале,
тогда получим
\[
  \Sx\vdash C_1\qquad \Sx,C_1\vdash C_2\qquad ...
  \qquad \Sx,C_1,...,C_{k}\vdash T
\]
и, по определению, $\Sx\to T$.

Сформулируем доказательство $abc$ в нормальном виде:
\begin{enumerate}[label=(\arabic*)]
  \item{}\label{normex_1}Пусть (предположим) $aab$.
  \item{}\label{normex_3}$ab$ по $\ruleR$, \ref{normex_1}.
  \item{}\label{normex_2}Пусть $c$.
  \item{}\label{normex_4}$abc$ по $\ruleC$, \ref{normex_2}, \ref{normex_3}.\qed
\end{enumerate}
Часто предположения аксиом опускают:
\begin{enumerate}[label=(\arabic*)]
  \item{}\label{normexshort_1}$ab$ по $\ruleR$ и аксиоме $aab$.
  \item{}\label{normexshort_2}$abc$ по $\ruleC$,
  \ref{normexshort_1} и аксиоме $c$.\qed
\end{enumerate}

\textsc{Формальная система является просто набором правил и её формулы
  могут не иметь смысл. Они и не будут иметь смысл, пока мы не начнём их
  интерпретировать, то есть придавать им этот смысл.}

Математику можно рассматривать как своеобразную ``игру'' с формулами,
правила которой определяются формальной системой.
Интерпретация её результатов, придание смысла формулам являются
отдельными от самой игры процессами.

\index{modes ponens}
Конечно, определяя правила игры, мы будем вдохновляться логическими рассуждениями
человеческого языка.
Например, из ``Если $A$, то $B$'' и ``$A$'' можем вывести ``$B$'', поэтому обычно
вводят правило ${A,[A\implies B]\vdash B}$,
которое называют {\it modes ponens}\footnote{лат. правило вывода}.

В общем случае формальные системы не обращаются с понятиями ``истины'' и ``лжи'',
они обращаются с понятиями ``выводимости'', однако для удобства определим формулу
как {\it истинную}\index{истинность}\index{формула!истинная},
если она является аксиомой или теоремой.

\section{Правила вывода}

Начнём описывать формальную систему, в которой мы будем работать.
Оставим тот же самый алфавит, понятия терма, переменной и формулы.
\begin{fullwidth}
  \begin{multicols}{2}
    \begin{enumerate}
      \item{}Любой знак переменной является термом.
      \item{}Если $f$ --- функциональный знак арности $n$, и $t_1,...,t_{n}$ --- термы,
      то $f(t_1,...,t_{n})$ --- терм.
      \item{}Если $\phi$ --- предикатный знак арности $n$, и $t_1,...,t_{n}$ --- термы,
      то $\phi(t_1,...,t_{n})$ --- формула.

      \columnbreak

      \item{}Пусть $F_1,F_2$ --- формулы, $F$ --- формула о $x$, тогда выражения
      \[
        \lnot F_1\qquad F_1\land F_2\qquad F_1\lor F_2\qquad
        F_1\implies F_2
      \]
      \[
        F_1\iff F_2\qquad (\forall x)~F\qquad
        \exists x:F
      \]
      являются формулами.
    \end{enumerate}
  \end{multicols}
\end{fullwidth}

\newcommand\taut{$\mathcal T$}
\newcommand\axiom{$\mathcal A$}
\newcommand\implic{$\mathcal I$}
\newcommand\Px{\mathcal P}
Возьмём формулу $\top$\footnote{Константы $\top$ и $\bot$ вводят как
  \mbox{$0$-арные} предикатные знаки.}
как аксиому и начнём вводить правила вывода.
Для использования законов логики (modus ponens, закон
исключённого тертьего и прочие) введём правило \taut{}.
\begin{enumerate}
  \item[(\taut)]{}$F_1(\Sx/\Px),F_2(\Sx/\Px)...,F_n(\Sx/\Px)\vdash T(\Sx/\Px)$, если
  \index{правило!\taut}
  \begin{enumerate}
    \item{}Выражение ${F_1\land F_2\land...\land F_n\implies T}$ --- простая тавтология.
    \item{}$\Px$ --- совокупность всех знаков переменных тавтологии.
    \item{}$\Sx$ --- совокупность произвольных формул,
    причём каждому знаку из $\Px$
    поставлена в соотвествие единственная формула из $\Sx$.
    По этому соответствию и идёт замена знаков формулами в выражениях.
  \end{enumerate}
\end{enumerate}

Например, возьмём простую тавтологию $(p\land (p\implies q))\implies q$.
\[
  F_1\equiv p\qquad F_2\equiv p\implies q\qquad T\equiv q
\]
Совокупность $\Px$ содержит $p$ и $q$.
Совокупность $\Sx$ содержит произвольные формулы $A$ и $B$.
По \taut{} имеем правило ${A,[A\implies B]\vdash B}$
для произвольных формул $A,B$.
Исходя из той же тавтологии, имеем правило ${A\land[A\implies B]\vdash B}$.

Если $T$ выводима из $A$ и $B$, то из $A$ должна быть выводима формула $B\implies T$,
поэтому введём правило \implic{}.
\begin{enumerate}
  \item[(\implic)]{}${\Gamma\vdash (S\implies T)}$, если $\Gamma,S\to T$,
  \index{правило!\implic}
  где $\Gamma$ --- совокупность формул
\end{enumerate}
В данном контексте три ``стрелочки'' имеют схожие значения:
\begin{enumerate}
  \item{}$\vdash$ означает выводимость за одно применение правил вывода.
  \item{}$\to$ означает доказуемость.
  \item{}$\implies$ --- знак алфавита формальной системы.
\end{enumerate}

{\it Теорема:}
Если $T$ --- простая тавтология, то
для произвольного набора формул $\Sx$ справедливо\footnote{
  Справедливо --- истинно, доказуемо.
} $T(\Sx/\Px)$
($\Px$ и $\Sx$ определены как в правиле \taut{}).

  {\it Доказательство:} Нетрудно заметить, что если
$T$ --- простая тавтология, то и
${\top\implies T}$ --- простая тавтология. Тогда по правилу \taut{} имеем
$\top\vdash T(\Sx/\Px)$.\qed

Докажем также, что если $A\to B$, то $A\implies B$ --- теорема.

{\it Доказательство:}
Из $A\to B$ следует $\top,A\to B$. Тогда по правилу \implic{} имеем
$\top\vdash A\implies B$.\qed

% Если $T$ --- теорема, то $\to T$.
% 
%   {\it Доказательство}
% Пусть теорема $T$ выводима из аксиом $A_1,...,A_{n}$.
% \begin{enumerate}[label=(\roman*)]
%   \item{}\label{1p}Предположим произвольную формулу $S$.
%   \item{}\label{2p}По \axiom{} из \ref{1p} можем вывести
%   формулы $A_1,...,A_{n}$.
%   \item{}\label{3p}$A_1,...,A_{n}\to T$, значит из \ref{2p} можем вывести $T$.\qed
% \end{enumerate}
% 
% Таким образом, $T$ истинна ттк $\to T$, значит для доказательства теоремы $T$
% достаточно показать $\to T$.
% 
% Строку \ref{1p} можно опустить, но подразумевать.
% Для доказательства $T$ сначала подразумеваем предположение $S$, потом
% выводим $T$. В ходе доказательства необходимые
% аксиомы и теоремы можно вывести из $S$.
% 
% Если $\to T$, то в доказательство в нормальном виде
% можно добавить строку $T$, подразумевая, что она была выведена из $S$.
% 
% \textsc{Понятия выводимости из ниоткуда не существует,
%   самое близкое к нему --- выводимость из произвольной формулы.}

\section{Кванторы}

\newcommand\Aii{$\forall$I}
\newcommand\Aee{$\forall$E}
\newcommand\Eii{$\exists$I}
\newcommand\Eee{$\exists$E}

\index{правило!введения}\index{правило!исключения}\index{правило!использования}
Логические союзы часто вводят набором из двух правил:
{\it правило введения} ($\lambda$I\footnote{I --- Introduction, введение}) и
{\it правило исключения, использования}
($\lambda$E\footnote{E --- Elimination, исключение}).

Начнём введение правил с квантора всеобщности. $(\forall x)P(x)$ означает
$P(x)$ для всякого $x$, что можно вывести из
доказуемости $P(x)$ для произвольного $x$.
Сформулируем это в правилах
\begin{enumerate}
  \item[(\Aii{})]{}$\Gamma\vdash(\forall x)P(x)$, если ${\Gamma\to P(x)}$, где
  $P(x)$ --- формула о $x$ и $x$ не свободна ни в одной из формул
  в совокупности $\Gamma$.\index{правило!\Aii}

  То есть если $\Gamma$ ``не ограничивает'' переменную $x$,
  то $P(x)$ доказано для произвольного объекта $x$.

  \item[(\Aee{})]{}$(\forall x)P(x)\vdash P(t)$, где $t$ --- терм
  (не обязательно переменная).
  % , ни одна из переменных коротого не связанна в $P$.
  \index{правило!\Aee}
\end{enumerate}

Для квантора существования вводят следующие правила:
\begin{enumerate}
  \item[(\Eii{})]${F\vdash [\exists x:F(x/'t)]}$, где $t$ --- терм,
  ни одна из переменных которого не связанна в $F$.
  Заметим схожесть с правилом $\forall$E.
  \index{правило!\Eii}

  \item[(\Eee{})]${\Gamma, [\exists x:F]\vdash C}$, если $\Gamma, F\to C$,
  где $F$ --- формула о $x$ и $x$ не свободна ни в $C$, ни в одной из формул $\Gamma$.
  Заметим схожесть с правилом $\forall$I.
  \index{правило!\Eee}
\end{enumerate}

Докажем, что знак связанной переменной не имеет значения.

{\it Теорема:}
$(\forall x)P(x)\implies (\forall t)P(t)$

{\it Доказательство:}
\begin{enumerate}[label=(\arabic*)]
  \item{}Пусть $(\forall x)P(x)$.
  \item{}$P(t)$ по \Aee{}.
\end{enumerate}
Из рассуждений получаем $(\forall x)P(x)\to P(t)$,
поэтому по \Aii{} имеем $(\forall x)P(x)\vdash (\forall t)P(t)$.
Тогда по ранее показанному результату
можем доказать $(\forall x)P(x)\implies (\forall t)P(t)$.\qed

Аналогичную теорему можно доказать и для квантора существования.

Как и предположения аксиом, будем опускать из доказательств вывод
теорем. Так, например, теоремами вида $A\implies B$
можно будет пользоваться как правилами вида $A\vdash B$:
\begin{enumerate}[label=(\arabic*)]
  \item{}\label{thm_ex_1}Пусть $A$.
  \item{}\label{thm_ex_2}$B$ по \taut{}, $p\land (p\implies q)\implies q$,
  \ref{thm_ex_1} и теореме $A\implies B$.
\end{enumerate}
Обычно пояснение строки \ref{thm_ex_2} сокращают до ссылки на теорему
и подразумевают использование modus ponens.

Законы отрицания кванторов можно разделить на четыре теоремы.
Докажем одну из них.

% \vspace{1em}
{\it Теорема:} $\lnot [\exists x:P(x)]\implies  (\forall x)~\lnot P(x)$

{\it Доказательство:}
\begin{enumerate}[label=(\arabic*)]
  \item{}\label{qneg_1}Пусть $\lnot[\exists x:P(x)]$.
  \item{}\label{qneg_2}Пусть $P(x)$.
  \item{}\label{qneg_3}$\exists x:P(x)$ по \Eii{}, \ref{qneg_2}.
  \item{}\label{qneg_4}$\bot$ по \taut{}, $p\land\lnot p\implies \bot$,
  \ref{qneg_1}, \ref{qneg_3}.
\end{enumerate}
Рассуждения показывают $\lnot[\exists x:P(x)], P(x)\to \bot$.
\begin{enumerate}[label=(\arabic*)]
  \item{}\label{qneg_5}Пусть $\lnot[\exists x:P(x)]$.
  \item{}\label{qneg_6}$P(x)\implies\bot$ по \implic{},
  ранее показанному и \ref{qneg_5}.
  \item{}\label{qneg_7}$\lnot P(x)$ по \taut{},
  $(p\implies\bot)\implies\lnot p$ и \ref{qneg_6}.
\end{enumerate}
Рассуждения показывают $\lnot[\exists x:P(x)]\to\lnot P(x)$.
Тогда по \Aii{} получим $\lnot[\exists x:P(x)]\vdash(\forall x)~\lnot P(x)$
и по ранее показанному результату
можем завершить доказательство.\qed

\vspace{1em}
{\it Упражнения:}
\begin{enumerate}
  \item{}Обосновать ограничения на связанность и свободу переменных в правилах
  \Aii{}, \Aee{}, \Eii{} и \Eee{}.
  \item{}Доказать теоремы:
  \begin{enumerate}
    \item{}$[(\forall x)P(x)]\land[(\forall x)(P(x)\implies Q(x))]
      \implies [(\forall x)Q(x)]$\label{thm:obv_forall}
    \item{}$[\exists x:P(x)]\land[(\forall x)(P(x)\implies Q(x))]
      \implies [\exists x:Q(x)]$
    \item{}$[(\forall x)P(x)]\land [\exists x:(P(x)\implies Q(x))]
      \implies [\exists x:Q(x)]$
    \item{}$[\exists x:F(x)]\implies [\exists t:F(t)]$
    \item{}$(\forall x)T(F/p)$, где $T$ --- простая тавтология,
    в которой $p$ --- единственный символ переменной, $F$ --- формула о $x$.
    \item{}$(\forall x)T(\Sx/\Px)$, где $T$ --- простая тавтология,
    $\Px$ --- символы её переменных, $\Sx$ --- формулы о $x$.\label{thm:obv_taut}
  \end{enumerate}
\end{enumerate}

\section{Цепочки импликаций}

Доказательства в нормальном виде на практике редко используются из-за их
громоздкости.
Чаще доказательства в математике записываются в виде {\it цепочек импликаций}
\index{цепочка!импиликаций} с рассуждениями.
Пусть в мы уже вывели совокупность формул $\Gamma$~и
\[
  \Gamma,A_1\to A_2\qquad \Gamma,A_2\to A_3
  \qquad  ... \qquad \Gamma,A_{n-1}\to A_{n},
\]
тогда по правилу \implic{} из $\Gamma$ можем вывести импликации
${A_{k}\implies A_{k+1}}$, которые кратко можно записать в виде цепочки
\[
  A_1\implies A_2\implies ...\implies A_{n}
\]
По \taut{} и тавтологии
\[
  (p\implies q)\land (q\implies r)\implies (p\implies r)
\]
построение такой цепочки доказывает $A_1\implies A_{n}$ из $\Gamma$.

Цепочки импликаций также можно назвать ``человеческим видом'',
поскольку они более понятны при чтении.

В человеческом виде часто используют следующие обороты:
\begin{multicols}{2}
  \begin{enumerate}[label=(\roman*)]
    \item{}
    Имеем $A$.\\
    $A\implies...\implies B$\\
    Тогда $B$.
    \item{}
    Пусть $A$.\\
    ...\\
    Тогда $B$.\\
    Тогда $A\implies B$.
    \item{}
    Имеем $A$.\\
    $B\implies ...\implies \bot$\\
    Тогда $\lnot B$.
    \item{}
    Имеем $A$. Пусть $B$.\\
    ...\\
    Тогда $\bot$.\\
    Тогда $\lnot B$.
  \end{enumerate}
\end{multicols}

\vspace{1em}
{\it Теорема:} $\lnot[\exists x:P(x)]\implies [(\forall x)~\lnot P(x)]$

{\it Доказательство:}
Пусть ${\lnot[\exists x:P(x)]}$.
Возьмём произвольный\footnote{
  Используя такие слова мы намекаем на смысл, который мы придаём доказательству
  формулы с $\forall x$.

  ``Произвольным'' является объект, обозначаемый термом $x$.
  Терм берётся самый конкретный --- $x$.}~$x$.
\[
  P(x)\xRightarrow{\text{$\exists$I}} [\exists x:P(x)]
  \xRightarrow{p,\lnot p\vdash\bot} \bot
\]
Тогда $\lnot P(x)$ и $(\forall x)~\lnot P(x)$
по $\forall$I.\qed

\vspace{1em}
{\it Теорема:} $\lnot[(\forall x)~P(x)]\implies [\exists x:\lnot P(x)]$

{\it Доказательство:}
Пусть $\lnot[(\forall x)~P(x)]$.
\[
  \lnot[\exists x:\lnot P(x)]\implies [(\forall x)~\lnot\lnot P(x)]
  \implies [(\forall x)~P(x)]\implies\bot
\]
Предпоследняя импликация справедлива по упражнениям
\ref{thm:obv_forall} и \ref{thm:obv_taut} из предыдущей главы.
Тогда $\exists x:\lnot P(x)$.\qed

Вообще можно доказать, что если ${A\iff B}$, то ${F\iff F(A/B)}$,
где $B$ --- подформула $F$.

\vspace{1em}
{\it Теорема:} $[(\forall x)~\lnot P(x)]\implies \lnot[\exists x:P(x)]$

{\it Доказательство:}
Пусть $(\forall x)~\lnot P(x)$. Возьмём произвольный $x$
и предположим $P(x)$.
Имеем $\lnot P(x)$ по $\forall$E. Можем вывести $\bot$.
Таким образом,
\[
  [(\forall x)~\lnot P(x)],P(x)\to \bot,
\]
значит
\[
  [\exists x:P(x)]\xRightarrow{\exists\text{E}}\bot
\]
и $\lnot[\exists x:P(x)]$.\qed

\vspace{1em}
{\it Теорема:} $[\exists x:\lnot P(x)]\implies\lnot[(\forall x)~P(x)]$

{\it Доказательство:}
Пусть $\exists x:\lnot P(x)$. Пусть $\lnot P(x)$ для произвольного $x$.
\[
  (\forall x)~P(x)\implies P(x)\implies \bot
\]
Тогда $\lnot[(\forall x)~P(x)]$ по $\exists$E.\qed

\vspace{1em}
\index{цепочка!эквивалентностей}
Аналогично определяются и {\it цепочки эквивалентностей}
\[
  A_1\iff A_2\iff ...\iff A_{n},
\]
где $\Gamma,A_{k}\to A_{k+1}$ и $\Gamma,A_{k+1}\to A_{k}$.

\vspace{1em}
{\it Упражнения:}
\begin{enumerate}
  \item{}Обосновать каждую импликацию в доказательствах законов отрицания кванторов.
  \item{}*Доказать с помощью рассуждений в нормальном виду законы отрицания кванторов.
\end{enumerate}

\section{Равенство}

Определяющее свойство равенства в том, что если два терма равны, то их можно
заменять друг другом в формулах.

\index{правило!$=$E}\index{правило!$=$I}
Введём бинарный предикатный знак $=$\footnote{
  Тогда, если $a,b$	 --- термы, то выражение $a=b$ --- формула.},
правило $=$E\index{равенство}
\begin{enumerate}
  \item[($=$E)]{}$P,[a=b]\vdash P(a/'b)$,
  где $P$ --- формула, $a,b$ --- термы.
\end{enumerate}
и {\it Аксиому Равенства}: $(\forall x)~x=x$.\index{Аксиома!Равенства}

\vspace{1em}
{\it Теорема:} Пусть $t$ --- терм, тогда
\[
  (\forall a,b)~a=b\implies t=t(a/b)
\]
То есть, например $a=b\implies f(a)=f(b)$.

  {\it Доказательство:}
Возьмём произвольные $a$ и $b$. Пусть $a=b$.
По Аксиоме Равенства имеем $t=t$, по $=$E имеем $t=t(a/b)$.
То есть
\[
  a=b\implies t=t(a/b)
\]
Можем обобщить по $\forall$I и закончить доказательство.\qed

\pagebreak

Равенство позволяет нам ввести понятие единственности.
Как и ранее, определим сокращение квантора единственности.

{\it Теорема:} $\forall a~\exists !b:a=b$

{\it Доказательство:}
Докажем существование.
Возьмём произвольный $a$. Имеем $a=a$ по Аксиоме Равенства,
тогда $\exists b:a=b$.

Докажем единственность. Пусть для произвольных $b_1,b_2$
справедливы $a=b_1$ и $a=b_2$.
По правилу $=$E получим $b_1=b_2$.

Таким образом,
\[
  [\exists b:a=b]\land[(\forall b_1,b_2)~a=b_1\land a=b_2\implies b_1=b_2],
\]
то есть $\exists !b:a=b$, что было доказанно для произвольного $a$.\qed

\vspace{1em}
{\it Упражнения:}
\begin{enumerate}
  \item{}Доказать теоремы
  \begin{enumerate}
    \item{}$(\forall x,y)~x=y\iff y=x$
    \item{}$(\forall x,y,z)~x=y\land y=z\implies x=z$
  \end{enumerate}
\end{enumerate}

\section{Введение новых знаков}

Введём правила введения новых функциональных
и предикатных знаков.

Пусть $\varphi(t_1,...,t_{n})$\footnote{
Если $F$ --- формула о $x_1,...,x_{n}$, то
\[
  F(x_1,...,x_{n})\equiv F
\]
\[F(y_1,...,y_{n})\equiv F(y_1/x_1)...(y_{n}/x_{n})\]} --- формула
только о $t_1,...,t_{n}$. Можем ввести новый $n$-арный предикатный знак $\eta$
и аксиому, его определяющую:
\[
  (\forall t_1,t_2,...,t_{n})~\varphi(t_1,...,t_{n})\iff\eta(t_1,...,t_{n})
\]
Часто  кванторы, относящиеся к переменным, опускают
при определении новых предикатных знаков.

Например, пусть $<$ --- бинарный предикатный знак.
Можем ввести новые предикатные знаки $\leq$ и $>$.
\[
  x\leq y\iff x<y\lor x=y\qquad x>y\iff y< x
\]
Для каждого бинарного предикатного знака $\prec$ обычно
вводят его отрицание --- знак $\nprec$.
\[
  x\nprec y\iff \lnot(x\prec y)
\]

\index{область!определения}
Функция --- правило, по которому объекту
из определённой совокупности ({\it области определения})
сопоставляется единственный объект.
Пусть формула $P(x_1,...,x_{n})$ определяет условие, при котором функция определена,
а $\varphi(x_1,...,x_{n},y)$ определяет значение функции, причём в них свободны
только переменные в скобках. Если доказуемо
\[
  \forall x_1...\forall x_n[P(x_1,...,x_{n})
  \implies \exists! y:\varphi(x_1,...,x_{n},y)],
\]
то можем расширить формальную систему новым $n$-арным фнукциональным знаком $f$
и аксиомой
\[
  \forall x_1...\forall x_{n}[P(x_1,...,x_{n})
  \implies \varphi(x_1,...,x_{n},f(x_1,...,x_{n}))]
\]

Пусть $T,T'$ --- формальная система до и после таких расширений.
Если формула $\psi$ доказуема в $T'$ и является формулой в $T$, то она доказуема в $T$.
Из этого следует, что внутри доказательства можно расширить формальную систему
введением новых функций, не влияя на справедливость доказательства.

Говорят, что $f$ {\it определена}\index{определённость функции}
для всех $x_1,...,x_{n}$ таких, что $P(x_1,...,x_{n})$.
Если $P\equiv \top$, то формулы выше упрощают до
\[
  \forall x_1,...,x_{n}~\exists !y:\varphi(x_1,...,x_{n},y)\qquad
  (\forall x_1,...,x_{n})~\varphi(x_1,...,x_{n},f(x_1,...,x_{n}))
\]

Рассмотрим примеры.
Пусть $+$ --- бинарный функциональный знак,
определённый для всех аргументов.
Возьмём формулу $y=(x+x)+a$. Очевидно,
\[
  \forall x,a~\exists !y:y=(x+x)+a
\]
Можем ввести бинарный функциональный знак $g$ и аксиому
\[
  (\forall x,a)~g(x,a)=(x+x)+a
\]
Часто некоторые аргументы опускают. Например, $g(x)\equiv g(x,a)$.

Часто рассуждения выше сокращают до
\[
  g(x):=(x+a)+a,
\]
где ``$:=$'' читают как ``равно по определению''
\index{равно по определению, $:=$}.

Пусть $\cdot$ --- бинарный функциональный знак,
определённый для всех аргументов, а
$0$ и $1$ --- константы и справедливо
\[
  (\forall a\neq 0)~\exists !b:a\cdot b=1
\]
Можем ввести унарный функциональный знак $f$ и аксиому
\[
  (\forall a\neq 0)~a\cdot f(a)=1
\]
Обычно такую функцию обозначают как $(\cdot)^{-1}$,
то есть $a^{-1}\equiv f(a)$.
Функция $(\cdot)^{-1}$ определена для всех $x\neq 0$.

Часто рассуждения выше сокращают до
\[
  (\forall a\neq 0)~\exists !a^{-1}:a\cdot a^{-1}=1
\]
