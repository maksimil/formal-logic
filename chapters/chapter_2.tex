\part{Вывод теорем}

\section{Формальная система}

Совокупность правил называется {\it формальной системой}\index{формальная система},
если выполняются следующие условия:
\begin{enumerate}
	\item{}Задан {\it алфавит}.
	\item{}Определено понятия {\it формулы}.
	\item{}Задана совокупность формул, называемых {\it аксиомами}\index{Аксиома}.
	\item{}Заданы {\it правила вывода}\index{правило!вывода} одних формул из других.
\end{enumerate}

Если формула $T$ выводима из формул $A_1,A_2,...,A_{n}$ за один шаг (одно применение
правил вывода), то пишут $A_1,A_2,...,A_{n}\vdash T$.

Слева от $\vdash$ порядок формул не имеет значения и могут быть излишние формулы:
если $A,B\vdash T$, то $B,A\vdash T$ и $B,A,C\vdash T$.

\newcommand\Sx{\mathcal S}
Возьмём совокупность формул $\Sx$. Пусть
\[
	\Sx\vdash A_1\qquad
	\Sx,A_1\vdash A_2\qquad...\qquad
	\Sx,A_1,A_2,...,A_{n-1}\vdash A_{n}
\]

Тогда пишут $\Sx\to A_{n}$ и говорят, что $A_{n}$
{\it доказуемо из}\index{доказуемость} $\Sx$.
Очевидно,	$A\vdash B$ подразумевает $A\to B$. Слева от $\to$ порядок формул
не имеет значения и могут быть излишние формулы.
Рассуждения, показывающие доказуемость называются
{\it доказательством}\index{доказательство}.
Если $\Sx$~---~совокупность аксиом и $\Sx\to T$, то $T$ называют
{\it теоремой}\index{теорема}.

\newcommand\ruleR{\mathbf{R}}
\newcommand\ruleC{\mathbf{C}}

Рассмотрим следующий пример формальной системы:
\begin{enumerate}
	\item{}Алфавит: $a$, $b$, $c$.
	\item{}Всякое непустое выражение является формулой.
	\item{}Аксиомы: $aab$, $c$.
	\item{}Правила вывода:
	\begin{multicols}{2}
		\begin{enumerate}
			\item[($\ruleR$)]{}$A\vdash A'$, если $A'$ можно получить из $A$,
			убрав один символ. Тогда $aab\vdash ab$.
			\columnbreak
			\item[($\ruleC$)]{}$A,B\vdash AB$, если у формул $A$ и $B$ нет
			общих символов. Тогда $b,c\vdash bc$ и $ac,bb\vdash acbb$.
		\end{enumerate}
	\end{multicols}
\end{enumerate}

{\it Теорема:} $abc$

{\it Доказательство:}
$aab\vdash ab$ по правилу $\ruleR$, значит $aab,c\vdash ab$.
$ab,c\vdash abc$ по правилу $\ruleC$, значит $aab,c,ab\vdash abc$.
Тогда $aab,c\to abc$ и $abc$ --- теорема, потому что $aab$ и $c$ --- аксиомы.
\qed\footnote{$\qed$ означает ``что и требовалось доказать''.}

Доказуемость можно показать с помощью рассуждений в
{\it нормальном виде}\index{нормальный вид}:
Чтобы показать $\Sx\to T$ можно предположить формулы из $\Sx$ и последовательно
увеличивать список выведенных формул, пока не дойдём до $T$.

\pagebreak

Доказательство $abc$ в нормальном виде:
\begin{enumerate}[label=(\arabic*)]
	\item{}\label{1a}Пусть (предположим) $aab$.
	\item{}\label{2a}Пусть $c$.
	\item{}\label{3a}$ab$ по $\ruleR$, \ref{1a}.
	\item{}\label{4a}$abc$ по $\ruleC$, \ref{2a}, \ref{3a}.\qed
\end{enumerate}

\textsc{Формальная система является просто набором правил и её формулы
	могут не иметь смысл. Они и не будут иметь смысл, пока мы не начнём их
	интерпретировать, то есть придавать им этот смысл.}

В общем случае формальные системы не обращаются с понятиями ``истины'' и ``лжи'',
они обращаются с понятиями ``выводимости'', но для удобства определим формулу
как {\it истинную}\index{истинность}\index{формула!истинная},
если она является аксиомой или теоремой.

\section{Правила вывода}

Начнём описывать формальную систему, в которой мы будем работать.
Оставим тот же самый алфавит, понятия терма, переменной и формулы:
\begin{fullwidth}
	\begin{multicols}{2}
		\begin{enumerate}
			\item{}Любой знак переменной является термом.
			\item{}Если $f$ --- функциональный знак арности $n$, и $t_1,...,t_{n}$ --- термы,
			то $f(t_1,...,t_{n})$ --- терм.
			\item{}Если $\phi$ --- предикатный знак арности $n$, и $t_1,...,t_{n}$ --- термы,
			то $\phi(t_1,...,t_{n})$ --- формула.

			\columnbreak

			\item{}Пусть $F_1,F_2$ --- формулы, $F$ --- формула о $x$, тогда выражения
			\[
				\lnot F_1\qquad F_1\land F_2\qquad F_1\lor F_2\qquad
				F_1\implies F_2
			\]
			\[
				F_1\iff F_2\qquad (\forall x)~F\qquad
				\exists x:F
			\]
			являются формулами.
		\end{enumerate}
	\end{multicols}
\end{fullwidth}

\newcommand\taut{$\mathcal T$}
\newcommand\axiom{$\mathcal A$}
\newcommand\implic{$\mathcal I$}
\newcommand\Px{\mathcal P}
Возьмём формулу $\top$\footnote{Константы $\top$ и $\bot$ вводятся как
	$0$-арные предикатные знаки.}
как аксиому и начнём вводить правила вывода.
Введём правило \axiom{} для более удобного использования аксиом в доказательствах.
\begin{enumerate}
	\item[(\axiom)]{}$S\vdash A$, где $A$ --- аксиома, $S$ --- любая формула.
\end{enumerate}

Если $S\to T$ или $S\vdash T$ для произвольной формулы $S$,
то пишут $\to T$ и $\vdash T$ соответственно. \axiom{} можно переписать:
\begin{enumerate}
	\item[(\axiom)]{}$\vdash A$, где $A$ --- аксиома.\index{правило!\axiom}
\end{enumerate}

Если $T$ выводимо из $A$ и $B$, то из $A$ должна быть выводима формула $B\implies T$,
поэтому введём правило \implic{}.
\begin{enumerate}
	\item[(\implic)]{}${\Gamma\vdash (S\implies T)}$, если $\Gamma,S\to T$,
	\index{правило!\implic}
	где $\Gamma$ --- совокупность формул\footnote[][-2cm]{
		Три ``стрелочки'' имеют схожие значения:
		\begin{enumerate}
			\item{}$\vdash$ означает выводимость за одно применение правил вывода.
			\item{}$\to$ означает доказуемость.
			\item{}$\implies$ --- знак из алфавита формальной системы.
		\end{enumerate}
	}.
\end{enumerate}

Для использования законов логики (modus ponens, modus tollens, закон
исключённого тертьего) введём правило \taut{}.
\begin{enumerate}
	\item[(\taut)]{}$F_1(\Sx/\Px),...,F_n(\Sx/\Px)\vdash T(\Sx/\Px)$, если
	\index{правило!\taut}
	\begin{enumerate}
		\item{}Выражение ${F_1\land...\land F_n\implies T}$ --- простая тавтология.
		\item{}$\Px$ --- совокупность всех знаков переменных тавтологии.
		\item{}$\Sx$ --- совокупность формул, причём каждому знаку из $\Px$
		поставлена в соотвествие единственная формула из $\Sx$.
	\end{enumerate}
\end{enumerate}

Например, возьмём простую тавтологию $(p\land (p\implies q))\implies q$.
\[
	F_1\equiv p\qquad F_2\equiv p\implies q\qquad T\equiv q
\]
Совокупность $\Px$ содержит $p$ и $q$.
Совокупность $\Sx$ содержит произвольные формулы $A$ и $B$.
По \taut{} имеем правило ${A,[A\implies B]\vdash B}$
для произвольных формул $A,B$.
Исходя из той же тавтологии также имеем правило ${A\land[A\implies B]\vdash B}$.

Докажем два занимательных факта:
\begin{enumerate}
	\item{}
	      {\it Теорема:}
	Если $T$ --- простая тавтология, то
	для произвольного набора формул $\Sx$ справедливо\footnote{
		Справедливо --- истинно, доказуемо.
	} $T(\Sx/\Px)$
	($\Px$ и $\Sx$ определены как в правиле \taut{}).

		{\it Доказательство:}
	$T$ --- простая тавтология, значит и
	${\top\implies T}$ --- простая тавтология. Тогда по правилу \taut{} имеем
	$\top\vdash T(\Sx/\Px)$.\qed

	\item{}Если $T$ --- теорема, то $\to T$.

		{\it Доказательство:}
	Пусть теорема $T$ выводима из аксиом $A_1,...,A_{n}$.
	\begin{enumerate}[label=(\roman*)]
		\item{}\label{1p}Предположим произвольную формулу $S$.
		\item{}\label{2p}По \axiom{} из \ref{1p} можем вывести
		формулы $A_1,...,A_{n}$.
		\item{}\label{3p}$A_1,...,A_{n}\to T$, значит из \ref{2p} можем вывести $T$.\qed
	\end{enumerate}
\end{enumerate}

Таким образом, $T$ истинна ттк $\to T$, значит для доказательства теоремы $T$
достаточно показать $\to T$.

Строку \ref{1p} можно опустить, но подразумевать.
Для доказательства $T$ сначала подразумеваем предположение $S$, потом
выводим $T$. В ходе доказательства необходимые
аксиомы и теоремы можно вывести из $S$.

Если $\to T$, то в доказательство в нормальном виде
можно добавить строку $T$, подразумевая, что она была выведена из $S$.

\textsc{Понятия выводимости из ниоткуда не существует,
	самое близкое к нему --- выводимость из произвольной формулы.}

\section{Кванторы}

\newcommand\Aii{$\forall$I}
\newcommand\Aee{$\forall$E}
\newcommand\Eii{$\exists$I}
\newcommand\Eee{$\exists$E}

Новый знак можно ввести набором из двух правил:
$\lambda$I\footnote{I --- Introduction, введение} --- {\it правило введения}
\index{правило!введения}
и $\lambda$E\footnote{E --- Elimination, исключение} --- {\it правило
исключения (использования)}\index{правило!исключения}\index{правило!использования},
где $\lambda$ --- знак, который мы хотим ввести.

Начнём введение правил с квантора всеобщности. $(\forall x)P(x)$ означает
$P(x)$ для всякого $x$, значит оно доказуемо для произвольного $x$.
Это можно сформулировать в правилах
\begin{enumerate}
	\item[(\Aii{})]{}$\Gamma\vdash(\forall x)P(x)$, если ${\Gamma\to P(x)}$, где
	$P(x)$ --- формула о $x$ и $x$ не свободна ни в одной из формул
	в совокупности $\Gamma$.\index{правило!\Aii}

	То есть если $\Gamma$ ``не ограничивает'' переменную $x$,
	то $P(x)$ доказано для произвольного объекта $x$.

	\item[(\Aee{})]{}$(\forall x)P(x)\vdash P(t)$, где $t$ --- терм
	(не обязательно переменная), ни одна из переменных коротого не связанна в $P$.
	\index{правило!\Aee}
\end{enumerate}

Для квантора существования вводятся следующие правила:
\begin{enumerate}
	\item[(\Eii{})]${F\vdash [\exists x:F(x/'t)]}$, где $t$ --- терм,
	ни одна из переменных которого не связанна в $F$.
	Заметим схожесть с правилом $\forall$E.
	\index{правило!\Eii}

	\item[(\Eee{})]${\Gamma, [\exists x:F]\vdash C}$, если $\Gamma, F\vdash C$,
	где $F$ --- формула о $x$ и $x$ не свободна ни в $C$, ни в одной из формул $\Gamma$.
	Заметим схожесть с правилом $\forall$I.
	\index{правило!\Eee}
\end{enumerate}

Из этих правил также следует, что знак связанной переменной не имеет значения.

При построении формальной системы мы сначала
выбираем, какой смысл мы придаём обозначениям, а потом вводим
правила и акисомы для формализации этого смысла.
\textsc{С помощью правил и аксиом мы формализуем смысл, придаваемый формулам и термам.}

\vspace{1em}
{\it Теорема:} $\lnot [\exists x:P(x)]\implies  (\forall x)\lnot P(x)$

{\it Доказательство:}
\begin{enumerate}[label=(\arabic*)]
	\item{}\label{2}Предположим $\lnot [\exists x:P(x)]$.
	\item{}\label{4}Предположим $P(x)$.
	\item{}\label{5}По \Eii{}, \ref{4} имеем $\exists x:P(x)$.
	\item{}\label{6}По \implic{}, \ref{4}-\ref{5} имеем
	$P(x)\implies \exists x:P(x)$.
	\item{}\label{7}${[p\implies q],\lnot q\vdash \lnot p}$ по \taut{}\footnote{
		Опираясь на тавтологию
		\[
			[(p\implies q)\land\lnot q]\implies\lnot p
		\]
	}, из
	\ref{2} и \ref{6} можем вывести $\lnot P(x)$.
	\item{}\label{8}Рассуждения \ref{2}-\ref{7} показывают
	$\lnot[\exists x:P(x)]\to\lnot P(x)$, значит по \Aii{} из \ref{2} можем
	вывести $(\forall x)\lnot P(x)$.\qed
\end{enumerate}

\pagebreak
% \vspace{1em}
{\it Упражнения:}
\begin{enumerate}
	\item{}Обосновать ограничения на связанность и свободу переменных в правилах
	\Aii{}, \Aee{}, \Eii{} и \Eee{}.
	\item{}\label{ex:obv_thm}Доказать теоремы:
	\begin{enumerate}
		\item{}$[(\forall x)P(x)]\land[(\forall x)(P(x)\implies Q(x))]
			\implies [(\forall x)Q(x)]$\label{thm:obv_forall}
		\item{}$[\exists x:P(x)]\land[(\forall x)(P(x)\implies Q(x))]
			\implies [\exists x:Q(x)]$
		\item{}$[(\forall x)P(x)]\land [\exists x:(P(x)\implies Q(x))]
			\implies [\exists x:Q(x)]$
		\item{}$(\forall x)T(F/p)$, где $T$ --- простая тавтология,
		в которой $p$ --- единственный символ переменной, $F$ --- формула о $x$.
		\item{}$(\forall x)T(\Sx/\Px)$, где $T$ --- простая тавтология,
		$\Px$ --- символы её переменных, $\Sx$ --- формулы о $x$.\label{thm:obv_taut}
	\end{enumerate}
\end{enumerate}

\section{Цепочки импликаций}

Доказательства в нормальном виде на практике редко используются из-за их
громоздкости.
Чаще доказательства в математике записываются в виде {\it цепочек импликаций}
\index{цепочка!импиликаций} с рассуждениями.
Пусть в мы уже вывели совокупность формул $\Gamma$~и
\[
	\Gamma,A_1\to A_2\qquad \Gamma,A_2\to A_3
	\qquad  ... \qquad \Gamma,A_{n-1}\to A_{n},
\]
тогда по правилу \implic{} из $\Gamma$ можем вывести импликации
${A_{k}\implies A_{k+1}}$, которые кратко можно записать в виде цепочки
\[
	A_1\implies A_2\implies ...\implies A_{n}
\]
По тавтологии
\[
	[(p\implies q)\land (q\implies r)]\implies (p\implies r)
\]
построение такой цепочки доказывает $A_1\implies A_{n}$ из $\Gamma$.

Цепочки импликаций также можно называть ``человеческим видом'', потому что они более
понятны при чтении.
В человеческом виде часто следующие обороты:
\begin{fullwidth}
	\begin{multicols}{4}
		\begin{enumerate}[label=(\roman*)]
			\item{}
			Пусть $A$.\\
			$A\implies...\implies B$\\
			Тогда $B$.
			\item{}
			Пусть $A$.\\
			...\\
			Тогда $B$.\\
			Тогда $A\implies B$.
			\item{}
			Пусть $A$.\\
			$B\implies ...\implies \bot$\\
			Тогда $\lnot B$.
			\item{}
			Пусть $A$. Пусть $B$.\\
			...\\
			Тогда $\bot$.\\
			Тогда $A\implies \lnot B$.
		\end{enumerate}
	\end{multicols}
\end{fullwidth}

% \vspace{1em}
{\it Теорема:} $\lnot[\exists x:P(x)]\implies (\forall x)\lnot P(x)$

{\it Доказательство:}

Пусть ${\lnot[\exists x:P(x)]}$.
Возьмём произвольный\footnote{
	Используя такие слова мы намекаем на смысл, который мы придаём доказательству
	формулы с $\forall x$.

	``Произвольным'' является объект, обозначаемый термом $x$.
	Терм берётся самый конкретный --- $x$.} $x$.
\[
	P(x)\xRightarrow{\text{$\exists$I}} \exists x:P(x)
	\xRightarrow{p,\lnot p\vdash\bot} \bot
\]

Тогда $\lnot P(x)$ и $(\forall x)\lnot P(x)$
по $\forall$I.\qed

\pagebreak
% \vspace{1em}
{\it Теорема:} $\lnot[(\forall x)P(x)]\implies \exists x:\lnot P(x)$

{\it Доказательство:}

Пусть $\lnot[(\forall x)P(x)]$.
\[
	\lnot[\exists x:\lnot P(x)]\implies (\forall x)\lnot\lnot P(x)
	\xRightarrow{\lnot\lnot q\implies q} (\forall x)P(x)\implies\bot
\]

Тогда $\exists x:\lnot P(x)$.\qed

\vspace{1em}
{\it Теорема:} $(\forall x)\lnot P(x)\implies \lnot[\exists x:P(x)]$

{\it Доказательство:}

Пусть $(\forall x)\lnot P(x)$. Предположим $P(x)$ для произвольного $x$.

Имеем $\lnot P(x)$ по $\forall$E, пришли к $\bot$, тогда
\[
	[\exists x:P(x)]\xRightarrow{\text{$\exists$E}}\bot
\]

Тогда $\lnot[\exists x:P(x)]$.\qed

\vspace{1em}
{\it Теорема:} $\exists x:\lnot P(x)\implies\lnot[(\forall x)P(x)]$

{\it Доказательство:}

Пусть $\exists x:\lnot P(x)$. Пусть $\lnot P(x)$ для произвольного $x$.
\[
	(\forall x)P(x)\implies P(x)\implies \bot
\]

Тогда $\lnot[(\forall x)P(x)]$ по $\exists$E.\qed

\vspace{1em}
Аналогично определяются и цепочки эквивалентностей\index{цепочка!эквивалентностей}
\[
	A_1\iff A_2\iff ...\iff A_{n},
\]
где $\Gamma,A_{k}\to A_{k+1}$ и $\Gamma,A_{k+1}\to A_{k}$.

\vspace{1em}
{\it Упражнения:}
\begin{enumerate}
	\item{}Обосновать каждую импликацию в доказательствах законов отрицания кванторов.
	\item{}*Доказать в нормальном виде законы отрицания кванторов.
\end{enumerate}

\section{Равенство}

Определяющее свойство равенства в том, что если два терма равны, то их можно
заменять друг другом в формулах.

Введём новый бинарный предикатный знак $=$\footnote{
	Тогда, если $a,b$	 --- термы, то $a=b$ --- формула.}
и правило $=$E\index{равенство}\index{правило!$=$E}\index{правило!$=$I}
\begin{enumerate}
	\item[($=$E)]{}$P,[a=b]\vdash P(a/'b)$,
	где $P$ --- формула, $a,b$ --- термы.
\end{enumerate}
и {\it Аксиому Равенства}: $\forall x(x=x)$.\index{Аксиома!Равенства}

\vspace{1em}
{\it Теорема:} Пусть $t$ --- терм, тогда
\[
	\forall a\forall b(a=b\implies t=t(a/b))
\]

{\it Доказательство:}

Возьмём произвольные $a$ и $b$. Пусть $a=b$.
По Аксиоме Равенства имеем $t=t$, по $=$E имеем $t=t(a/b)$.
Тогда
\[
	a=b\implies t=t(a/b)
\]

Можем обобщить по $\forall$I и получить
\[
	\forall a\forall b(a=b\implies t=t(a/b))
	\qed
\]

Равенство позволяет нам ввести понятие единственности.
Не будем вводить $\exists!$ в язык формальной системы, а просто определим сокращение
\[
	[\exists! x:P(x)]\equiv[\exists x:P(x)]\land
	[\forall x\forall y(P(x)\land P(y)\implies x=y)],
\]
где $P(x)$ --- формула о $x$, не содержащая $y$.

\vspace{1em}
{\it Упражнения:}
\begin{enumerate}
	\item{}Доказать теоремы
	\begin{enumerate}
		\item{}$(\forall a)~\exists! b:a=b$\footnote{
			Для доказательства $\exists!$ обычно сначала
			доказывают существование, а потом единственность.
		}
		\item{}$(\forall x,y)(x=y\iff y=x)$
		\item{}$\forall x,y,z[(x=y)\land (y=z)\implies (x=z)]$
	\end{enumerate}
\end{enumerate}

\section{Введение новых знаков}

Введём правила для введения сокращений, то есть правила введения новых функциональных
и предикатных знаков.

Начнём с предикатных знаков. Пусть $\varphi$ --- формула,
в которой $t_1,...,t_{n}$ --- все
свободные переменные. Тогда можем определить сокращение $\eta(t_1,...,t_{n})\equiv\varphi$
и обращаться с ним как с новым предикатным знаком арности $n$.
Очевидно, выполняется
\[
	\forall t_1...\forall t_{n}[\eta(t_1,...,t_{n})\iff\varphi]
\]

Вспомните, что аналогичным образом мы ввели $\exists!$. Каждую формулу,
записанную используя такие сокращения можно
``перевести'' на язык формальной системы.

Например, пусть $<$ --- бинарный предикатный знак.
Можем ввести новые предикатные знаки $\leq$ и $>$
\[
	x\leq y\equiv x<y\lor x=y\qquad x>y\equiv y< x
\]

Для каждого бинарного предикатного знака $\prec$ можем
ввести его отрицание --- знак $\nprec$.
\[
	x\nprec y\equiv \lnot(x\prec y)
\]

Перейдём к функциональным знакам. Функция --- правило, по которому объекту
из определённой совокупности ({\it области определения})\index{область!определения}
сопоставляется единственный объект.
Введём $n$-арный функциональный знак. Пусть $P(x_1,...,x_{n})$\footnote{
Если $F$ --- формула о $x_1,...,x_{n}$, то
\[
	F(x_1,...,x_{n})\equiv F
\]
\[F(y_1,...,y_{n})\equiv F(y_1/x_1)...(y_{n}/x_{n})\]}
--- формула, содержащая
условие, при котором функция определена, а $\varphi(x_1,...,x_{n},y)$ ---
формула, определяющая функцию. Если доказуемо
\[
	\forall x_1...\forall x_n[P(x_1,...,x_{n})
	\implies \exists! y:\varphi(x_1,...,x_{n},y)],
\]
где в $P(x_1,...,x_{n})$ и $\varphi(x_1,...,x_{n},y)$
могут быть свободны только
соответствующие переменные в скобках,
то мы можем расширить формальную систему новым $n$-арным фнукциональным знаком $f$
и аксиомой
\[
	\forall x_1...\forall x_{n}[P(x_1,...,x_{n})
	\implies \varphi(x_1,...,x_{n},f(x_1,...,x_{n}))]
\]

Пусть $T,T'$ --- формальная система до и после такого расширения.
Если формула $\psi$ доказуема в $T'$ и является формулой в $T$, то она доказуема в $T$.
Из этого следует, что внутри доказательства можно расширить формальную систему
введением новых функций, не влияя на справедливость доказательства.

Если $P(x)\equiv \top$, то формулы можно упростить до
\[
	\forall x_1...\forall x_{n}[\exists !y:\varphi(x_1,...,x_{n},y)]\qquad
	\forall x_1...\forall x_{n}[\varphi(x_1,...,x_{n},f(x_1,...,x_{n}))]
\]
Говорят, что $f$ {\it определена}\index{определённость функции}
для всех $x_1,...,x_{n}$ таких, что $P(x_1,...,x_{n})$.

Рассмотрим примеры.
\begin{fullwidth}
	\begin{multicols}{2}
		\raggedcolumns
		Пусть $+$ --- бинарный функциональный знак,
		определённый для всех $x,y$.
		Будем использовать для него {\it инфиксную запись:}\index{инфиксная запись}
		\[
			a+b\equiv(+)(a,b)
		\]
		Возьмём формулу $y=(x+a)+a$. Очевидно,
		\[
			\forall x\forall a\exists !y:y=(x+a)+a
		\]

		Можем ввести бинарный функциональный знак $g$ и аксиому
		\[
			(\forall x,a)~g(x,a)=(x+a)+a
		\]

		Обычно в записи формул некоторые аргументы опускают.
		Например, $g(x)\equiv g(x,a)$.

		Часто рассуждения выше сокращают до
		\[
			g(x):=(x+a)+a,
		\]
		где ``$:=$'' означает ``равно по определению''
		\index{равно по определению, $:=$}.
		\columnbreak

		Пусть $\cdot$ --- бинарный функциональный знак,
		определённый для всех $x,y$,
		$0$ и $1$ --- константы (термы) и справедливо
		\[
			\forall a(a\neq 0\implies \exists !b:a\cdot b=1)
		\]

		Можем ввести унарный функциональный знак $f$ и аксиому
		\[
			\forall a(a\neq 0\implies a\cdot f(a)=1)
		\]

		Обычно такую функцию обозначают как $(\cdot)^{-1}$,
		то есть $a^{-1}\equiv f(a)$.

		Часто рассуждения выше сокращают до
		\[
			\forall a(a\neq 0\implies \exists !a^{-1}:a\cdot a^{-1}=1)
		\]

		Заметим, что $(\cdot)^{-1}$ определена для всех $x\neq 0$.
	\end{multicols}
\end{fullwidth}

\pagebreak
