\part{Вывод теорем}

\section{Формальная система}

\index{формальная система}\index{Аксиома}\index{правило!вывода}
Совокупность правил называют {\it формальной системой},
если выполняются следующие условия:
\begin{enumerate}
  \item{}Задан {\it алфавит}.
  \item{}Определено понятия {\it формулы}.
  \item{}Задана совокупность формул, называемых {\it аксиомами}.
  \item{}Заданы {\it правила вывода} одних формул из других.
  Они состоят из утверждений вида ``$T$ выводима из $S_1,...,S_{m}$, если...''
\end{enumerate}

\index{выводимость}
\index{выводимость!элементарная}
\index{элементарное!выводимость}
\index{доказательство}
\index{доказуемость}
\index{теорема}
\newcommand\Sx{\mathcal S}
\newcommand\Ax{\mathcal A}
Определим понятие выводимости. Формула $T$ {\it элементарно выводима}
из формул $S_1,...,S_{m}$ (обозначение: $S_1,...,S_{m}\vdash_{e} T$),
если существует правило вывода, по которому $T$ выводима из формул
$S_1,...,S_{m},A_1,...,A_{n}$, где $A_1,...,A_{n}$ --- аксиомы.
Причём не обязательно использовать все формулы $S_1,...,S_{m}$.

Формула $T$ {\it выводима} из формул\footnote{В данном контексте
$\Gamma$ обозначает несколько формул, например формулы $S_1,...,S_{m}$.} $\Gamma$
(обозначение: $\Gamma\vdash T$), если существует такая последовательность
формул $A_1,A_2,...,A_{n-1}$, что
\[
  \Gamma\vdash_{e} A_1\qquad \Gamma,A_1\vdash_{e} A_2\qquad...\qquad
  \Gamma,A_1,A_2,...,A_{n-1}\vdash_{e} T
\]
То есть каждым шагом мы добавляем новую формулу к выведенным, пока
не достигнем $T$.
Очевидно, из элементарной выводимости следует выводимость.

Если $\Gamma\vdash T$, то также говорят, что $T$ {\it доказуема} из $\Gamma$.
Рассуждения, показывающие доказуемость называют
{\it доказательством}. Если $\vdash T$ (то есть $T$ выводима из аксиом),
то $T$ называют {\it теоремой}.
Часто слова ``теорема'', ``доказуемость'' и ``доказательство'' также используют по
отношению к утверждениям, которые не являются формулами формальной системы.

\newcommand\ruleR{\mathbf{R}}
\newcommand\ruleC{\mathbf{C}}

Рассмотрим следующий пример формальной системы:
\begin{enumerate}
  \item{}Алфавит: $a$, $b$, $c$.
  \item{}Всякое непустое выражение является формулой.
  \item{}Аксиомы: $aab$, $c$.
  \item{}Правила вывода:
  \begin{multicols}{2}
    \begin{enumerate}
      \item[($\ruleR$)]{}$A'$ выводима из $A$,
      если $A'$ можно получить из $A$, убрав один символ.
      \columnbreak
      \item[($\ruleC$)]{}$AB$ выводима из $A,B$,
      если у формул $A$ и $B$ нет общих символов.
    \end{enumerate}
  \end{multicols}
\end{enumerate}

{\it Теорема:} $abc$

{\it Доказательство:}
$\vdash ab$ по правилу $\ruleR$ и аксиоме $aab$.
$ab\vdash abc$ по правилу $\ruleC$ и аксиоме $c$.
Тогда $\vdash abc$, поскольку в определении выводимости можем
взять $A_1\equiv ab$, а $T\equiv abc$. \qed\footnote{
  $\square$ означает ``что и требовалось доказать''.}

Более понятно доказательство можно
сформулировать с помощью рассуждений в
{\it нормальном виде}\index{нормальный вид}, где доказательство
разбивают на пронумерованные шаги.

В рассуждениях, которые мы будем рассматривать, каждый шаг будет предположением
или выводом формулы. Предположение вводит новую формулу в рассуждение.
Вывод получает новую формулу из предыдущих.

Пусть формулы $C_1,...,C_{k-1},T$ были в этой последовательности введены выводом,
а формулы $\Gamma$ были введены предположением. Тогда
\begin{equation}\label{eq:nonel_vdash}
  \Gamma\vdash C_1\qquad \Gamma,C_1\vdash C_2\qquad ...\qquad
  \Gamma,C_1,...,C_{k}\vdash T
\end{equation}
и, по определению, $\Gamma\vdash T$.

Сформулируем доказательство $abc$ в нормальном виде:
% \begin{enumerate}[label=(\arabic*)]
%   \item{}\label{normex_1}Пусть (предположим) $aab$.
%   \item{}\label{normex_3}$ab$ по $\ruleR$, \ref{normex_1}.
%   \item{}\label{normex_2}Пусть $c$.
%   \item{}\label{normex_4}$abc$ по $\ruleC$, \ref{normex_2}, \ref{normex_3}.\qed
% \end{enumerate}
% Часто предположения аксиом опускают:
\begin{enumerate}[label=(\arabic*)]
  \item{}\label{normexshort_1}$ab$ по $\ruleR$ и аксиоме $aab$.
  \item{}\label{normexshort_2}$abc$ по $\ruleC$,
  \ref{normexshort_1} и аксиоме $c$.\qed
\end{enumerate}

\textsc{Формальная система является просто набором правил и её формулы
  могут не иметь смысл. Они и не будут иметь смысл, пока мы не начнём их
  интерпретировать, то есть придавать им этот смысл.}

Математику можно рассматривать как своеобразную ``игру'' с формулами,
правила которой определяются формальной системой.
Интерпретация её результатов, придание смысла формулам являются
отдельными от самой игры процессами.

\index{modes ponens}
Конечно, определяя правила игры, мы будем вдохновляться логическими рассуждениями
человеческого языка.
Например, из ``Если $A$, то $B$'' и ``$A$'' можем вывести ``$B$'', поэтому обычно
вводят правило ``Формула $B$ выводима из $A$ и $A\implies B$'',
которое называют {\it modes ponens}\footnote{лат. правило вывода}.

\index{истинность}\index{формула!истинная}
Определим формулу
как {\it истинную}, если она является аксиомой или теоремой.
Поэтому аксиомами также называют утверждения, принимаемые как истинные
без требования доказательства.

\vspace{1em}
{\it Упражнения:}
\begin{enumerate}
  \item{}Показать, что если $A\vdash B$ и $B\vdash C$, то $A\vdash C$.
  \item{}Показать, что из \eqref{eq:nonel_vdash} следует $\Gamma\vdash T$,
  несмотря на неэлементарные выводимости (см. определение выводимости).
\end{enumerate}

\section{Правила вывода}

Начнём описывать формальную систему, в которой мы будем работать.
Оставим тот же самый алфавит, понятия терма, переменной и формулы.
\begin{fullwidth}
  \begin{multicols}{2}
    \begin{enumerate}
      \item{}Любой знак переменной является термом.
      \item{}Если $f$ --- функциональный знак арности $n$, и $t_1,...,t_{n}$ --- термы,
      то $f(t_1,...,t_{n})$ --- терм.
      \item{}Если $\phi$ --- предикатный знак арности $n$, и $t_1,...,t_{n}$ --- термы,
      то $\phi(t_1,...,t_{n})$ --- формула.

      \columnbreak

      \item{}Пусть $F_1,F_2$ --- формулы, $F$ --- формула о $x$, тогда выражения
      \[
        \lnot F_1\qquad F_1\land F_2\qquad F_1\lor F_2\qquad
        F_1\implies F_2
      \]
      \[
        F_1\iff F_2\qquad (\forall x)~F\qquad
        \exists x:F
      \]
      являются формулами.
    \end{enumerate}
  \end{multicols}
\end{fullwidth}

\newcommand\taut{$\mathcal T$}
\newcommand\axiom{$\mathcal A$}
\newcommand\implic{$\mathcal I$}
\newcommand\Px{\mathcal P}

Введём единственное правило вывода --- modus ponens:
\[
  \text{$B$ выводима из $A$ и $A\implies B$,}
\]
где $A,B$ --- формулы. Будем его обозначать как MP.

\index{схема аксиом}\index{аксиома!схема}
Зададим совокупность аксиом правилами,
которые определяют, является ли формула аксиомой.
Такие правила называют {\it схемами аксиом}.
Введём схему: $T(\Sx/\Px)$ является аксиомой,
если выражение $T$ --- простая тавтология,
$\Px$ --- совокупность всех знаков переменных $T$,
$\Sx$ --- совокупность произвольных формул,
причём каждому знаку из $\Px$
поставлена в соотвествие единственная формула из $\Sx$.
По этому соответствию и идёт замена знаков формулами в выражениях.
Таким образом, $\top$\footnote{Константы $\top$ и $\bot$ вводят как
  \mbox{$0$-арные} предикатные знаки.}, $[x=0\implies x=0]$ --- аксиомы.

\index{теорема!дедукции}
Для данной системы справедливо,
что из $\Gamma,A\vdash B$ следует ${\Gamma\vdash A\implies B}$.
То есть из существования доказательства первого следует
существование доказательства второго.
Этот результат называют {\it теоремой дедукции} и он сильно упрощает
доказательства импликаций, однако его доказательство
слишком объёмно для данной книги.

{\it Теорема:} $A,B\vdash A\land B$

{\it Доказательство:}
\begin{enumerate}[label=(\arabic*)]
  \item{}Пусть $A$.
  \item{}$B\implies A\land B$ по\footnote{Мы не указываем, по
    какому правилу сделан вывод, поскольку оно единственно.}
  (1) и аксиоме $A\implies (B\implies A\land B)$.
  \item{}Пусть $B$.
  \item{}$A\land B$ по (2) и (3).\qed
\end{enumerate}

Аналогично из существования аксиомы вида
\[
  A_1\implies (A_2\implies (...\implies (A_{n}\implies B)))
\]
следует $A_1,A_2,...,A_{n}\vdash B$.

\vspace{1em}
{\it Упражнения:}
\begin{enumerate}
  \item{}Пусть $A_1\land...\land A_{n}\implies B$ --- аксиома.
  Доказать $A_1,...,A_{n}\vdash B$.
  \item{}Доказать
  % \begin{fullwidth}
  %   \begin{multicols}{2}
  \begin{enumerate}
    \item{}$A\lor B,[A\implies C],[B\implies C]\vdash C$
    \item{}$[A\implies B],[B\implies C]\vdash A\implies C$
    \item{}$[A\implies B],\lnot B\vdash\lnot A$
  \end{enumerate}
  %   \end{multicols}
  % \end{fullwidth}
\end{enumerate}

\section{Кванторы}

\newcommand\At{$\forall$i}
\newcommand\Ai{$\forall$I}
\newcommand\Ae{$\forall$E}
\newcommand\Et{$\exists$i}
\newcommand\Ei{$\exists$I}
\newcommand\Ee{$\exists$E}
Введём схемы аксиом, определяющие поведение кванторов\footnote{
  Буквы в названиях аксиом означают следующее:
  \begin{itemize}
    \item{}i --- implication, следствие.
    \item{}I --- introduction, введение.
    \item{}E --- elimination, исключение.
  \end{itemize}
}.
\begin{enumerate}[label=Q\arabic*]
  \item[($\forall$A)]{}$(\forall x)~P(x)$, если $P(x)$ --- аксиома.
  \item[(\At{})]{}$[(\forall x)~P(x)\implies Q(x)]\implies
    [((\forall x)~P(x))\implies ((\forall x)~Q(x))]$
  \item[(\Ae{})]{}$[(\forall x)~P(x)]\implies P(t)$, где $t$ --- терм.
  \item[(\Ai{})]{}$F\implies [(\forall x)~F]$, где $x$ не содержится в $F$.
  \item[(\Et{})]{}$[(\forall x)~P(x)\implies Q(x)]\implies
    [(\exists x:P(x))\implies (\exists x:Q(x))]$
  \item[(\Ee{})]{}$[\exists x:F]\implies F$, где $x$ не содержится в $F$.
  \item[(\Ei{})]{}$P\implies [\exists x:P(x/'t)]$, где $t$ --- терм.
\end{enumerate}
Заметим схожесть между аксиомами \At{} и \Et{},
\Ae{} и \Ei{}, \Ai{} и \Ee{}.

Аналогично с теоремой дедукции справедлива и {\it теорема обобщения}:
Если $\Gamma\vdash P(x)$, где $x$ не свободна ни в одной из формул
совокупности $\Gamma$, то $\Gamma\vdash (\forall x)~P(x)$.
\index{теорема!обобщения}

Докажем, что знак связанной переменной не имеет значения.

{\it Теорема:}
$[(\forall x)~P(x)]\implies [(\forall t)~P(t)]$

{\it Доказательство:}
\begin{enumerate}[label=(\arabic*)]
  \item{}Пусть $(\forall x)~P(x)$.
  \item{}Тогда $P(t)$ по (1) и \Ae{}.
\end{enumerate}
Значит $(\forall x)~P(x)\vdash P(t)$ и $(\forall x)~P(x)\vdash (\forall t)~P(t)$
по теореме обобщения. Тогда ${\vdash [(\forall x)~P(x)]\implies [(\forall t)~P(t)]}$
по теореме дедукции.\qed

{\it Теорема:}
$[\exists x:P(x)]\implies [\exists t:P(t)]$

{\it Доказательство:}
${P(x)\vdash\exists t:P(t)}$ по аксиоме \Ei{}, значит
по теоремам дедукции и обобщения
\[
  \vdash (\forall x)~[P(x)\implies \exists t:P(t)]
\]
Тогда
\[
  \vdash [\exists x:P(x)]\implies [\exists x,t:P(t)]
\]
по аксиоме \Et{}.
\begin{enumerate}[label=(\arabic*)]
  \item{}Пусть $\exists x:P(x)$.
  \item{}$\exists x:\exists t:P(t)$ по (1) и доказанному.
  \item{}$\exists t:P(t)$ по (2) и \Ee{}.
\end{enumerate}
Значит $[\exists x:P(x)]\implies[\exists t:P(t)]$
по теореме дедукции.\qed

При применении modus ponens на теоремы можно ссылаться точно так же,
как и на аксиомы. Так, если $A\implies B$ --- теорема, то
справедливы рассуждения
\begin{enumerate}[label=(\arabic*)]
  \item{}Пусть $A$.
  \item{}$B$ по (1) и теореме $A\implies B$.
\end{enumerate}

Законы отрицания кванторов можно разделить на четыре теоремы.
Докажем одну из них.

% \vspace{1em}
{\it Теорема:} $\lnot [\exists x:P(x)]\implies  (\forall x)~\lnot P(x)$

{\it Доказательство:}
\begin{enumerate}[label=(\arabic*)]
  \item{}\label{qneg_1}Пусть $\lnot[\exists x:P(x)]$.
  \item{}\label{qneg_2}Пусть $P(x)$.
  \item{}\label{qneg_3}$\exists x:P(x)$ по \ref{qneg_2} и \Ei{}.
  \item{}\label{qneg_4}$\bot$ по \ref{qneg_1}, \ref{qneg_3} и
  тавтологии $p\implies (\lnot p\implies \bot)$\footnote{Здесь мы опустили один шаг.
    Данная строка получена через применение MP два раза.}.
\end{enumerate}
По теореме дедукции имеем $\lnot[\exists x:P(x)]\implies (P(x)\implies\bot)$.
\begin{enumerate}[label=(\arabic*)]
  \item{}\label{qneg_5}Пусть $\lnot[\exists x:P(x)]$.
  \item{}\label{qneg_6}$P(x)\implies\bot$ по \ref{qneg_5} и доказанному.
  \item{}\label{qneg_7}$\lnot P(x)$ по \ref{qneg_6} и
  тавтологии $(p\implies \bot)\implies\lnot p$.
\end{enumerate}
Можем завершить доказательство применением теорем обобщения и дедукции.\qed

\vspace{1em}
{\it Упражнения:}
\begin{enumerate}
  \item{}Обосновать аксиомы кванторов.
  \item{}Доказать теоремы:
  \begin{enumerate}
    \item{}$[(\forall x)~P(x)\implies F]\land [\exists x:P(x)]\implies F$,
    где $F$ не содержит $x$.
    \item{}$[(\forall x)~P(x)]\land [\exists x:P(x)\implies Q(x)]
      \implies [\exists x:Q(x)]$
    \item{}$(\forall x)~T(F/p)$, где $T$ --- простая тавтология,
    $p$ --- её единственный символ переменной, $F$ --- формула о $x$.
    \item{}$(\forall x)~T(\Sx/\Px)$, где $T$ --- простая тавтология,
    $\Px$ --- символы её переменных, $\Sx$ --- формулы о $x$.
  \end{enumerate}
\end{enumerate}

\section{Цепочки импликаций}

\index{цепочка!импиликаций}
Доказательства в нормальном виде на практике редко используют из-за их
громоздкости.
Чаще доказательства в математике записывают в виде {\it цепочек импликаций}
с рассуждениями. Пусть мы уже вывели совокупность формул $\Gamma$~и
\[
  \Gamma,A_1\vdash A_2\qquad \Gamma,A_2\vdash A_3
  \qquad  ... \qquad \Gamma,A_{n-1}\vdash A_{n},
\]
тогда из $\Gamma$ можем вывести импликации
${A_{k}\implies A_{k+1}}$, которые кратко можно записать в виде цепочки
\[
  A_1\implies A_2\implies ...\implies A_{n}
\]
По тавтологии
\[
  (p\implies q)\land (q\implies r)\implies (p\implies r)
\]
построение такой цепочки доказывает $A_1\implies A_{n}$ из $\Gamma$.

Цепочки импликаций также можно назвать ``человеческим видом'',
поскольку они более понятны при чтении.

В человеческом виде часто используют следующие обороты:
\begin{multicols}{2}
  \begin{enumerate}[label=(\roman*)]
    \item{}
    Имеем $A$.\\
    $A\implies...\implies B$\\
    Тогда $B$.
    \item{}
    Пусть $A$.\\
    ...\\
    Тогда $B$.\\
    Тогда $A\implies B$.
    \item{}
    Имеем $A$.\\
    $B\implies ...\implies \bot$\\
    Тогда $\lnot B$.
    \item{}
    Имеем $A$. Пусть $B$.\\
    ...\\
    Тогда $\bot$.\\
    Тогда $\lnot B$.
  \end{enumerate}
\end{multicols}

\vspace{1em}
{\it Теорема:} $\lnot[\exists x:P(x)]\implies [(\forall x)~\lnot P(x)]$

{\it Доказательство:}
Пусть ${\lnot[\exists x:P(x)]}$.
Возьмём произвольный\footnote{
  Используя такие слова, мы намекаем на смысл, который придаём доказательству
  формулы с $\forall x$.

  ``Произвольным'' является объект, обозначаемый термом $x$.
  Терм берётся самый конкретный --- $x$.}~$x$.
\[
  P(x)\xRightarrow{\text{\Ei{}}} [\exists x:P(x)]
  \xRightarrow{p\land\lnot p\implies \bot} \bot
\]
Тогда $\lnot P(x)$ и $(\forall x)~\lnot P(x)$
по теореме обобщения.\qed

\vspace{1em}
{\it Теорема:} $\lnot[(\forall x)~P(x)]\implies [\exists x:\lnot P(x)]$

{\it Доказательство:}
Пусть $\lnot[(\forall x)~P(x)]$.
\[
  \lnot[\exists x:\lnot P(x)]\implies [(\forall x)~\lnot\lnot P(x)]
  \implies [(\forall x)~P(x)]\implies\bot
\]
Тогда $\exists x:\lnot P(x)$.\qed

\pagebreak
% \vspace{1em}
{\it Теорема:} $[(\forall x)~\lnot P(x)]\implies \lnot[\exists x:P(x)]$

{\it Доказательство:}
Пусть $(\forall x)~\lnot P(x)$. Возьмём произвольный $x$
и предположим $P(x)$. Имеем $\lnot P(x)$ по \Ae{} и $\bot$ по
$p\land\lnot p\implies\bot$.
Можем обобщить до $(\forall x)~P(x)\implies \bot$.
Тогда
\[
  [\exists x:P(x)]\xRightarrow{\text{\Et{}}} [\exists x:\bot]
  \xRightarrow{\text{\Ee{}}} \bot
\]
и $\lnot[\exists x:P(x)]$.\qed

\vspace{1em}
{\it Теорема:} $[\exists x:\lnot P(x)]\implies\lnot[(\forall x)~P(x)]$

{\it Доказательство:}
Пусть $\exists x:\lnot P(x)$. Пусть $\lnot P(x)$ для произвольного $x$.
\[
  [(\forall x)~P(x)]\implies P(x)\implies \bot
\]
Тогда $\lnot[(\forall x)~P(x)]$ по $\exists$E.\qed

\vspace{1em}
\index{цепочка!эквивалентностей}
Аналогично определяют и {\it цепочки эквивалентностей}
\[
  A_1\iff A_2\iff ...\iff A_{n},
\]
где $\Gamma,A_{k}\vdash A_{k+1}$ и $\Gamma,A_{k+1}\vdash A_{k}$.

\vspace{1em}
{\it Упражнения:}
\begin{enumerate}
  \item{}Обосновать каждую импликацию в доказательствах законов отрицания кванторов.
  \item{}*Доказать с помощью рассуждений в нормальном виду законы отрицания кванторов.
\end{enumerate}

\section{Равенство}

Определяющее свойство равенства в том, что если два терма равны, то их можно
заменять друг другом в формулах.

\index{равенство}
Введём бинарный предикатный знак $=$\footnote{
  Тогда, если $a,b$	 --- термы, то выражение $a=b$ --- формула.}
и аксиомы
\begin{enumerate}
  \item[($=$I)]{}$(\forall x)~x=x$
  \item[($=$E)]{}$(\forall x,y)~x=y\implies (P\implies P(y/'x))$
\end{enumerate}

\vspace{1em}
{\it Теорема:} Пусть $t$ --- терм, тогда
\[
  (\forall a,b)~a=b\implies f(a)=f(b)
\]

{\it Доказательство:}
Возьмём произвольные $a$ и $b$. Пусть $a=b$.
По аксиоме $=$I имеем $f(a)=f(a)$, по $=$E имеем $f(a)=f(b)$.
То есть
\[
  a=b\implies f(a)=f(b)
\]
Можем завершить доказательство обобщением.\qed

% \pagebreak

Равенство позволяет нам ввести понятие единственности.
Как и ранее, определим сокращение квантора единственности.

{\it Теорема:} $\forall a~\exists !b:a=b$

{\it Доказательство:}
Докажем существование.
Возьмём произвольный $a$. Имеем $a=a$ по $=$I,
тогда $\exists b:a=b$.

Докажем единственность. Пусть для произвольных $b_1,b_2$
справедливы $a=b_1$ и $a=b_2$. По $=$E имеем
\[
  a=b_1\implies (a=b_2\implies b_1=b_2)
\]
Тогда $b_1=b_2$.

Таким образом,
\[
  [\exists b:a=b]\land[(\forall b_1,b_2)~a=b_1\land a=b_2\implies b_1=b_2],
\]
то есть $\exists !b:a=b$, можем использовать теорему обобщения.\qed

\vspace{1em}
{\it Упражнения:}
\begin{enumerate}
  \item{}Доказать теоремы
  \begin{enumerate}
    \item{}$(\forall x,y)~x=y\iff y=x$
    \item{}$(\forall x,y,z)~x=y\land y=z\implies x=z$
  \end{enumerate}
\end{enumerate}

\section{Введение новых знаков}

Пусть $\varphi(t_1,...,t_{n})$\footnote{
Если $F$ --- формула о $x_1,...,x_{n}$, то
\[
  F(x_1,...,x_{n})\equiv F
\]
\[F(y_1,...,y_{n})\equiv F(y_1/x_1)...(y_{n}/x_{n})\]} --- формула
только о $t_1,...,t_{n}$. Можем ввести новый $n$-арный предикатный знак $\eta$
и аксиому, его определяющую:
\[
  (\forall t_1,t_2,...,t_{n})~\varphi(t_1,...,t_{n})\iff\eta(t_1,...,t_{n})
\]
Часто  кванторы, относящиеся к аргументам, опускают
при определении новых предикатных знаков.

Например, пусть $<$ --- бинарный предикатный знак.
Можем ввести новые предикатные знаки $\leq$ и $>$.
\[
  x\leq y\iff x<y\lor x=y\qquad x>y\iff y< x
\]
Для каждого бинарного предикатного знака $\prec$ обычно
вводят его отрицание --- знак $\nprec$.
\[
  x\nprec y\iff \lnot(x\prec y)
\]

\index{область!определения}
Функция --- правило, по которому объекту
из определённой совокупности ({\it области определения})
сопоставляется единственный объект.
Пусть $P(x_1,...,x_{n}),\varphi(x_1,...,x_{n},y)$ --- формулы только
о переменных в скобках.
Пусть формула $P(x_1,...,x_{n})$ определяет условие, при котором функция определена,
а $\varphi(x_1,...,x_{n},y)$ определяет значение функции.
Если доказуемо
\[
  (\forall x_1,...,x_{n})~P(x_1,...,x_{n})
  \implies \exists! y:\varphi(x_1,...,x_{n},y),
\]
то можем расширить формальную систему новым $n$-арным фнукциональным знаком $f$
и аксиомой
\[
  (\forall x_1,...,x_{n})~P(x_1,...,x_{n})
  \implies \varphi(x_1,...,x_{n},f(x_1,...,x_{n}))
\]

Пусть $T,T'$ --- формальная система до и после таких расширений.
Если формула $\psi$ доказуема в $T'$ и является формулой в $T$, то она доказуема в $T$.
Из этого следует, что внутри доказательства можно расширить формальную систему
введением новых функций, не влияя на справедливость доказательства.

Говорят, что $f$ {\it определена}\index{определённость функции}
для всех $x_1,...,x_{n}$ таких, что $P(x_1,...,x_{n})$.
Если $P(x_1,...,x_{n})\equiv \top$, то формулы выше упрощают до
\[
  \forall x_1,...,x_{n}~\exists !y:\varphi(x_1,...,x_{n},y)\qquad
  (\forall x_1,...,x_{n})~\varphi(x_1,...,x_{n},f(x_1,...,x_{n}))
\]

Рассмотрим примеры.
Пусть $+$ --- бинарный функциональный знак.
Возьмём формулу ${y=(x+x)+a}$. Очевидно,
\[
  \forall x,a~\exists !y:y=(x+x)+a
\]
Можем ввести бинарный функциональный знак $g$ и аксиому
\[
  (\forall x,a)~g(x,a)=(x+x)+a
\]
Обычно некоторые аргументы опускают. Например, $g(x)\equiv g(x,a)$.

\index{равно по определению, $:=$}
Часто рассуждения выше сокращают обозначением
\[
  g(x):=(x+a)+a,
\]
где ``$:=$'' читают как ``равно по определению''.
Его часто используют при введении новых знаков и переменных.

Пусть $\cdot$ --- бинарный функциональный знак,
$0$ и $1$ --- константы и справедливо
\[
  (\forall a\neq 0)~\exists !b:a\cdot b=1
\]
Можем ввести унарный функциональный знак $f$ и аксиому
\[
  (\forall a\neq 0)~a\cdot f(a)=1
\]
Обычно такую функцию обозначают как $(\cdot)^{-1}$,
то есть $a^{-1}\equiv f(a)$.
Она определена для всех $x\neq 0$.

Часто рассуждения выше сокращают до
\[
  (\forall a\neq 0)~\exists !a^{-1}:a\cdot a^{-1}=1
\]
