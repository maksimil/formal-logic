\part{Заключение}

\section{Истина в математике}

Истинность любого утверждения в математике исходит только из определения правил
вывода и определения аксиом как истинных утверждений.
То есть $1+1=2$ --- истина по определению формальной системы\footnote{
  Поэтому истинные утверждения в математике также называют тавтологиями.}.
С помощью натуральных чисел мы формализуем понятие ``если взять одно яблоко
и ещё одно яблоко, то будет два яблока''.

Важно понимать связь между понятием истины в формальных системах
и эмпиричной истины\footnote{{\it Эмпиризм}\index{эмпиризм}
  --- метод познания через ощущение (наблюдение).

  Если вы видите, что яблоко упало, то ``ялоко упало'' --- эмпирическая истина.
  Обобщение этого наблюдения на все яблоки (``яблоки падают'') уже не является
  эмпирической истиной.}.
Физики подбирают аксиомы (постулаты),
из которых выводимы теоремы (законы), соответствующие наблюдениям.
Они подгоняют истину формальной системы к эмпиричной истине.
Они формализуют наблюдения в постулатах, чтобы использовать достаточно мощные
инструменты матанализа для предсказания движения тел.

\section{В защиту интуиции}

Ценность формальных доказательств в их точности.
Каждый шаг полностью обоснован и,
поэтому, хорошее формальное доказательство неоспоримо.

Но нельзя забывать и о ценности интуитивного понимания формул.
Оно легче укладывается в голове\footnote{Например, диаграммы
  Эйлера-Венна помогают понять и запомнить операции и теоремы, связанные с
  множествами, но их нельзя приводить как доказательства.},
а интуитивное доказательство может натолкнуть на формальное.
Поэтому я привожу пример некоторых словесных интерпретаций
формул в таблице~\ref{table:formula_interp}.
Также я привожу определения основных множеств в таблице~\ref{table:set_def}.

Не приступайте к задачам и теоремам, не поняв определения понятий,
которые в них упоминаются. Если видите на лекции понятие,
которое не было определено ранее,
попробуйте поискать его в этой книге\footnote{
  За исключением, разве что, арифметических операций на числах.
  Предполагается, что они были введены в школе.
} (предметный указатель в помощь) или спросите у преподавателя.
Слушайте лекции, читайте книги, работайте
на практиках. Удачи.

\vspace{1em}
{\it Упражнения:}
\begin{enumerate}
  \item{}Обосновать интерпретации в таблице~\ref{table:formula_interp}.
  \item{}Дать определение всем терминам из предметного указателя.
  \item{}Взять лист бумаги и записать все правила и определения, составляющие
  формальную систему, в которой мы работали, то есть построить её с нуля.
\end{enumerate}

\begin{table}
  \centering
  \begin{tabular}{r|l}
    $\exists k:(\forall n>k)~P(n)$        & $P(n)$ начиная с какого-то $n$.      \\
                                          & $P(n)$ для достаточно больших $n$.   \\[1em]

    $\exists \delta:\forall x
    \big(|x|<\delta\implies P(x)\big)$    & $P(x)$ для достаточно малых $x$.     \\[1em]
		
    $\exists \delta:\forall x
    \big(|x-a|<\delta \implies P(x)\big)$ & $P(x)$ для $x$
    достаточно близких к $a$.                                                    \\[1em]

    $\forall \varepsilon~
    \exists \delta:P(\varepsilon,\delta)$ & Для всякого $\varepsilon$ можно      \\
                                          & подобрать такой $\delta$,
    что $P(\varepsilon,\delta)$.                                                 \\[1em]

    $A\cap B=\eset$                       & Множества $A$ и $B$ не пересекаются.
  \end{tabular}
  \caption{Интерпретации формул}\label{table:formula_interp}
\end{table}

\vspace{1em}

\begin{table}
  \centering
  \begin{tabular}{r|l}
    $S$                        & Условие $S$                        \\\hline
    $A\cap B$                  & $x\in A\land x\in B$               \\
    $A\cup B$                  & $x\in A\lor x\in B$                \\
    $A\setminus B$             & $x\in A \land x\notin B$           \\
    $\cup U$                   & $\exists u\in U:x\in u$            \\
    $\cap U$                   & $(\forall u\in U)~x\in u$          \\
    $2^{S}$                    & $x\subseteq S$                     \\
    $\{a\in A\;\big|\; P(a)\}$ & $x\in A\land P(x)$                 \\
    $f(A)$                     & $\exists a\in A:f(a)=x$            \\
    $f^{-1}(A)$                & $f(x)\in A$                        \\
    % $A\circ B$                 & $\exists a\in A,b\in B:x=a\circ b$ \\
    $\eset$                    & $\bot$                             \\
    $\{a\}$                    & $x=a$                              \\
    $\{a_1,...,a_{n}\}$        & $x=a_1\lor ...\lor x=a_{n}$        \\
    $\{a_{i}\}_{i\in I}$       & $\exists i\in I:x=a_{i}$           \\
    $A\times B$                & $\exists a\in A,b\in B:x=(a,b)$    \\
    $[x_0]$                    & $xRx_0$
  \end{tabular}

  \[
    (x,y)\in A\times B\iff x\in A\land y\in B
  \]

  \caption{Основные определения множеств.}\label{table:set_def}
\end{table}
