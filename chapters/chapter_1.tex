\part{Предисловие}

Между школьной и университетской математиками существует категорическое
различие: в школе математика рассматривается как определённый ``закон природы'',
то есть $1+1=2$ потому что это так. В университете же математика изучается
в рамках формальной системы, которая не упоминается в рамках курса.

\textsc{Университетская математика вводит аксиомы, из которых выводит теоремы,
	в отличие от школьной математики, которая
	в основном строится на интуитивных рассуждениях.}

Данная книга стремится доступно и точно сформулировать понятия аксиомы,
теоремы, доказательства и дать необходимый фундамент для построения
интуитивного понимания математических формул.

В тексте достаточно много сносок, они содержат важные оговорки,
улучшающие понимание материала. В качестве практики
после некоторых глав предложены упражнения. Упражнения с повышенной сложностью
отмечены~*.
По всем вопросам обращайтесь по адресу {\sl kksenya758@gmail.com}.

\part{Основные понятия}

\section{Формулы и выражения}

\textsc{Центральным понятием в математике является формула.}
Определим это понятие.

Определим {\it алфавит} --- совокупность знаков\footnote{
	Используется слово ``знак'', потому что знак ``$:=$'' состоит из двух символов,
	но рассматривается как один.
}, на котором эти
формулы будут записаны. Знаки можно разделить на три части:
\begin{enumerate}
	\item{}{\it Символов переменных и констант} --- букв латинского, греческого алфавита,
	иврита, арабских цифр и так далее.
	\item{}{\it Союзов} --- знаков логических операций, операций сравнения.
	\item{}{\it Знаков препинания} --- скобок, запятых, двоеточий и прочих\footnote{
		Данная книга не будет углубляться в детали расставления скобок и будет ставить
		их по необходимости и опускать по возможности.
	}.
\end{enumerate}

{\it Выражение} --- любая последовательность знаков алфавита.
Выражение не обязательно имеет смысл.

Возьмём знаки $\top$ и $\bot$ (знаки тавтологии и противоречия),
будем говорить, что выражение $\top$ {\it имеет (принимает) значение} ``истина'',
а $\bot$ --- ``ложь''. Выражения из одного знака, всегда принимающие одно значение,
называются {\it константами}. Если выражение принимает значение константы $\xi$,
то будем говорить, что оно принимает значение $\xi$.

Будем говорить, что {\it род выражения} --- формула\footnote{
	Или выражение имеет род формулы, или выражение принадлежит к роду формулы и так далее.
}, если оно принимает значение $\top$
или $\bot$. Тогда можем определить {\it формулу (утверждение)} как выражение,
имеющее род формулы.

Позже будут введены и другие роды выражений. В общем случае, род определяет
совокупность значений, которые принимает выражение.
Одно выражение может принадлежать сразу нескольким родам.

Выражение из одного знака, имеющее род и не являющееся константой,
называется {\it переменной}.

Введём в алфавит союзы $\land$, $\lor$, $\lnot$, $\implies$, $\iff$, $=$
и сформулируем правила определения значения сложных выражений.
\begin{enumerate}
	\item{}Равенство ($=$).

	Если выражения $p$ и $q$ принимают одно значение,
	то выражение $p=q$\footnote[][-2cm]{
		Если буквой $A$ мы обозначим выражение $a\land b$,
		а буквой $B$ --- $\lnot b$, то $A=B$ будет обозначать выражение
		${(a\land b)=\lnot b}$.

		Скобки в данном случае помогают определить порядок
		применения правил определения рода:
		\begin{enumerate}
			\item{}${(a\land b)=\lnot b}$ --- формула,
			если $a\land b$ и $\lnot b$ --- формулы.
			\item{}$a\land b$ и $\lnot b$ --- формулы, если переменные $a$ и $b$ --- формулы.
		\end{enumerate}
	} принимает значение $\top$, иначе --- $\bot$.

	Обычно слова ``выражение (формула) $A$ принимает значение $\top$''
	сокращают до ``$A$''.

	\textsc{Основное свойство равенства:} Если выражения равны, то их можно
	заменять друг другом в выражениях, чтобы получить равные выражения.

	Например, если $a=(b\land c)$, то $(a\land\top)=[(b\land c)\land\top]$\footnote{
		Квадратные скобки означают то же самое,
		что и круглые. Они обычно используются, чтобы не запутаться в круглых скобках.
	}.

	\item{}И ($\land$). $p\land q$ тогда и только тогда, когда $p$ и $q$.

	То есть $p\land q$, если $p$ и $q$,
	и $p\land q$ ложно\footnote{Принимает значение $\bot$.} иначе.

	\item{}Или ($\lor$). $p\lor q$ ттк\footnote{тогда и только тогда, когда} $p$ или $q$.

	\item{}Отрицание ($\lnot$). $\lnot p$ ттк $p$ ложно.

	\item{}Следствие, импиликация ($\implies$). $(p\implies q):=(\lnot p)\lor q$.

	Знак ``$:=$'' означает ``равно по определению'', его можно заменить
	на знак ``$=$''. Он используется некоторыми авторами при введении новых
	обозначений\footnote{
		Отрицание можно определить так:
		\[
			\lnot\top:=\bot\qquad \lnot\bot:=\top
		\]
	}.

	\item{}Равносильность, эквивалентность ($\iff$).

	$(p\iff q):=(p\implies q)\land (q\implies p)$.
\end{enumerate}

Из этих правил можно вывести следующие правила определения рода выражений:
\begin{enumerate}
	\item{}Если выражения $A$ и $B$ имеют род,
	то выражение $A=B$ имеет род формулы.

	\item{}Если выражения $A$ и $B$ --- формулы, то формулами являются выражения
	\[
		A\land B\qquad A\lor B\qquad \lnot A\qquad A\implies B\qquad A\iff B
	\]
\end{enumerate}

Заметим, что если выражение имеет род, а все его переменные
имеют определённые значения, то это выражение имеет определённое значение.

\textsc{Смысл определённых значений в математике проявляется в том,
	как эти значения взаимодействуют
	друг с другом в формулах.}
Смысл выражения $\top$ не в том, что мы называем его истинным,
а в том, что, например, $\top\land a=a$.

% Обычно род переменных в выражении не указывается, но подразумевается.
% Роды переменных выражения $A$ называются {\it допустимыми}, если при них $A$ имеет род
% определённый род (является формулой, числом и тд).
% Например в формуле $\top\land a$, допустимый род $a$ --- формула.

Чтобы лучше понимать формулы нужно уметь их переводить на человеческий язык. Делается
это последовательной заменой союзов их аналогами в таблице~\ref{table:read_form}.
\begin{margintable}
	\begin{tabular}{cl}
		$p\land q$    & $p$ и $q$                 \\\hline
		$p\lor q$     & $p$ или $q$               \\\hline
		$\lnot p$     & не $p$                    \\\hline
		$p\implies q$ & если $p$, то $q$          \\
		              & из $p$ следует $q$        \\
		              & для $q$ достаточно $p$    \\
		              & для $p$ необходимо $q$    \\
		              & $q$ всякий раз, когда $p$ \\\hline
		$p\iff q$     & $p$ эквивалентно $q$      \\
		              & $p$ равносильно $q$       \\
		              & $p$ ттк $q$
	\end{tabular}
	\caption{Аналоги формул}\label{table:read_form}
\end{margintable}

Например,
\begin{enumerate}
	\item{}$(A\land (A\implies B))\implies B$
	\item{}Из $A\land (A\implies B)$ следует $B$
	\item{}Из $A$ и $(A\implies B)$ следует $B$
	\item{}Из $A$ и ``для $B$ достаточно $A$'' следует $B$
\end{enumerate}

Выражение из символов переменных и знаков $\top$, $\bot$, $\land$, $\lor$, $\lnot$,
$\implies$ и $\iff$ называется {\it простой тавтологией}, если
оно принимает значение $\top$, когда символы переменных являются формулами\footnote{
	Символ переменной рассматривается как выражение из одного знака
}.
Заметим, что проверить, является ли выражение простой тавтологией,
можно за конечное количество шагов: достаточно просто перебрать все возможные значения
переменных.

{\it Упражнения:}

\begin{enumerate}
	\item{}Показать, что следующие выражения --- простые тавтологии\label{ex:simple_taut}
	\begin{enumerate}
		\item{}$\top$\footnote{Знак $\top$ по этой причине называется
			знаком элементарной тавтологии.}
		\item{}$p\lor (\lnot p)$ --- закон исключённого третьего
		\item{}$(p\land p)\iff p$
		\item{}$(p\land (p\implies q))\implies q$ --- modus ponens (лат. правило вывода)
		\item{}$\lnot(p\lor q)\iff (\lnot p)\land (\lnot q)$ --- закон Де Моргана
		\item{}$(\lnot(p)\land p)\iff\bot$\footnote{
			Знак $\bot$ по этой причине называется
			знаком элементарного противоречия.}
		\item{}$p\implies (q\implies p)$\footnote{
			\textsc{Математическое следствие не подразумевает причинность.}
			Это будет освящено подробнее позже.
		}
		\item{}${(p\implies q)\iff ((\lnot q)\implies (\lnot p))}$
		\item{}$(\lnot(p)\implies\bot)\implies p$\footnote{На этой тавтологии основаны
			доказательства от противного. То есть если из $\lnot S$ мы можем прийти
			к противоречию, то $S$.}
	\end{enumerate}
	\item{}Почему если $A$ --- тавтология и $(A\implies B)$, то $B$?
	\item{}Обосновать словесные интерпретации формул из таблицы~\ref{table:read_form},
	дополнить таблицу своими примерами.
	\item{}Прочитать выражения из упражнения~\ref{ex:simple_taut}.
\end{enumerate}

\section{Кванторы}

Добавим в математический язык возмножность составлять утверждения о существовании
определённого значения и о всеобщности какого-то свойства для значений
определённого рода.

В этой главе мы введём кванторы интуитивно, в следующей главе
мы введём их более формально.

Введём в алфавит знаки $\forall$ и $\exists$ (квантор всеобщности и
квантор существования). Будем говорить, что переменная $\gamma$
{\it связанна} в формуле, если эта формула содержит выражение $K\gamma$,
где $K$ --- квантор. Если формула $F$ содержит переменную $\gamma$ и $\gamma$
не связанна в $F$, то она {\it свободна} в $F$. Если переменная не связанна в $F$,
то либо $F$ не содержит $\gamma$, либо $\gamma$ свободна в $F$.

Сформулируем правила определения значения выражений с кванторами.
Пусть $P$ --- формула, $x$ --- переменная и $x$ не связанна в $P$, тогда
\begin{enumerate}
	\item{}$(\forall x)P$ ттк $P$ для произвольного значения $x$.
	\item{}$\exists x:P$ ттк существует такое значение переменной $x$, что $P$.
\end{enumerate}

Можем вывести и следующее правило определение рода выражений:
Если $P$ --- формула, $x$ --- переменная и $x$ не связанна в $P$,
то выражения $(\forall x)P$ и ${\exists x:P}$ --- формулы.
Заметим, что в одной формуле можно использовать несколько кванторов.

Заметим также, что род выражения можно определить зная только роды переменных.

% Если ${[\exists x:P(x)]\land[(\forall x)(\forall y)(P(x)
%         \land P(y)\implies x=y)]}$\footnote{
%   Последнее
%   условие гарантирует единственность значения $x$, потому что иначе
%   могут существовать такие $x$ и $y$, что $x\neq y\land P(x)\land P(y)$.
%   В нём использован оператор ``$=$''.
%   Формула $x=y$ принимает значение $\top$, если $x$ и $y$ имеют одно значение,
%   и $\bot$ --- иначе.}, то пишут
% \[
%   \exists! x:P(x)
% \]

% $\exists!$ --- {\it квантор существования и единственности}. Расширим
% понятие формулы: пусть если
% $A$ --- формула, $\gamma$ --- переменная, то выражение ${\exists!\gamma: A}$
% тоже является формулой.

Часто формулы с кванторами сокращают:
\begin{enumerate}
	\item{}Формулу ${(\forall x)(Q\implies R)}$ можно сократить
	до ${(\forall x:Q)R}$. Читается как ``для всякого $x$ такого, что
	$Q$, $R$''.

	\item{}Формулу $(\forall x:x\prec a)P$ можно сократить до $(\forall x\prec a)P$,
	где вместо $\prec$ могут быть знаки $<$, $>$, $\leq$, $\in$ и прочие.

	\item{}Формулу $\exists x:(x\prec a)\land P$ можно сократить
	до $\exists x\prec a:P$.

	\item{}Выражения с одинаковыми кванторами можно объединить:
	выражение $(\forall x)(\forall y)P$ можно записать как $(\forall x,y)P$,
	а ${\exists x:(\exists y:P)}$ --- как $\exists x,y:P$.

	\item{}Скобки вокруг $\forall x$ можно опустить, если после него следует $\exists$ или
	выражение в скобках.
	То есть ${(\forall \varepsilon)\exists \delta:P}$
	можно сократить до $\forall \varepsilon~\exists \delta:P$,
	а $(\forall x)(x=x)$ до $\forall x(x=x)$.

	\item{}Некоторые авторы опускают ``$:$'' после $\exists$ и
	ставят выражение $\exists x$ в скобки: $(\forall a)(\exists b)P$.
\end{enumerate}

Для кванторов выполняются следующие {\it законы отрицания}\footnote{
	На данный момент мы не можем их доказать, а только
	обосновать словесно (см. упражнение~\ref{ex:quantor_neg_def}).
}:
\[
	\lnot[(\forall x)~P]:=[\exists x:\lnot P]
\]
\[
	\lnot[\exists x:P]:=[(\forall x)~\lnot P]
\]

Если переменная $\gamma$ связанна в формуле $F$, то её буква не имеет значения.
Например, следующие две формулы эквивалентны:
\[
	(\forall p)~(p\implies (q\implies p))\qquad
	(\forall \chi)~(\chi\implies (q\implies\chi))
\]

Возьмём выражение $P$, заменим в ней выражения $A$ выражением $B$ (не обязательно все)
и обозначим полученное выражение как $P(A/B)$. Если все выражения $A$ заменяны
на выражение $B$, то обозначим полученное выражение как $P(A/_{a}B)$.

Если формула содержит две формулы, и переменная
связанна в обоих, то эти два использования
одного знака не связанны и даже могут иметь различные роды.
Например, следующие две формулы эквивалентны:
\[
	[(\forall x)P]\land[\exists x:Q]\qquad
	[(\forall \alpha)P(x/_{a}\alpha)]\land[\exists \beta:Q(x/_{a}\beta)],
\]

{\it Упражнения:}
\begin{enumerate}
	\item{}Обосновать на основе словесных рассуждений законы отрицания
	кванторов\label{ex:quantor_neg_def}.
	\item{}Записать отрицание утверждения
	\[
		\forall \varepsilon>0~\exists \delta>0:
		(\forall x\in\mathbb{R})~[|x|<\delta\implies |f(x)|<\varepsilon]
	\]

	Где $f$ --- числовая функция, $\mathbb{R}$ --- множество действительных чисел.
	Подразумевается, что $\varepsilon$ и $\delta$ --- действительные числа.
	${(x\in\mathbb{R})}$ --- предикат, принимающий значение $\top$,
	если $x$ --- действительное чило.

	Подсказка:
	\[
		\lnot[\exists a:(\forall b)P]=(\forall a)\lnot[(\forall b)P]=
		(\forall a)(\exists b:\lnot P)
	\]

	\item{}Показать, что если $P$ --- тавтология, то $(\forall x)P$.

	\item{}В каком случае $(\forall x)P$ означает, что $P$ --- тавтология?

	\item{}*Привести пример выражения без свободных переменных,
	являющегося тавтологией.
\end{enumerate}
