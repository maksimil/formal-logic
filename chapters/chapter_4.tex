\part{Бинарные отношения}

Рассмотрим множества $X,Y$ и $R\subseteq X\times Y$. Множество $R$ можно рассматривать
как определённое отношение между элементами множеств $X$ и $Y$. Можем ввести бинарный
предикатный знак
\[
	\forall x\forall y(xRy\iff (x,y)\in R)
\]

Обычно предикатный знак и множество $R$ обозначают одинаково.
$R$ называют {\it бинарным отношением между} $X$ и $Y$. Если $X=Y$,
то $R$ называют {\it бинарным отношением на} $X$.

Например, рассмотрим множество $S=2^{\{a,b\}}$.
\[
	S=\{\eset,\{a\},\{b\},\{a,b\}\}
\]
Пусть $R\subseteq S^{2}$ --- отношение включения ($\subseteq$) на $S$, тогда
\[
	\begin{aligned}
		R
		&=\left\{(x,y)\in S^{2}\;\big|\; x\subseteq y\right\}=\\
		&=\left\{
			\begin{array}{cccc}
				(\eset,\eset),&(\eset,\{a\}),&(\eset,\{b\}),&(\eset,\{a,b\}),\\
				(\{a\},\{a\}),&(\{a\},\{a,b\}),&(\{b\},\{a,b\}),&(\{a,b\},\{a,b\})
			\end{array}\right\}
	\end{aligned}
\]

\section{Отображения}

Рассмотрим бинарное отношение $f\subseteq X\times Y$ со следующим свойством:
\[
	\forall x\in X~\exists! y\in Y:(x,y)\in f
\]
Можем ввести унарный функциональный знак аксиомой
\[
	(\forall x\in X)~(x,f(x))\in f
\]
Заметим, что из $(x,f(x))\in f$ следует $f(x)\in Y$.

Обычно функциональный знак и бинарное отношение $f$ обозначают одинаково.
Такое $f$ называют {\it отображением} из {\it области определения} $X$ в
{\it область значений} $Y$ и пишут $f:X\to Y$.
Отображения также называют функциями.

Например, если $f:\Z\to\Z$ определено как $f(x)=2x$, то
\[
	f=\left\{(0,0),(1,2),(-1,-2),(2,4),(-2,-4),...\right\}
\]

Для отображений вида $g:X_1\times ...\times X_{n}\to Y$
введём также $n$-арные функциональные знаки аксиомами
\[
	(\forall x_1\in X_1)...(\forall x_{n}\in X_{n})~
	((x_1,...,x_{n}), g(x_1,...,x_{n}))\in g
\]

Возьмём $f:X\to Y$. Если $y=f(x)$, то $y$ называют {\it образом} $x$,
а $x$ --- {\it прообразом} $y$. {\it Образом множества} $A\subseteq X$
называют множество
\[
	\widetilde f(A):=\left\{y\in Y\;\big|\; \exists x\in A:y=f(x)\right\},
\]
то есть в $\widetilde f(A)$ лежат образы элементов $A$. Таким образом, мы определили
отображение $\widetilde f:2^{A}\to 2^{B}$ ---
{\it естественное расширение функции} $f$, обычно его обозначают
тем же знаком.

{\it Прообразом множества} $B\subseteq Y$ называют множество
\[
	f^{-1}(B):=\left\{x\in A\;\big|\; f(x)\in B\right\},
\]
то есть в $f^{-1}(B)$ лежат все прообразы элементов $B$.

Например, пусть $f:\Z\to\Z$ и $f(x)=|x|$, тогда
\[
	f(\{0,-1,1,2\})=\{0,1,2\}\qquad f^{-1}(\{0,1,2\})=\{0,-1,1,-2,2\}
\]

Прообразы и образы отображений --- одни из основных способов выделения
подмножеств множеств $X$ и $Y$.

Будем говорить, что отображение $f:X\to Y$ {\it сюръективно}, если $f(X)=Y$,
то есть у каждого $y\in Y$ существует (не обязательно единственный) прообраз.

Рассмотрим множество $U$ и сюръекцию ${u:I\to U}$,
введём обозначения
\[
	u_{i}\equiv u(i)\qquad \{u_{i}\}_{i\in I}\equiv u(I)=U,
\]
тогда $U$ называют {\it индексированным семейством}, $I$ --- {\it множеством индексов},
а $u_{i}$ --- {\it общим членом}.

Индексированное семейство определяется множеством индексов $I$ и отображением
${u:I\to U'}$, тогда $U=u(I)\subseteq U'$,
причём каждому $a\in U$ соответствует хотя бы один индекс $i\in I$.

Например, пусть $I=\Z$ и $a:I\to \Z$, $a(n)=2n$, тогда
\[
	a_{n}\equiv a(n)=2n\qquad A:=\{a_{n}\}_{n\in\Z}=\{2n\}_{n\in\Z}=a(\Z)
\]
Заметим, что $a$ не является сюръекцией и $A\neq \Z$.

Часто общий член и семейство обозначают одним знаком.
Объединение и пересечение множеств из семейства $\{U_{i}\}_{i\in I}$
обычно записывают так:
\[
	\bigcup_{i\in I}U_{i}\equiv\cup \{U_{i}\}_{i\in I}\qquad
	\bigcap_{i\in I}U_{i}\equiv\cap \{U_{i}\}_{i\in I}
\]

Пусть $\N$ --- множество натуральных чисел. Если $\N$ --- множество индексов
семейства множеств $\{U_{k}\}_{k\in\N}$, то принимаются следующие обозначения:
\[
	\{U_{i}\}_{k=1}^{\infty}\equiv\{U_{i}\}_{k\in\N}\qquad
	\bigcup_{k=1}^{\infty}U_{k}\equiv\bigcup_{k\in\N}U_{k}\qquad
	\bigcap_{k=1}^{\infty}U_{k}\equiv\bigcap_{k\in\N}U_{k}
\]

Если $I=\{1,2,...,n\}$, то для семейства множеств $\{U_{k}\}_{k\in I}$ принимают
обозначения
\[
	\{U_{i}\}_{k=1}^{n}\equiv\{U_{k}\}_{k\in I}\qquad
	\bigcup_{k=1}^{n}U_{k}\equiv\bigcup_{k\in I}U_{k}\qquad
	\bigcap_{k=1}^{n}U_{k}\equiv\bigcap_{k\in I}U_{k}
\]
причём
\[
	\bigcup_{k=1}^{n}U_{k}=U_1\cup U_2\cup ...\cup U_{n}\qquad
	\bigcap_{k=1}^{n}U_{k}=U_1\cap U_2\cap ...\cap U_{n}
\]

\vspace{1em}
{\it Упражнения:}
\begin{enumerate}
	\item{}Доказать, что $y\in f(A)$ ттк $y$ является образом элемента из $A$.
	\item{}Пусть ${f:X\to Y}$, ${A_{X},B_{X}\subseteq X}$ и
		${A_{Y},B_{Y}\in Y}$\footnote{Запись $a_1,a_2,...,a_{n}\in A$ означает
		$a_1\in A$, $a_2\in A$,...,$a_{n}\in A$}.
		Доказать следующие утверждения:
		\begin{enumerate}
			\item{}$A_{X}\subseteq B_{X}\implies f(A_{X})\subseteq f(B_{X})$
			\item{}$A_{Y}\subseteq B_{Y}\implies f^{-1}(A_{Y})\subseteq f^{-1}(B_{Y})$
			\item{}$A_{X}\subseteq f^{-1}(f(A_{X}))$
			\item{}$A_{Y}=f(f^{-1}(A_{Y}))$
			\item{}$f^{-1}(f(f^{-1}(f(A_{X}))))=f^{-1}(f(A_{X}))$
			\item{}$f(A_{X})\cap f(B_{X})=\eset\implies A\cap B=\eset$
			\item{}$A_{Y}\cap B_{Y}=\eset\implies f^{-1}(A_{Y})\cap f^{-1}(B_{Y})=\eset$
		\end{enumerate}
	\item{}*Могут ли существовать отображения
		\begin{multicols}{3}
			\begin{enumerate}
				\item{}$f:\eset\to A$
				\item{}$f:A\to\eset$
				\item{}$f:\eset\to\eset$
			\end{enumerate}
		\end{multicols}
\end{enumerate}

\section{Частичный порядок}

\begin{marginfigure}
	\center
	\begin{tikzpicture}
		\node (e) at (-1.5,2) {$\eset$};
		\node (a) at (0,3) {$\{a\}$};
		\node (b) at (0,1) {$\{b\}$};
		\node (ab) at (2,1) {$\{a,b\}$};
		\node (abc) at (2,3) {$\{a,b,c\}$};

		\draw [-Latex] (e) -- (a);
		\draw [-Latex] (e) -- (b);
		\draw [-Latex] (e) -- (ab);
		\draw [-Latex] (e) -- (abc);

		\draw [-Latex] (a) -- (ab);
		\draw [-Latex] (a) -- (abc);

		\draw [-Latex] (b) -- (ab);
		\draw [-Latex] (b) -- (abc);

		\draw [-Latex] (ab) -- (abc);
	\end{tikzpicture}

	\caption{Диаграмма отношения включения.}\label{fig:inc_diag}
\end{marginfigure}

\begin{marginfigure}
	\center
	\begin{tikzpicture}
		\node (e) at (-1.5,2) {$\eset$};
		\node (a) at (0,3) {$\{a\}$};
		\node (b) at (0,1) {$\{b\}$};
		\node (ab) at (2,1) {$\{a,b\}$};
		\node (abc) at (2,3) {$\{a,b,c\}$};

		\draw [-Latex] (e) -- (a);
		\draw [-Latex] (e) -- (b);

		\draw [-Latex] (a) -- (ab);

		\draw [-Latex] (b) -- (ab);

		\draw [-Latex] (ab) -- (abc);
	\end{tikzpicture}

	\caption{Сокращённая диаграмма отношения включения.}\label{fig:inc_diag_short}
\end{marginfigure}

Возьмём бинарное отношение $R$ на множестве $X$. $R$ называют {\it частичным порядком},
если оно имеет следующие свойства:
\begin{enumerate}
	\item{}$(\forall x\in X)~xRx$ --- {\it рефлективность}.
	\item{}$(\forall x,y\in X)~xRy\land yRx\implies x=y$
		--- {\it аснтиимметричность}.
	\item{}$(\forall x,y,z\in X)~xRy\land yRz\implies xRz$ --- {\it транзитивность}.
\end{enumerate}
Например, отношение $\leq$ на множестве $\Z$ является отношением частичного порядка.
Для произвольного множества $S$ отношение включения является
отношением частичного порядка на $2^{S}$.

Благодаря транзитивности бинарные отношения можно изображать в виде диаграмм, где
различные $x$ и $y$ соединены путём по направлению стрелок ттк $xRy$.
Это позволяет опустить стрелку $xRy$, если существует такой $z$, что $xRz$ и $zRy$.
Примеры диаграмм отношения включения на множестве
$\{\eset,\{a\},\{b\},\{a,b\}, \{a,b,c\}\}$
можно видеть на рис.~\ref{fig:inc_diag},~\ref{fig:inc_diag_short}.

Если $xRy$ или $yRx$, то элементы $x$ и $y$ называются {\it сравнимыми}.
Отношение частичного порядка называется отношением порядка, если также выполняется
\[
	(\forall x,y\in X)~xRy\lor yRx,
\]
то есть любые 2 элемента сравнимы. Отношение $\leq$ на $\Z$ является отношением порядка,
но отношение включения на $2^{\{a,b\}}$ таковым не является,
потому что $\{a\}$ и $\{b\}$ несравнимы.

Пусть $\preceq$ --- произвольный частичный порядок на $X$.
Элемент $x_0\in X$ называют {\it минимальным} по $\preceq$, если
\[
	(\forall x\in X)~x\npreceq x_0
\]
Элемент $x_0\in X$ называют {\it наименьшим} по $\preceq$, если
\[
	(\forall x\in X)~x_0\preceq x
\]
Аналогично определяются {\it максимальные} и {\it наибольшие} элементы.
Заметим, что из наименьшести (наибольшести) следует минимальность (максимальность).

\begin{marginfigure}
	\center
	\begin{tikzpicture}
		\node (a) at (0,0) {$0$};
		\node (b) at (1,0) {$1$};
		\node (c) at (2,0) {$2$};

		\draw [-Latex] (a) -- (b);
		\draw [-Latex] (b) -- (c);
	\end{tikzpicture}

	\caption{$\leq$ на $\{0,1,2\}$}\label{fig:less_higher}
\end{marginfigure}

\begin{marginfigure}
	\center
	\begin{tikzpicture}
		\node (a) at (0,1.5) {$\{a\}$};
		\node (b) at (0,0.5) {$\{b\}$};
		\node (ab) at (2,1) {$\{a,b\}$};

		\draw [-Latex] (a) -- (ab);
		\draw [-Latex] (b) -- (ab);
	\end{tikzpicture}

	\caption{$\subseteq$ на $\{\{a\},\{b\},\{a,b\}\}$}\label{fig:min_max}
\end{marginfigure}

Таким образом, множество $\{0,1,2\}$ содержит минимальные и максимальные по $\leq$
элементы $0$ и $1$, они же являются наименьшим и
наибольшим элементом соответственно (см.~рис.~\ref{fig:less_higher}).

Множество $\{\{a\},\{b\},\{a,b\}\}$ содержит
два минимальных: $\{a\}$ и $\{b\}$ и один максимальный
по включению элемент: $\{a,b\}$. Причём элемент $\{a,b\}$ является наибольшим, а
наименьшего элемента не существует (см.~рис.~\ref{fig:min_max}).

\textsc{Другими словами, элемент является наименьшим,
если он меньше других и является минимальным, если его нельзя
``уменьшить''.} Аналогично с наибольшестью и максимальностью.

\vspace{1em}
{\it Теорема:} Пусть $A,B\subseteq U$. Тогда множество $A\cap B$ ---
наибольшее по включению подмножество $S$, содержащееся\footnote{Если $T\subseteq V$,
то говорят, что $T$ {\it содержится} в $V$} в них обоих,
а множество $A\cup B$ --- наименьшее по включению подмножество $S$,
содержащее их оба.

{\it Доказательство:}
\[
	S_{*}:=\left\{X\in S\;\big|\; X\subseteq A\land X\subseteq B\right\}
\]
\[
	S^{*}:=\left\{X\in S\;\big|\; A\subseteq X\land B\subseteq X\right\}
\]

Нужно доказать, что множество $A\cap B$ --- наибольший по включению элемент $S_{*}$,
а $A\cup B$ --- наименьший $S^{*}$.
\[
	A\cap B\subseteq A\quad A\cap B\subseteq B\quad
	A\subseteq A\cup B\quad B\subseteq A\cup B,
\]
значит $A\cap B\in S_{*}$ и $A\cup B\in S^{*}$.

Пусть $Z\in S_{*}$ и $x\in Z$, тогда
\[
	Z\in S_{*}\iff Z\subseteq A\land Z\subseteq B
\]
\[
	x\in Z\implies x\in A\land x\in B\implies x\in A\cap B,
\]
что справедливо для любого $x$,
значит $Z\subseteq A\cap B$ и $A\cap B$ является наибольшим элементом $S_{*}$.

Пусть $Z\in S^{*}$ и $x\in A\cup B$, тогда
\[
	Z\in S^{*}\iff A\subseteq Z\land B\subseteq Z
\]
\[
	x\in A\cup B\implies x\in A\land x\in B\implies x\in Z
\]
что справедливо для любого $x$,
значит $A\cup B\subseteq Z$ и $A\cup B$ является наименьшим элементом $S^{*}$.\qed

Возьмём множество $A$ c произвольным частичным порядком $\leq$. Отображение
$\Gamma:A\to A$ называют {\it замыканием на $A$ с $\leq$}, если
\begin{enumerate}
	\item{}$(\forall a\in A)~a\leq\Gamma(a)$
	\item{}$(\forall a,b\in A)~a\leq b\implies \Gamma(a)\leq \Gamma(b)$
	\item{}$(\forall a\in A)~\Gamma(\Gamma(a))=\Gamma(a)$
\end{enumerate}
Элемент ${a\in A}$ называют {\it замкнутым по $\Gamma$}, если ${\Gamma(a)=a}$.

Например, рассмотрим $I$ --- множество всех интервалов $(a,b)$, $[a,b]$, $(a,b]$
и $[a,b)$, где $a,b$ --- конечные числа, с частичным порядком $\subseteq$.
Тогда отображение $\Gamma:I\to I$, определённое как
\[\begin{array}{ll}
	\Gamma((a,b)):=[a,b]\qquad &\Gamma([a,b]):=[a,b]\\
	\Gamma((a,b]):=[a,b]\qquad &\Gamma([a,b)):=[a,b]
\end{array}
\]
является замыканием, а $[a,b]$ --- замкнутые по $\Gamma$ элементы $I$.

\vspace{1em}
{\it Теорема:} $\Gamma(a)$ --- наименьший по $\leq$ замкнутый элемент $A$,
больший (содержащий\footnote{В зависимости от отношения $\leq$ формулу
$a\leq b$ также иногда читают как ``$b$ содержит $a$''.}) $a$.

{\it Доказательство:}
\[
	\gamma:=\left\{q\in A\;\big|\; \Gamma(q)=q\land a\leq q\right\}
\]
Нужно доказать, что $\Gamma(a)$ --- наименьший элемент $\gamma$.
\[
	\Gamma(\Gamma(a))=\Gamma(a)\land a\leq\Gamma(a)\implies\Gamma(a)\in\gamma
\]

Пусть $q\in \gamma$, тогда
\[
	a\leq q\implies \Gamma(a)\leq\Gamma(q)=q\implies\Gamma(a)\leq q
\]
и $\Gamma(a)$ --- минимальный элемент $\gamma$.\qed

Возьмём множество $S$ и представим его как объединение семейства непустых
непересекающихся множеств $\{B_{i}\}_{i\in I}$.
\[
	S=\bigcup_{i\in I}B_{i}\qquad
	(\forall i,j\in I)~i\neq j\implies B_{i}\cap B_{j}=\eset\qquad
	(\forall i\in I)~B_{i}\neq\eset
\]
Такое представление называется {\it разбиением}.
Определим следующее множество:
\[
	T:=\left\{\alpha\in 2^{S}\;\big|\; (\forall x,y\in\alpha,i,j\in I)~
	x\neq y\land	x\in B_{i}\land y\in B_{j}\implies i\neq j\right\},
\]
то есть все элементы множества $\alpha\in T$ содержатся в разных $B_{i}$.

\vspace{1em}
{\it Теорема:} Если $\alpha$ --- максимальный по включению элемент $T$, то
\[
	(\forall i\in I)~\exists a\in\alpha:a\in B_{i}
\]

{\it Доказательство:}
Пусть $\alpha$ --- максимальный по включению элемент $T$ и
\[
	\exists i\in I:(\forall a\in\alpha)~a\notin B_{i}
\]

Множество $B_{i}$ не пустое, значит существует элемент $b\in B_{i}$.
Тогда множество $\alpha\cup \{b\}$ лежит в $T$, что противоречит
максимальности $\alpha$\footnote{Вспомним, что максимальный элемент
нельзя ``увеличить''.}.
Источник противоречия --- отрицание утверждения теоремы.\qed

Например, пусть $S=\{a,b,c\}$, $B_1=\{a,c\}$ и $B_2=\{b\}$, тогда
и $\{a,b\}$, $\{b,c\}$ --- максимальные элементы $T$.
Наглядно это можно увидеть на рис. $\ref{fig:inc_t}$.

\begin{marginfigure}
	\center
	\begin{tikzpicture}
		\node (e) at (-2,2) {$\eset$};

		\node (a) at (0,3) {$\{a\}$};
		\node (b) at (0,2) {$\{b\}$};
		\node (c) at (0,1) {$\{c\}$};

		\node (ab) at (2,2.5) {$\{a,b\}$};
		\node (bc) at (2,1.5) {$\{b,c\}$};

		\draw [-Latex] (e) -- (a);
		\draw [-Latex] (e) -- (b);
		\draw [-Latex] (e) -- (c);

		\draw [-Latex] (a) -- (ab);
		\draw [-Latex] (b) -- (ab);
		\draw [-Latex] (b) -- (bc);
		\draw [-Latex] (c) -- (bc);
	\end{tikzpicture}

	\caption{Отношение $\subseteq$ на $T$.}\label{fig:inc_t}
\end{marginfigure}

\newcommand\Q{\mathbb Q}
\vspace{1em}
{\it Упражнения:}
\begin{enumerate}
	\item{}Доказать, что $\leq$ на $\Z$ и $\subseteq$ на $2^{S}$ --- частичные порядки.
	\item{}Отношение строгого порядка
	\item{}Найти максимальные, минимальные, наибольшие и
		наименьшие по включению элементы $2^{S}$, где $S$ --- произвольное.
	\item{}Пусть $f:X\to Y$. Доказать, что отображение
		\[
			\Gamma:2^{X}\to 2^{X},\quad \Gamma(A):= f^{-1}(f(A))
		\]
		является замыканием на $2^{X}$ с $\subseteq$.
	\item{}*Доказать, что если наименьший элемент существует, то
		он единственен.
	\item{}*Пусть $\leq$ --- порядок (не частичный) на $X$, а $x_0\in X$ --- минимальный
		элемент. Доказать, что $x_0$ --- наименьший элемент.
	\item{}*Пусть $\Q$ --- множество рациональных чисел.
		Пусть $A\subseteq \Q$ и множество
		\[
			M:=\left\{x\in \Q\;\big|\;(\forall a\in A)~a\leq x\right\}
		\]
		непустое. Существует ли в $M$ наименьший по $\leq$ элемент?
\end{enumerate}
