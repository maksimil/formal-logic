\part{Заключение}

\section{Истина в математике}

Истинность любого утверждения в математике исходит только из определения правил
вывода и определения аксиом как истинных утверждений.
То есть $1+1=2$ --- истина по определению формальной системы\footnote{
	Поэтому истинные утверждения в математике часто называют тавтологиями.
}.
С помощью натуральных чисел мы формализуем понятие ``если взять одно яблоко
и ещё одно яблоко, то будет два яблока''.

Важно понимать связь между понятием истины в формальных системах
и эмпиричной истины\footnote{Эмпиризм --- метод познания через ощущение (наблюдение).}.
Физики подбирают аксиомы (постулаты),
из которых выводимы теоремы (законы), соответствующие наблюдениям.
Они подгоняют истину формальной системы к эмпиричной истине.
Они формализуют наблюдения в постулатах, чтобы использовать достаточно мощные
инструменты матанализа для предсказания движения тел.

\section{В защиту интуиции}

Ценность формальных доказательств в их точности. Каждый шаг полностью обоснован и,
поэтому, хорошее формальное доказательство неоспоримо.

Но нельзя забывать и о ценности интуитивного понимания формул.
Оно легче укладывается в голове\footnote{Например, диаграммы
	Эйлера-Венна помогают понять и запомнить операции и теоремы, связанные с
	множествами, но их нельзя приводить как доказательства.},
а интуитивное доказательство может натолкнуть на формальное. Поэтому я привожу пример
некоторых словесных интерпретаций формул:
\begin{table}
	\centering
	\begin{tabular}{r|l}
		$\exists k:(\forall n>k)~P$     & $P$ начиная с какого-то $n$       \\
		                                & $P$ для достаточно больших $n$    \\[1em]
		$\exists \delta:\forall x~
		\big(|x|<\delta\implies P\big)$ & $P$ для достаточно малых $x$      \\[1em]
		$\forall \varepsilon~
		\exists \delta:P$               & Для всякого $\varepsilon$ можно   \\
		                                & подобрать такой $\delta$, что $P$
	\end{tabular}
	\caption{Интерпретации формул}\label{table:formula_interp}
\end{table}

{\it Упражнения:}
\begin{enumerate}
	\item{}Обосновать интерпретации в таблице~\ref{table:formula_interp}.
	\item{}Взять лист бумаги и записать все правила и определения, составляющие
	формальную систему, в которой мы работали, то есть построить её с нуля.
\end{enumerate}


