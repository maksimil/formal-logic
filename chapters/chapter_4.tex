\section{Отображения}

\newcommand\G{\mathcal G}
\index{отображение}\index{область!определения}
\index{область!значений}\index{функция}\index{график}
{\it Отображением из $X$ в $Y$} называют тройку $f=(X,Y,G)$, где
\[
  G\subseteq X\times Y\qquad
  \forall x\in X~\exists! y\in Y:(x,y)\in G
\]
Множество $X$ называют {\it областью определения} отображения $f$,
$Y$ --- {\it областью значений}, а $G$ --- {\it графиком}.
Отображения также называют {\it функциями}.

Обозначим $\G(f):=\pi_{3}^{3}(f)$ --- график~$f$.
Запись ${f:X\to Y}$ означает, что $f$ является отображением из
области определения $X$ в область значений $Y$, то есть
\[
  \begin{aligned}
     & f:X\to Y\iff \\
     & \iff
    f\in \{X\}\times \{Y\}\times 2^{X\times Y}\land
    \forall x\in X~\exists !y\in Y:(x,y)\in \G(f)
  \end{aligned}
\]
Можем ввести функциональный знак $\mathcal F$ и аксиому
\[
  (\forall X,Y,f,x)~f:X\to Y\land x\in X\implies
  (x,\mathcal F(X,Y,f,x))\in \G(f)
\]
Тогда обозначим $f(x)\equiv \mathcal F(X,Y,f,x)$.

Чтобы задать отображение, нужно определить его график, но
из-за громоздкости такого подхода отображения чаще задают правилами,
по которым каждому элементу $x\in X$
ставят в соответствие единствененный элемент $f(x)\in Y$.
По такому правилу уже можно
построить график, однако в некоторых случаях
эта задача может оказаться нетривиальной\footnote{
  Тривиальный --- крайне простой, примитивный.}.

Например, пусть для формулы $\varphi(x,y)$ выполняется
\[
  \forall x\in X~\exists !y\in Y:\varphi(x,y),
\]
тогда $f=(X,Y,G)$, где
$G=\left\{(x,y)\in X\times Y\;\big|\; \varphi(x,y)\right\}$,
является отображением, причём $(\forall x\in X)~\varphi(x,f(x))$.

Рассмотрим формулу ${y=2x}$. Можем определить отображение
\[
  f:\Z\to\Z,\qquad
  \G(f)=\left\{(x,y)\in \Z\times \Z\;\big|\; y=2x\right\}
\]
Такое определение можно записать как $f:\Z\to\Z,f(x):=2x$.

\index{отображение!$n$-арное}
Для отображений вида ${g:X_1\times ...\times X_{n}\to Y}$
введём обозначение $g(x_1,...,x_{n})\equiv g((x_1,...,x_{n}))$.
Такие отображения называют {\it \mbox{$n$-арными}}.
Например, сумму натуральных чисел можно
задать как бинарное отображение $(+):\N^{2}\to\N$.

Возьмём ${f:X\to Y}$. Если ${y=f(x)}$, то $y$ называют {\it значением $f$ в $x$}
\index{значение отображения} или {\it образом $x$ по $f$}\index{образ},
а $x$ --- {\it прообразом $y$ по $f$}\index{прообраз}.

{\it Образом множества} $A\subseteq X$\index{образ!множества}
называют множество
\[
  \widetilde f(A):=\left\{y\in Y\;\big|\; \exists x\in A:y=f(x)\right\},
\]
то есть в $\widetilde f(A)$ лежат образы элементов $A$. Таким образом, мы определили
отображение $\widetilde f:2^{X}\to 2^{Y}$ ---
{\it естественное расширение отображения}\index{естественное расширение}
$f$, обычно его обозначают
тем же знаком\footnote{Мы можем себе это позволить, поскольку
  в большинстве случаев из контекста понятно, что имеется
  в виду под знаком $f$.}.

{\it Прообразом множества}\index{прообраз!множества}
$B\subseteq Y$ называют множество
\[
  f^{-1}(B):=\left\{x\in A\;\big|\; f(x)\in B\right\},
\]
то есть в $f^{-1}(B)$ лежат все прообразы элементов $B$.

Например, пусть $f:\Z\to\Z$ и $f(x)=|x|$, тогда
\[
  f(\{0,-1,1,2\})=\{0,1,2\}\qquad f^{-1}(\{0,1,2\})=\{0,-1,1,-2,2\}
\]

Прообразы и образы отображений --- одни из основных способов выделения
подмножеств множеств $X$ и $Y$.

\vspace{1em}

{\it Упражнения:}
\begin{enumerate}
  \item{}Пусть ${f:X\to Y}$, $A\subseteq X$ и $B\subseteq Y$.
  \begin{enumerate}
    \item{}Почему для $y\in f(A)$ необходимо и достаточно,
    чтобы $y$ был образом элемента $A$?
    \item{}Почему для $x\in f^{-1}(B)$ необходимо и достаточно,
    чтобы $x$ был прообразом элемента $B$?
  \end{enumerate}
  \item{}Пусть ${f:X\to Y}$, ${A_{X},B_{X}\subseteq X}$ и
  ${A_{Y},B_{Y}\subseteq Y}$. Доказать
  \begin{fullwidth}
    \begin{multicols}{2}
      \begin{enumerate}
        \item{}$A_{X}\subseteq B_{X}\implies f(A_{X})\subseteq f(B_{X})$
        \item{}$A_{Y}\subseteq B_{Y}\implies f^{-1}(A_{Y})\subseteq f^{-1}(B_{Y})$
        \item{}$A_{X}\subseteq f^{-1}(f(A_{X}))$
        \item{}$A_{Y}=f(f^{-1}(A_{Y}))$
        \item{}$f^{-1}(f(f^{-1}(f(A_{X}))))=f^{-1}(f(A_{X}))$
        \item{}$f(A_{X})\cap f(B_{X})=\eset\implies A\cap B=\eset$
        \item{}$A_{Y}\cap B_{Y}=\eset\implies f^{-1}(A_{Y})\cap f^{-1}(B_{Y})=\eset$
      \end{enumerate}
    \end{multicols}
  \end{fullwidth}
  \item{}*Могут ли существовать отображения
  \begin{multicols}{3}
    \begin{enumerate}
      \item{}$f:\eset\to A$
      \item{}$f:A\to\eset$
      \item{}$f:\eset\to\eset$
    \end{enumerate}
  \end{multicols}
\end{enumerate}

\section{Индексированные семейства}

\index{индексированное семейство}\index{множество!индексов}
\index{общий член}\index{индекс}
Рассмотрим отображение $u:I\to U$, введём обозначения
\[
  u_{i}\equiv u(i)\qquad \{u_{i}\}_{i\in I}\equiv u(I),
\]
Множество $\{u_{i}\}_{i\in I}$ называют {\it индексированным семейством},\\
$I$ --- {\it множеством индексов}, $u_{i}$ --- {\it общим членом},
$i$ --- его {\it индексом}.

Например, пусть $I=\Z$ и $a:I\to \Z$, $a(n)=2n$, тогда
\[
  a_{n}\equiv a(n)=2n\qquad \{a_{n}\}_{n\in\Z}=\{2n\}_{n\in\Z}=a(\Z)
\]

Объединение и пересечение множеств из семейства $\{U_{i}\}_{i\in I}$
часто записывают как
\[
  \bigcup_{i\in I}U_{i}\equiv\cup \{U_{i}\}_{i\in I}\qquad
  \bigcap_{i\in I}U_{i}\equiv\cap \{U_{i}\}_{i\in I}
\]
Если $I=\N$, то принимают обозначения
\[
  \{U_{k}\}_{k=1}^{\infty}\equiv\{U_{k}\}_{k\in\N}\qquad
  \bigcup_{k=1}^{\infty}U_{k}\equiv\bigcup_{k\in\N}U_{k}\qquad
  \bigcap_{k=1}^{\infty}U_{k}\equiv\bigcap_{k\in\N}U_{k}
\]
Если $I=\{1,2,...,n\}=\left\{m\in\N\;\big|\;m\leq n\right\},n\in\N$, то
\[
  \{U_{k}\}_{k=1}^{n}\equiv\{U_{k}\}_{k\in I}\qquad
  \bigcup_{k=1}^{n}U_{k}\equiv\bigcup_{k\in I}U_{k}\qquad
  \bigcap_{k=1}^{n}U_{k}\equiv\bigcap_{k\in I}U_{k}
\]
% \[
%   \bigcup_{k=1}^{n}U_{k}=U_1\cup U_2\cup ...\cup U_{n}\qquad
%   \bigcap_{k=1}^{n}U_{k}=U_1\cap U_2\cap ...\cap U_{n}
% \]

% Вспомним определение натуральных чисел из множеств.
% Множество $X$ называют {\it конечным}, если
% \[
%   \exists n\in\N,f:(f:n\to X)\land f(n)=X,
% \]
% то есть когда $X$ можно ``пронумеровать конечным количеством натуральных чисел''.

\section{Бинарные отношения}

\index{бинарное отношение!на}\index{бинарное отношение}\index{бинарное отношение!между}
Рассмотрим множества $X,Y$ и $R\subseteq X\times Y$. Множество $R$ можно рассматривать
как определённое отношение между элементами множеств $X$ и $Y$. Можем ввести бинарный
предикатный знак
\[
  xRy\iff (x,y)\in R
\]
Множество $R$ называют {\it бинарным отношением между}
$X$ и $Y$. Если множества $X$ и $Y$
совпадают, то $R$ называют {\it бинарным отношением на} $X$.
Обычно бинарный предикатный знак и множество обозначают одним знаком.

Например, рассмотрим множество $S=2^{\{a,b\}}$.
\[
  S=\{\eset,\{a\},\{b\},\{a,b\}\}
\]
Пусть $R\subseteq S^{2}$ --- отношение включения ($\subseteq$) на $S$, тогда
\[
  \begin{aligned}
    R
     & =\left\{(x,y)\in S^{2}\;\big|\; x\subseteq y\right\}= \\
     & =\left\{
    \begin{array}{cccc}
      (\eset,\eset), & (\eset,\{a\}),   & (\eset,\{b\}),   & (\eset,\{a,b\}),  \\
      (\{a\},\{a\}), & (\{a\},\{a,b\}), & (\{b\},\{a,b\}), & (\{a,b\},\{a,b\})
    \end{array}\right\}
  \end{aligned}
\]

\vspace{1em}
{\it Упражнения:}
\begin{enumerate}
  \item{}Построить множества бинарных отношений
  \begin{multicols}{2}
    \begin{enumerate}
      \item{}$\subseteq$ на $2^{\{a\}}$
      \item{}$\subseteq$ на $2^{\{a,b,c\}}$
      \item{}$=$ на $\{a,b\}$
      \item{}$\leq$ на $\Z$
      \item{}Равенство по остатку при делении на 3 на $\Z$.
    \end{enumerate}
  \end{multicols}
\end{enumerate}

\pagebreak

\section{Частичный порядок}

\begin{marginfigure}[1cm]
  \center
  \begin{tikzpicture}
    \node (e) at (-1.5,2) {$\eset$};
    \node (a) at (0,3) {$\{a\}$};
    \node (b) at (0,1) {$\{b\}$};
    \node (ab) at (2,1) {$\{a,b\}$};
    \node (abc) at (2,3) {$\{a,b,c\}$};

    \draw [-Latex] (e) -- (a);
    \draw [-Latex] (e) -- (b);
    \draw [-Latex] (e) -- (ab);
    \draw [-Latex] (e) -- (abc);

    \draw [-Latex] (a) -- (ab);
    \draw [-Latex] (a) -- (abc);

    \draw [-Latex] (b) -- (ab);
    \draw [-Latex] (b) -- (abc);

    \draw [-Latex] (ab) -- (abc);
  \end{tikzpicture}

  \caption{Диаграмма отношения включения.}\label{fig:inc_diag}
\end{marginfigure}

\begin{marginfigure}
  \center
  \begin{tikzpicture}
    \node (e) at (-1.5,2) {$\eset$};
    \node (a) at (0,3) {$\{a\}$};
    \node (b) at (0,1) {$\{b\}$};
    \node (ab) at (2,1) {$\{a,b\}$};
    \node (abc) at (2,3) {$\{a,b,c\}$};

    \draw [-Latex] (e) -- (a);
    \draw [-Latex] (e) -- (b);

    \draw [-Latex] (a) -- (ab);

    \draw [-Latex] (b) -- (ab);

    \draw [-Latex] (ab) -- (abc);
  \end{tikzpicture}

  \caption{Сокращённая диаграмма отношения включения.}\label{fig:inc_diag_short}
\end{marginfigure}

\index{порядок!частичный}\index{рефлексивность}
\index{антиcимметричность}\index{транзитивность}
Возьмём бинарное отношение $R$ на множестве $X$. $R$ называют
{\it частичным порядком}\index{частичный порядок},
если оно имеет следующие свойства:
\begin{enumerate}
  \item{}$(\forall x\in X)~xRx$ --- {\it рефлексивность}.
  \item{}$(\forall x,y\in X)~xRy\land yRx\implies x=y$
  --- {\it антиcимметричность}.
  \item{}$(\forall x,y,z\in X)~xRy\land yRz\implies xRz$
  --- {\it транзитивность}.
\end{enumerate}
Например, отношение $\leq$ на множестве $\Z$ является отношением частичного порядка.
Для произвольного множества $S$ отношение включения является
отношением частичного порядка на $2^{S}$.

Благодаря транзитивности бинарные отношения можно изображать в виде диаграмм, где
различные $x$ и $y$ соединены путём по направлению стрелок тогда и только
тогда, когда $xRy$.
Это позволяет опустить стрелку $xRy$, если существует такой $z$, что $xRz$ и $zRy$.
Примеры диаграмм отношения включения на множестве
$\{\eset,\{a\},\{b\},\{a,b\}, \{a,b,c\}\}$
можно видеть на рис.~\ref{fig:inc_diag},~\ref{fig:inc_diag_short}.

\index{сравнимая пара}\index{сравнимые элементы}
\index{порядок!линейный}\index{линейный порядок}
\index{полнота}
Если $xRy$ или $yRx$, то элементы $x$ и $y$ называются
{\it сравнимыми}, а пара $(x,y)$ --- {\it сравнимой}.
Отношение частичного порядка называют
отношением {\it линейного порядка}, если также выполняется
\begin{enumerate}[resume*]
  \item{}$(\forall x,y\in X)~xRy\lor yRx$ --- {\it полнота}.
\end{enumerate}
То есть любая пара элементов сравнима. Отношение $\leq$ на $\Z$
является отношением линейного порядка,
но отношение включения на $2^{\{a,b\}}$ таковым не является,
потому что $\{a\}$ и $\{b\}$ несравнимы.

Пусть $\preceq$ --- произвольный частичный порядок на $X$.
Элемент $x_0\in X$ называют {\it минимальным}
\index{элемент!минимальный}\index{минимальный элемент}
по $\preceq$, если
\[
  (\forall x\in X)~x\npreceq x_0
\]
Элемент $x_0\in X$ называют {\it наименьшим}
\index{элемент!наименьший}\index{наименьший элемент} по $\preceq$, если
\[
  (\forall x\in X)~x_0\preceq x
\]
Другими словами, \textsc{элемент является наименьшим,
  если он меньше всех элементов и является минимальным, если его нельзя
  ``уменьшить''.}

Аналогично определяют {\it максимальные} и {\it наибольшие} элементы.
Заметим, что из наименьшести (наибольшести) следует минимальность (максимальность).
\index{элемент!максимальный}\index{максимальный элемент}
\index{элемент!наибольший}\index{наибольший элемент}

\begin{marginfigure}
  \center
  \begin{tikzpicture}
    \node (a) at (0,0) {$0$};
    \node (b) at (1,0) {$1$};
    \node (c) at (2,0) {$2$};

    \draw [-Latex] (a) -- (b);
    \draw [-Latex] (b) -- (c);
  \end{tikzpicture}

  \caption{$\leq$ на $\{0,1,2\}$}\label{fig:less_higher}
\end{marginfigure}

\begin{marginfigure}
  \center
  \begin{tikzpicture}
    \node (a) at (0,1.5) {$\{a\}$};
    \node (b) at (0,0.5) {$\{b\}$};
    \node (ab) at (2,1) {$\{a,b\}$};

    \draw [-Latex] (a) -- (ab);
    \draw [-Latex] (b) -- (ab);
  \end{tikzpicture}

  \caption{$\subseteq$ на $\{\{a\},\{b\},\{a,b\}\}$}\label{fig:min_max}
\end{marginfigure}

Таким образом, множество $\{0,1,2\}$ содержит минимальные и максимальные по $\leq$
элементы: $0$ и $2$, они же являются наименьшим и
наибольшим элементом соответственно (см.~рис.~\ref{fig:less_higher}).

Множество $\{\{a\},\{b\},\{a,b\}\}$ содержит
два минимальных: $\{a\}$ и $\{b\}$ и один максимальный
по включению элемент: $\{a,b\}$. Причём элемент $\{a,b\}$ является наибольшим, а
наименьшего элемента не существует (см.~рис.~\ref{fig:min_max}).

\pagebreak
% \vspace{1em}
{\it Теорема:} Пусть $A,B\subseteq S$. Тогда множество $A\cap B$ ---
наибольшее по включению подмножество $S$, содержащееся\footnote{Если $T\subseteq V$,
  то говорят, что $T$ {\it содержится}\index{содержаться, $\subseteq$}
  в $V$} в них обоих,
а множество $A\cup B$ --- наименьшее по включению подмножество $S$,
содержащее их оба.

{\it Доказательство:}
\[
  S_{*}:=\left\{X\in 2^{S}\;\big|\; X\subseteq A\land X\subseteq B\right\}\qquad
  S^{*}:=\left\{X\in 2^{S}\;\big|\; A\subseteq X\land B\subseteq X\right\}
\]

Нужно доказать, что множество $A\cap B$ --- наибольший по включению элемент $S_{*}$,
а $A\cup B$ --- наименьший элемент $S^{*}$.
Очевидно, $A\cap B\in S_{*}$ и $A\cup B\in S^{*}$.

Пусть $Z\in S_{*}$ и $x\in Z$, тогда
\[
  x\in Z\implies x\in A\land x\in B\implies x\in A\cap B,
\]
что справедливо для любого $x$,
значит ${(\forall Z\in S_{*})~Z\subseteq A\cap B}$. Тогда
$A\cap B$ является наибольшим элементом $S_{*}$.
Аналогично доказываем, что $A\cup B$ --- наименьший элемент $S^{*}$.\qed

\newcommand\Q{\mathbb Q}
\vspace{1em}
{\it Упражнения:}
\begin{enumerate}
  \item{}Доказать, что $A\cup B$ --- наименьший элемент $S^{*}$.
  \item{}Доказать, что $\leq$ на $\Z$ и $\subseteq$ на $2^{S}$ --- частичные порядки.
  \item{}Пусть $S$ --- произвольное множество.
  Найти максимальные, минимальные, наибольшие и
  наименьшие по включению элементы множеств
  \[
    2^{S}\qquad 2^{S}\setminus\{\eset\}\qquad
    2^{S}\setminus \{S\}
  \]
  Нарисовать диаграммы для случая $S=\{a,b,c\}$.
  \item{}*Доказать, что если наименьший элемент существует, то
  он единственен.
  \item{}*Пусть $\leq$ --- линейный порядок на $X$, а $x_0\in X$ --- минимальный
  элемент. Доказать, что $x_0$ --- наименьший элемент.
  \index{множество!рациональных чисел, $\Q$}
  \item{}*Пусть $\Q$ --- множество рациональных чисел\footnote{
    Рациональные числа --- числа, которые можно представить в виде дроби
    $p/q$, где $p\in\Z,q\in\N$.}.
  Пусть $A\subseteq \Q$ и множество
  \[
    M:=\left\{x\in \Q\;\big|\;(\forall a\in A)~a\leq x\right\}
  \]
  не пустое. Существует ли в $M$ наименьший по $\leq$ элемент
  в следующих случаях:
  \begin{multicols}{2}
    \begin{enumerate}
      \item{}$A=[a,b]\cap\Q$, где $a,b\in\Q$
      \item{}$A=\{a_1,a_2,...,a_{n}\}$
      \item{}$A=\eset$
      \item{}$A=\left\{x\in\Q\;\big|\; x^{2}<2\right\}$
    \end{enumerate}
  \end{multicols}
\end{enumerate}

\section{Замыкания}

Возьмём множество $A$ c произвольным частичным порядком $\leq$. Отображение
$\Gamma:A\to A$ называют {\it замыканием на $A$ с $\leq$}\index{замыкание}, если
\begin{enumerate}
  \item{}$(\forall a\in A)~a\leq\Gamma(a)$
  \item{}$(\forall a,b\in A)~a\leq b\implies \Gamma(a)\leq \Gamma(b)$
  \item{}$(\forall a\in A)~\Gamma(\Gamma(a))=\Gamma(a)$
\end{enumerate}
Элемент ${a\in A}$ называют {\it замкнутым по $\Gamma$}\index{замкнутость},
если ${\Gamma(a)=a}$.

Например, пусть $I$ --- множество всех интервалов $(a,b)$, $[a,b]$, $(a,b]$
и $[a,b)$, где $a,b$ --- конечные числа.
Рассмотрим $I$ с частичным порядком $\subseteq$.
Отображение $\Gamma:I\to I$, определённое как
\[
  \Gamma((a,b)):=[a,b]\quad \Gamma([a,b]):=[a,b]\quad
  \Gamma((a,b]):=[a,b]\quad \Gamma([a,b)):=[a,b]
\]
является замыканием, а $[a,b]$ --- замкнутые по $\Gamma$ элементы $I$.

\vspace{1em}
{\it Теорема:} Пусть $a\in A$. Тогда $\Gamma(a)$ --- наименьший по $\leq$
замкнутый по $\Gamma$ элемент $A$,
больший (содержащий\footnote{В зависимости от отношения $\leq$ формулу
  $a\leq b$ также иногда читают как ``$b$ содержит $a$''.}) $a$.

  {\it Доказательство:}
\[
  \gamma:=\left\{q\in A\;\big|\; \Gamma(q)=q\land a\leq q\right\}
\]
Нужно доказать, что $\Gamma(a)$ --- наименьший элемент $\gamma$.
\[
  \Gamma(\Gamma(a))=\Gamma(a)\land a\leq\Gamma(a)\implies\Gamma(a)\in\gamma
\]
Пусть $q\in \gamma$, тогда
\[
  a\leq q\implies \Gamma(a)\leq\Gamma(q)=q\implies\Gamma(a)\leq q
\]
и $\Gamma(a)$ --- наименьший элемент $\gamma$.\qed

\newcommand\R{\mathbb R}
\vspace{1em}
{\it Теорема:}
Будем говорить, что ${a\in A}$ замкнут, если $P(a)$\footnote{В данном случае
$P(a)$ может быть любой формулой о $a$.

В примере она примет вид
\[
  P(I)\equiv \exists a,b\in\R:I=[a,b],
\]
где $\R$ --- множество действительных чисел.}.
Пусть ${\Gamma':A\to A}$ --- такое отображение, что $\Gamma'(a)$ --- наименьший по
$\leq$ замкнутый элемент $A$, содержащий $a$. Тогда $\Gamma'$ --- замыкание,
причём замкнутость эквивалентна замкнутости по $\Gamma'$.

  {\it Доказательство:}
Проверим свойства замыкания.
\begin{enumerate}
  \item{}$(\forall a\in A)~a\leq \Gamma'(a)$ по определению $\Gamma'$.

  \item{}Пусть $a,b\in A$ и $a\leq b$. Элемент $\Gamma'(b)$ замкнут и
  $a\leq b\leq\Gamma'(b)$, значит $a\leq\Gamma'(b)$. Тогда, по наименьшести,
  $\Gamma'(a)\leq\Gamma'(b)$.

  \item{}$a$ --- наименьший элемент, содержащий $a$, поэтому
  если $P(a)$, то $\Gamma'(a)=a$. Если $\Gamma'(a)=a$, то $P(a)$
  по определению $\Gamma'$.

  \item{}$\Gamma'(a)$ замкнут, поэтому $\Gamma'(\Gamma'(a))=\Gamma'(a)$.\qed
\end{enumerate}

{\it Упражнения:}
\begin{enumerate}
  \item{}Пусть $\Gamma:A\to A$ --- замыкание на $A$ с $\leq$ и $a$ ---
  максимальный по $\leq$ элемент $A$. Доказать, что $a$ замкнут по $\Gamma$.
  \item{}Пусть $f:X\to Y$. Доказать, что отображение
  \[
    \Gamma:2^{X}\to 2^{X},\quad \Gamma(A):= f^{-1}(f(A))
  \]
  является замыканием на $2^{X}$ с $\subseteq$. Какие множества
  будут замкнутыми, если $X,Y=\Z,f(x)=|x|$? Если $f(x)=x^{2}$?
  \item{}Пусть $A$ --- множество с частичным порядком $\leq$.
  Доказать, что $a\in A$ --- наибольший элемент меньший $a$
  и наименьший элемент больший $a$.
\end{enumerate}

\section{Разбиения}

\index{разбиение}
\newcommand\B{\mathcal B}
Возьмём множество $S$ и представим его как объединение непустых
попарно непересекающихся множеств.
\[
  S=\cup\B\qquad
  \eset\notin\B\qquad
  (\forall B_1,B_2\in\B:B_1\neq B_2)~B_1\cap B_2=\eset
\]
Такое представление называют {\it разбиением}. Пусть
\[
  T:=\left\{\alpha\in 2^{S}\;\big|\; (\forall x,y\in\alpha,B\in\B)~
  x,y\in B\implies x=y\right\},
\]
то есть любые два $x,y\in \alpha\in T$ лежат в разных
элементах $\B$.

\vspace{1em}
{\it Теорема:} Если $\alpha$ --- максимальный по включению элемент $T$, то
\begin{equation}\label{eq:thm_ex_in_class}
  (\forall B\in\B)~\exists a\in\alpha:a\in B
\end{equation}

{\it Доказательство:}
Пусть $\alpha$ --- максимальный по включению элемент $T$ и
$\lnot\eqref{eq:thm_ex_in_class}$, то есть
\[
  \exists B\in\B:(\forall a\in\alpha)~a\notin B
\]
Множество $B$ не пустое, значит существует элемент $b\in B$.
Тогда множество $\alpha\cup \{b\}$ лежит в $T$, что противоречит
максимальности элемента $\alpha$\footnote{Вспомним, что максимальный элемент
  нельзя ``увеличить''.}.
Источник противоречия --- отрицание \eqref{eq:thm_ex_in_class}.\qed

\vspace{1em}
{\it Теорема:} Максимальные по включению элементы $T$ определяют
отображения $f:\B\to S$ такие, что
\[
  (\forall B\in\B)~f(B)\in B
\]
Отображения с этим свойством
называют {\it функциями выбора}\footnote{Их так называют, поскольку
  они ``выбирают'' один элемент из множества.

  Отображение $f:X\to Y$ называют функцией выбора, если
  \[
    (\forall x\in X)~f(x)\in x
  \]
}.

{\it Доказательство:} Пусть $\alpha$ --- максимальный по включению
элемент $T$. По определению $T$ имеем
\[
  (\forall x,y\in\alpha,B\in\B)~x,y\in B\implies x=y
\]
Используя это утверждение и предыдущую теорему получим
\[
  (\forall B\in\B)~\exists !a\in\alpha:a\in B
\]
Тогда отображение $f:\B\to S$, где
\[
  \G(f)=\left\{(B,a)\in \B\times S\;\big|\; a\in\alpha\cap B\right\}
\]
будет обладать необходимым свойством.\qed

Например, пусть ${S=\{a,b,c\}, B_1=\{a,c\}, B_2=\{b\},\B=\{B_1,B_2\}}$,
тогда $\{a,b\}$ и $\{b,c\}$ --- максимальные элементы $T$
(см. рис.~\ref{fig:inc_t}),
они определяют функции выбора:
\[
  f(B_1)=a\quad f(B_2)=b\qquad g(B_1)=c\quad g(B_2)=b
\]

\index{Аксиома!Выбора}
Утверждение, что для каждого множества $X$ существует функция
выбора $f:2^{X}\setminus\{\eset\}\to X$,
называют {\it Аксиомой Выбора}.
Утверждения, что
\begin{enumerate}
  \item{}Функция выбора существует для каждого разбиения.
  \item{}Максимальные элементы $T$ существуют для любого разбиения.
\end{enumerate}
эквивалентны Аксиоме Выбора.

\begin{marginfigure}[-9cm]
  \center
  \begin{tikzpicture}
    \node (e) at (-2,2) {$\eset$};

    \node (a) at (0,3) {$\{a\}$};
    \node (b) at (0,2) {$\{b\}$};
    \node (c) at (0,1) {$\{c\}$};

    \node (ab) at (2,2.5) {$\{a,b\}$};
    \node (bc) at (2,1.5) {$\{b,c\}$};

    \draw [-Latex] (e) -- (a);
    \draw [-Latex] (e) -- (b);
    \draw [-Latex] (e) -- (c);

    \draw [-Latex] (a) -- (ab);
    \draw [-Latex] (b) -- (ab);
    \draw [-Latex] (b) -- (bc);
    \draw [-Latex] (c) -- (bc);
  \end{tikzpicture}

  \caption{Отношение $\subseteq$ на $T$.}\label{fig:inc_t}
\end{marginfigure}

\vspace{1em}
{\it Упражнения:}
\begin{enumerate}
  \item{}Доказать, что $\alpha\in T\implies 2^{\alpha}\subseteq T$.
  \item{}*В каких случаях $T$ содержит наибольший по включению элемент?
  Доказать, что если $\alpha\in T$ --- наибольший по включению элемент $T$,
  то $T=2^{\alpha}$.
  \item{}*Доказать, что для максимальности элемента $\alpha\in T$ необходимо
  и достаточно \eqref{eq:thm_ex_in_class}.
\end{enumerate}

\section{Отношения эквивалентности}

Возьмём бинарное отношение $R$ на множестве $X$. $R$ называют
{\it отношением эквивалентности}\footnote{отношение эквивалентности},
если оно имеет следующие свойства:
\begin{enumerate}
  \item{}$(\forall x\in X)~xRx$ --- {\it рефлексивность}.
  \item{}$(\forall x,y\in X)~xRy\implies yRx$
  --- {\it cимметричность}\index{симметричность}.
  \item{}$(\forall x,y,z\in X)~xRy\land yRz\implies xRz$ --- {\it транзитивность}.
\end{enumerate}
Например, отношение равенства на любом множестве является отношением эквивалентности.

Пусть $R$ --- отношение эквивалентности на $X$ и $x\in X$.
Назовём множество
\[
  [x]:=\left\{y\in X\;\big|\; xRy\right\}
\]
{\it классом эквивалентности} элемента $x$.\index{класс эквивалентности}
Он содержит все элементы, эквивалентные $x$.
По рефлексивности $x \in [x]$.

\vspace{1em}
{\it Теорема:}
Для элементов $x,y\in X$ справедливо
\[
  [x]\cap [y]\neq\eset\implies [x]=[y]
\]

{\it Доказательство:}
\[
  [x]\cap [y]\neq\eset\implies\exists z:z\in[x]\land z\in[y]
\]
Тогда существует $z\in X$ такой, что
\[
  xRz\land yRz\xRightarrow{\text{симметричность}} xRz\land zRy
  \xRightarrow{\text{транзитивность}} xRy
\]
Тогда для произвольного $z\in X$ справедливо
\[
  z\in [y]\iff yRz \xLeftrightarrow{xRy} xRz \iff z\in [x]
\]
и $[x]=[y]$.\qed

Множество всех классов эквивалентности называют
{\it фактормножеством}\index{фактормножество}.
Его можно определить через естественное
расширение отображения $[\cdot]:X\to 2^{X}$
\[
  X/R:=[X]=\left\{\alpha\in 2^{X}\;\big|\; \exists x\in X:\alpha=[x]\right\}
\]
Фактормножество является разбиением множества $X$.

Произвольное разбиение $\B$ множества $X$ также задаёт
отношение эквивалентности. Пусть $f:X\to \B$ --- такое отображение,
что $(\forall x\in X)~x\in f(x)$. Бинарное отношение $R$, где
\[
  xRy\iff f(x)=f(y)
\]
является отношением эквивалентности, причём $X/R=\B$.

Пусть $R_{m}:\Z\to\N_0$ --- функция остатка при делении на $m\in\N$.
% определим её следующим образом:
% \[
%   S_{m}(z):=\{r\in\N\;\big|\; \exists q\in\Z:z=qm+r\}\qquad
%   R_{m}(z):=\min S_{m}(z),
% \]
% где $\min A$ --- наименьший элемент $A$. В данном случае он существует, потому
% что у любого непустого подмножества $\N$ существует наименьший элемент.
% 
% Если бы $R_{m}(z)\geq m$, то $(R_{m}(z)-m)\in S_{m}(z)$,
% что противоречит определению
% наименьшего элемента, значит $0\leq R_{m}(z)< m$.
Определим отношение эквивалентности на $\Z$:
\[
  x=_{m}y\iff R_{m}(x)=R_{m}(y)\qquad \Z_{m}:=\Z/=_{m}
\]
% Это определение эквивалентно
% \[
%   x=_{m}y\iff \exists q\in\Z:x-y=qm
% \]
Найдём такое отображение $(+):\Z^{2}_{m}\to\Z_{m}$, что
\begin{equation}\label{eq:sum_axiom}
  (\forall x,y\in\Z)~[x]+[y]=[x+y]
\end{equation}
Заметим, что первое использование $(+)$ --- отображение $\Z_{m}^{2}\to \Z_{m}$,
а второе --- отображение $\Z^{2}\to\Z$.

Пусть $f:Z_{m}\to \Z$ --- функция выбора, тогда сложение можно
определить как
\[
  x+y:=[f(x)+f(y)]
\]

\vspace{1em}
{\it Теорема:}
Такое сложение удовлетворяет \eqref{eq:sum_axiom}.

{\it Доказательство:}
Пусть $x,y\in \Z$ и
\[
  x':=f([x])\quad x'=_{m}x\qquad y':=f([y])\quad y'=_{m}y
\]
Заметим, что
\[
  a=_{m}b\iff \exists q\in\Z:b-a=qm
\]
Тогда
\[
  \begin{aligned}
     & x'=_{m}x\land y'=_{m}y\implies                                 \\
     & \implies \exists q_{x}\in\Z:x'-x=q_{x}m\land
    \exists q_{y}\in\Z:y'-y=q_{y}m\implies                            \\
     & \implies \exists q\in\Z:(x'+y')-(x+y)=qm\implies x'+y'=_{m}x+y
  \end{aligned}
\]
\[
  [x]+[y]=[x'+y']=[x+y]\qed
\]

\vspace{1em}
{\it Теорема:} Пусть $+,+':\Z_{m}^{2}\to\Z_{m}$ --- отображения,
удовлетворяющие \eqref{eq:sum_axiom}. Тогда
\[
  (\forall x,y\in\Z_{m})~x+y=x+'y,
\]
то есть $+$ и $+'$ равны\footnote{Пусть $f_1:X_1\to Y_1$ и $f_2:X_2\to Y_2$.
  Тогда для $f_1=f_2$ необходимо и достаточно
  \[
    X_1=X_2\qquad Y_1=Y_2
  \]
  \[
    (\forall x\in X_1)~f_1(x)=f_2(x)
  \]
}.

{\it Доказательство:} Возьмём $x,y\in\Z_{m}$. Они не пустые, значит
существуют $x'\in x,y'\in y$, тогда
\[
  x+y=[x']+[y']=[x'+y']=[x']+'[y']=x+'y\qed
\]

Осталось только определить функцию выбора. Это можно сделать либо через
Аксиому Выбора, либо $f(\alpha):=\min(\alpha\cap\N)$, причём такая функция
выбора будет обладать свойством
\[
  (\forall a\in\Z)~f([a])=R_{m}(a)
\]

Заметим, что из \eqref{eq:sum_axiom} следует существование
и единственность сложения, поэтому можно сказать,
что эта формула является определением сложения на $\Z_{m}^{2}$.

\index{расширение}
Также говорят, что мы {\it расширили} $+$ на $\Z_{m}^{2}$,
а $(+):\Z_{m}^{2}\to\Z_{m}$ называют {\it расширением} $(+):\Z^{2}\to\Z$.
Аналогично естественное расширение также называют расширением.

\vspace{1em}
{\it Упражнения:}
\begin{enumerate}
  \item{}Пусть $f:X\to T$. Доказать, что отношение\label{ex:fn_equiv}
  \[
    aRb\iff f(a)=f(b)
  \]
  является отношением эквивалентности на $X$.

  \item{}Доказать, что следующие отношения являются отношениями эквивалентности:
  \begin{enumerate}
    \item{}$=$ на $X$, где $X$ --- произвольное.
    \item{}Отношение параллельности прямых на множестве прямых.
    \item{}Отношение подобия треугольников на множестве треугольников
    на плоскости.
    \item{}Отношение равенства длины на множестве отрезков.
  \end{enumerate}

  \item{}Пусть $R$ --- отношение эквивалентности на множестве $X$.
  Пусть $(\circ):X^{2}\to X$, причём выполняется
  \[
    (\forall x,x',y,y'\in X)~xRx'\land yRy'\implies (x\circ y)R(x'\circ y')
  \]

  Доказать, опираясь на Аксиому Выбора,
  что существует единственное расширение $\circ$ на $X/R$, удовлетворяющее
  \[
    (\forall x,y\in X)~[x]\circ[y]=[x\circ y]
  \]

  \item{}*Пусть $X$ --- произвольное.
  Определим отношение $\leq_{\mu}$ на $2^{X}$:
  \[
    \beta\leq_{\mu}\alpha\iff \exists f:(f:\alpha\to\beta)\land f(\alpha)=\beta
  \]
  Доказать, что отношение
  \[
    \alpha R\beta\iff (\alpha\leq_{\mu}\beta)\land (\beta\leq_{\mu}\alpha)
  \]
  является отношением эквивалентности на $2^{X}$.

  \index{индуктивное множество}\index{множество!индуктивное}
  \index{достижимость}\index{элемент!достижимый}
  \item{}*Пусть $\sigma(x):=x\cup\{x\}$ и $0:=\eset$. Будем говорить, что
  множество $M$ {\it индуктивно} и писать $I(M)$, если
  \[
    I(M)\equiv 0\in M\land (\forall x)~x\in M\implies \sigma(x)\in M
  \]
  Будем говорить, что $a$ {\it достижим} и писать $S(a)$,
  если он содержится во всех индуктивных множествах, то есть
  \[
    S(a)\equiv \forall M(I(M)\implies a\in M)
  \]

  Пусть $P$ --- произвольное индуктивное множество\footnote{
    Аксиома Бесконечности:\index{Аксиома!Бесконечности}
    \[
      \exists M:I(M)
    \]}.
  Определим множество натуральных чисел с $0$ как
  \[
    \N_0:=\left\{a\in P\;\big|\; S(a)\right\}
  \]
  \begin{enumerate}
    \item{}Пусть $P'$ --- другое индуктивное множество, которое может
    отличаться от $P$ и $\N_0':=\left\{a\in P'\;\big|\; S(a)\right\}$.
    Доказать $\N_0=\N_0'$.
    \item{}Доказать, что $\N_0$ --- наименьшее индуктивное множество.
    В этом случае достаточно доказать утверждение
    \[
      (\forall M)~I(M)\implies\N_0\subseteq M
    \]
    \item{}Доказать принцип математической индукции\footnote{Возьмём формулу $P(x)$,
    для которой доказано
    \[
      P(0)\land (\forall n\in\N_0)~[P(n)\implies P(\sigma(n))]
    \]
    Тогда множество
    \[
      M:=\left\{n\in\N_0\;\big|\; P(n)\right\}
    \]
    удовлетворяет условию упражнения и $\N_0\subseteq M$,
    тогда
    \[
      (\forall n\in\N_0)~P(n)
    \]}:
    \index{индукция}
    \[
      [0\in M\land (\forall n\in\N_0)~(n\in M\implies
          \sigma(n)\in M)]\implies \N_0\subseteq M
    \]
    \item{}Пусть отображения $+,+':\N_0^{2}\to\N_0$ оба удовлетворяют
    \begin{enumerate}
      \item{}$(\forall a\in\N_0)~a+0=a$
      \item{}$(\forall a,b\in\N_0)~a+\sigma(b)=\sigma(a+b)$
    \end{enumerate}
    Доказать, что они равны. Подсказка:
    использовать индукцию с формулой
    \[
      P(n)\equiv (\forall a\in\N_0)~a+n=a+'n
    \]
    \item{}Пусть отображения $\cdot,\cdot':\N_0^{2}\to\N_0$ оба удовлетворяют
    \begin{enumerate}
      \item{}$(\forall a\in\N_0)~a\cdot 0=0$
      \item{}$(\forall a,b\in\N_0)~a\cdot\sigma(b)=a\cdot b+a$
    \end{enumerate}
    Доказать, что они равны.
  \end{enumerate}

  \item{}*Пусть $a\in S$ и $f:S\to S$. Дать определение множеству
  \[
    V=\{a,f(a),f(f(a)),f(f(f(a))),...\}
  \]
  Сколько в нём элементов, если $f(a)=a$? $f(f(a))=a$?
  В каких случаях оно бесконечно?
\end{enumerate}
