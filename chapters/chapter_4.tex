\part{Бинарные отношения}

Рассмотрим множества $X,Y$ и $R\subseteq X\times Y$. Множество $R$ можно рассматривать
как определённое отношение между элементами множеств $X$ и $Y$. Можем ввести бинарный
предикатный знак
\[
	\forall x\forall y(xRy\iff (x,y)\in R)
\]

Обычно предикатный знак и множество $R$ обозначают одинаково.
$R$ называют {\it бинарным отношением между}
\index{бинарное отношение}
\index{бинарное отношение!между} $X$ и $Y$. Если $X=Y$,
то $R$ называют {\it бинарным отношением на}\index{бинарное отношение!на} $X$.

Например, рассмотрим множество $S=2^{\{a,b\}}$.
\[
	S=\{\eset,\{a\},\{b\},\{a,b\}\}
\]
Пусть $R\subseteq S^{2}$ --- отношение включения ($\subseteq$) на $S$, тогда
\[
	\begin{aligned}
		R
		 & =\left\{(x,y)\in S^{2}\;\big|\; x\subseteq y\right\}= \\
		 & =\left\{
		\begin{array}{cccc}
			(\eset,\eset), & (\eset,\{a\}),   & (\eset,\{b\}),   & (\eset,\{a,b\}),  \\
			(\{a\},\{a\}), & (\{a\},\{a,b\}), & (\{b\},\{a,b\}), & (\{a,b\},\{a,b\})
		\end{array}\right\}
	\end{aligned}
\]

\section{Отображения}

Рассмотрим бинарное отношение $f\subseteq X\times Y$ со следующим свойством:
\[
	\forall x\in X~\exists! y\in Y:(x,y)\in f
\]
Можем ввести унарный функциональный знак аксиомой
\[
	(\forall x\in X)~(x,f(x))\in f
\]
Заметим, что из $(x,f(x))\in f$ следует $f(x)\in Y$.

Обычно функциональный знак и бинарное отношение $f$ обозначают одинаково.
Такое $f$ называют {\it отображением}\index{отображение}
из {\it области определения}\index{область!определения} $X$ в
{\it область значений}\index{область!значений} $Y$ и пишут $f:X\to Y$.
Отображения также называют функциями\index{функция}.

Например, если $f:\Z\to\Z$ определено как $f(x)=2x$, то
\[
	f=\left\{(0,0),(1,2),(-1,-2),(2,4),(-2,-4),...\right\}
\]

Для отображений вида $g:X_1\times ...\times X_{n}\to Y$
введём также $n$-арные функциональные знаки аксиомами
\[
	(\forall x_1\in X_1)...(\forall x_{n}\in X_{n})~
	((x_1,...,x_{n}), g(x_1,...,x_{n}))\in g
\]
Такие отображения называют {\it $n$-арными}\index{отображение!$n$-арное}.

Возьмём ${f:X\to Y}$. Если ${y=f(x)}$, то $y$ называют {\it значением $f$ в $x$}
\index{значение отображения} или {\it образом $x$ по $f$}\index{образ},
а $x$ --- {\it прообразом $y$ по $f$}\index{прообраз}.

{\it Образом множества} $A\subseteq X$\index{образ!множества}
называют множество
\[
	\widetilde f(A):=\left\{y\in Y\;\big|\; \exists x\in A:y=f(x)\right\},
\]
то есть в $\widetilde f(A)$ лежат образы элементов $A$. Таким образом, мы определили
отображение $\widetilde f:2^{X}\to 2^{Y}$ ---
{\it естественное расширение функции}\index{естественное расширение}
$f$, обычно его обозначают
тем же знаком\footnote{Мы можем себе это позволить, потому что
	в большинстве случаев из контекста понятно, что имеется
	в виду под знаком $f$.}.

{\it Прообразом множества}\index{прообраз!множества}
$B\subseteq Y$ называют множество
\[
	f^{-1}(B):=\left\{x\in A\;\big|\; f(x)\in B\right\},
\]
то есть в $f^{-1}(B)$ лежат все прообразы элементов $B$.

Также определяют естественные расширения бинарных отображений.
Пусть $(\circ):X\times Y\to Z$ --- произвольное бинарное отображение, тогда
\[
	a\in X,~ b\in Y,~A\subseteq X,~ B\subseteq Y
\]
\[
	a\circ B:=\left\{z\in Z\;\big|\;\exists b\in B:z=a\circ b\right\}
\]
\[
	A\circ b:=\left\{z\in Z\;\big|\;\exists a\in A:z=a\circ b\right\}
\]
\[
	A\circ B:=\left\{z\in Z\;\big|\;\exists a\in A,b\in B:z=a\circ b\right\}
\]

Например, пусть $f:\Z\to\Z$ и $f(x)=|x|$, тогда
\[
	f(\{0,-1,1,2\})=\{0,1,2\}\qquad f^{-1}(\{0,1,2\})=\{0,-1,1,-2,2\}
\]
\[
	\{0,1,2\}+\{10,20\}=\{10,20,11,21,12,22\}
\]

Прообразы и образы отображений --- одни из основных способов выделения
подмножеств множеств $X$ и $Y$.

Будем говорить, что отображение $f:X\to Y$ {\it сюръективно}
\index{сюръективность}\index{отображение!сюръективное}
или является {\it сюръекцией}\index{сюръекция}, если $f(X)=Y$,
то есть у каждого $y\in Y$ существует (не обязательно единственный) прообраз.

Рассмотрим множество $U$ и сюръекцию ${u:I\to U}$,
введём обозначения
\[
	u_{i}\equiv u(i)\qquad \{u_{i}\}_{i\in I}\equiv u(I)=U,
\]
тогда $U$ называют {\it индексированным семейством}\index{индексированное семейство},
$I$ --- {\it множеством индексов}\index{множество!индексов},
$u_{i}$ --- {\it общим членом}\index{общий член},
$i$ --- его {\it индексом}\index{индекс}.

Индексированное семейство определяется множеством индексов $I$ и отображением
${u:I\to U'}$, тогда $U=u(I)\subseteq U'$,
причём каждому $a\in U$ соответствует хотя бы один индекс $i\in I$.

Например, пусть $I=\Z$ и $a:I\to \Z$, $a(n)=2n$, тогда
\[
	a_{n}\equiv a(n)=2n\qquad A:=\{a_{n}\}_{n\in\Z}=\{2n\}_{n\in\Z}=a(\Z)
\]
Заметим, что $a$ не является сюръекцией и $A\neq \Z$.

Часто общий член и семейство обозначают одним знаком.
Объединение и пересечение множеств из семейства $\{U_{i}\}_{i\in I}$
обычно записывают так:
\[
	\bigcup_{i\in I}U_{i}\equiv\cup \{U_{i}\}_{i\in I}\qquad
	\bigcap_{i\in I}U_{i}\equiv\cap \{U_{i}\}_{i\in I}
\]

Если $I=\N$, то принимают обозначения:
\[
	\{U_{k}\}_{k=1}^{\infty}\equiv\{U_{k}\}_{k\in\N}\qquad
	\bigcup_{k=1}^{\infty}U_{k}\equiv\bigcup_{k\in\N}U_{k}\qquad
	\bigcap_{k=1}^{\infty}U_{k}\equiv\bigcap_{k\in\N}U_{k}
\]

Если $I=\{1,2,...,n\}=\left\{m\in\N\;\big|\;m\leq n\right\},n\in\N$,
то принимают обозначения:
\[
	\{U_{k}\}_{k=1}^{n}\equiv\{U_{k}\}_{k\in I}\qquad
	\bigcup_{k=1}^{n}U_{k}\equiv\bigcup_{k\in I}U_{k}\qquad
	\bigcap_{k=1}^{n}U_{k}\equiv\bigcap_{k\in I}U_{k}
\]
Заметим также, что
\[
	\bigcup_{k=1}^{n}U_{k}=U_1\cup U_2\cup ...\cup U_{n}\qquad
	\bigcap_{k=1}^{n}U_{k}=U_1\cap U_2\cap ...\cap U_{n}
\]

В формулах выше термы индексов $i$ и $k$ также называют связанными переменными.

% Вспомним определение натуральных чисел из множеств.
% Множество $X$ называют {\it конечным}, если
% \[
%   \exists n\in\N,f:(f:n\to X)\land f(n)=X,
% \]
% то есть когда $X$ можно ``пронумеровать конечным количеством натуральных чисел''.

\vspace{1em}
{\it Упражнения:}
\begin{enumerate}
	\item{}Почему $y\in f(A)$ ттк $y$ является образом элемента из $A$?
	\item{}Пусть ${f:X\to Y}$, ${A_{X},B_{X}\subseteq X}$ и
	${A_{Y},B_{Y}\in Y}$\footnote{Запись $a_1,a_2,...,a_{n}\in A$ означает
	$a_1\in A$, $a_2\in A$,...,$a_{n}\in A$}.
	Доказать следующие утверждения:
	% \begin{fullwidth}
	%   \begin{multicols}{2}
	\begin{enumerate}
		\item{}$A_{X}\subseteq B_{X}\implies f(A_{X})\subseteq f(B_{X})$
		\item{}$A_{Y}\subseteq B_{Y}\implies f^{-1}(A_{Y})\subseteq f^{-1}(B_{Y})$
		\item{}$A_{X}\subseteq f^{-1}(f(A_{X}))$
		\item{}$A_{Y}=f(f^{-1}(A_{Y}))$
		\item{}$f^{-1}(f(f^{-1}(f(A_{X}))))=f^{-1}(f(A_{X}))$
		\item{}$f(A_{X})\cap f(B_{X})=\eset\implies A\cap B=\eset$
		\item{}$A_{Y}\cap B_{Y}=\eset\implies f^{-1}(A_{Y})\cap f^{-1}(B_{Y})=\eset$
	\end{enumerate}
	%   \end{multicols}
	% \end{fullwidth}
	\item{}*Могут ли существовать отображения
	\begin{multicols}{3}
		\begin{enumerate}
			\item{}$f:\eset\to A$
			\item{}$f:A\to\eset$
			\item{}$f:\eset\to\eset$
		\end{enumerate}
	\end{multicols}
\end{enumerate}

\section{Частичный порядок}

\begin{marginfigure}[1cm]
	\center
	\begin{tikzpicture}
		\node (e) at (-1.5,2) {$\eset$};
		\node (a) at (0,3) {$\{a\}$};
		\node (b) at (0,1) {$\{b\}$};
		\node (ab) at (2,1) {$\{a,b\}$};
		\node (abc) at (2,3) {$\{a,b,c\}$};

		\draw [-Latex] (e) -- (a);
		\draw [-Latex] (e) -- (b);
		\draw [-Latex] (e) -- (ab);
		\draw [-Latex] (e) -- (abc);

		\draw [-Latex] (a) -- (ab);
		\draw [-Latex] (a) -- (abc);

		\draw [-Latex] (b) -- (ab);
		\draw [-Latex] (b) -- (abc);

		\draw [-Latex] (ab) -- (abc);
	\end{tikzpicture}

	\caption{Диаграмма отношения включения.}\label{fig:inc_diag}
\end{marginfigure}

\begin{marginfigure}
	\center
	\begin{tikzpicture}
		\node (e) at (-1.5,2) {$\eset$};
		\node (a) at (0,3) {$\{a\}$};
		\node (b) at (0,1) {$\{b\}$};
		\node (ab) at (2,1) {$\{a,b\}$};
		\node (abc) at (2,3) {$\{a,b,c\}$};

		\draw [-Latex] (e) -- (a);
		\draw [-Latex] (e) -- (b);

		\draw [-Latex] (a) -- (ab);

		\draw [-Latex] (b) -- (ab);

		\draw [-Latex] (ab) -- (abc);
	\end{tikzpicture}

	\caption{Сокращённая диаграмма отношения включения.}\label{fig:inc_diag_short}
\end{marginfigure}

Возьмём бинарное отношение $R$ на множестве $X$. $R$ называют
{\it частичным порядком}\index{частичный порядок}\index{порядок!частичный},
если оно имеет следующие свойства:
\begin{enumerate}
	\item{}$(\forall x\in X)~xRx$ --- {\it рефлексивность}\index{рефлексивность}.
	\item{}$(\forall x,y\in X)~xRy\land yRx\implies x=y$
	--- {\it антиcимметричность}\index{антиcимметричность}.
	\item{}$(\forall x,y,z\in X)~xRy\land yRz\implies xRz$
	--- {\it транзитивность}\index{транзитивность}.
\end{enumerate}
Например, отношение $\leq$ на множестве $\Z$ является отношением частичного порядка.
Для произвольного множества $S$ отношение включения является
отношением частичного порядка на $2^{S}$.

Благодаря транзитивности бинарные отношения можно изображать в виде диаграмм, где
различные $x$ и $y$ соединены путём по направлению стрелок ттк $xRy$.
Это позволяет опустить стрелку $xRy$, если существует такой $z$, что $xRz$ и $zRy$.
Примеры диаграмм отношения включения на множестве
$\{\eset,\{a\},\{b\},\{a,b\}, \{a,b,c\}\}$
можно видеть на рис.~\ref{fig:inc_diag},~\ref{fig:inc_diag_short}.

Если $xRy$ или $yRx$, то элементы $x$ и $y$ называются
{\it сравнимыми}\index{сравнимые элементы},
а пара $(x,y)$ --- {\it сравнимой}\index{сравнимая пара}.
Отношение частичного порядка называют отношением {\it линейного порядка}
\index{порядок!линейный}\index{линейный порядок}, если также выполняется
\begin{enumerate}[resume*]
	\item{}$(\forall x,y\in X)~xRy\lor yRx$ --- {\it полнота}\index{полнота}.
\end{enumerate}
То есть любая пара элементов сравнима. Отношение $\leq$ на $\Z$
является отношением линейного порядка,
но отношение включения на $2^{\{a,b\}}$ таковым не является,
потому что $\{a\}$ и $\{b\}$ несравнимы.

Пусть $\preceq$ --- произвольный частичный порядок на $X$.
Элемент $x_0\in X$ называют {\it минимальным}
\index{элемент!минимальный}\index{минимальный элемент}
по $\preceq$, если
\[
	(\forall x\in X)~x\npreceq x_0
\]

Элемент $x_0\in X$ называют {\it наименьшим}
\index{элемент!наименьший}\index{наименьший элемент} по $\preceq$, если
\[
	(\forall x\in X)~x_0\preceq x
\]
Другими словами, \textsc{элемент является наименьшим,
	если он меньше всех элементов и является минимальным, если его нельзя
	``уменьшить''.}

Аналогично определяются {\it максимальные} и {\it наибольшие} элементы.
Заметим, что из наименьшести (наибольшести) следует минимальность (максимальность).
\index{элемент!максимальный}\index{максимальный элемент}
\index{элемент!наибольший}\index{наибольший элемент}

\begin{marginfigure}
	\center
	\begin{tikzpicture}
		\node (a) at (0,0) {$0$};
		\node (b) at (1,0) {$1$};
		\node (c) at (2,0) {$2$};

		\draw [-Latex] (a) -- (b);
		\draw [-Latex] (b) -- (c);
	\end{tikzpicture}

	\caption{$\leq$ на $\{0,1,2\}$}\label{fig:less_higher}
\end{marginfigure}

\begin{marginfigure}
	\center
	\begin{tikzpicture}
		\node (a) at (0,1.5) {$\{a\}$};
		\node (b) at (0,0.5) {$\{b\}$};
		\node (ab) at (2,1) {$\{a,b\}$};

		\draw [-Latex] (a) -- (ab);
		\draw [-Latex] (b) -- (ab);
	\end{tikzpicture}

	\caption{$\subseteq$ на $\{\{a\},\{b\},\{a,b\}\}$}\label{fig:min_max}
\end{marginfigure}

Таким образом, множество $\{0,1,2\}$ содержит минимальные и максимальные по $\leq$
элементы $0$ и $1$, они же являются наименьшим и
наибольшим элементом соответственно (см.~рис.~\ref{fig:less_higher}).

Множество $\{\{a\},\{b\},\{a,b\}\}$ содержит
два минимальных: $\{a\}$ и $\{b\}$ и один максимальный
по включению элемент: $\{a,b\}$. Причём элемент $\{a,b\}$ является наибольшим, а
наименьшего элемента не существует (см.~рис.~\ref{fig:min_max}).

\pagebreak
% \vspace{1em}
{\it Теорема:} Пусть $A,B\subseteq S$. Тогда множество $A\cap B$ ---
наибольшее по включению подмножество $S$, содержащееся\footnote{Если $T\subseteq V$,
	то говорят, что $T$ {\it содержится}\index{содержаться, $\subseteq$}
	в $V$} в них обоих,
а множество $A\cup B$ --- наименьшее по включению подмножество $S$,
содержащее их оба.

{\it Доказательство:}
\[
	S_{*}:=\left\{X\in 2^{S}\;\big|\; X\subseteq A\land X\subseteq B\right\}\qquad
	S^{*}:=\left\{X\in 2^{S}\;\big|\; A\subseteq X\land B\subseteq X\right\}
\]

Нужно доказать, что множество $A\cap B$ --- наибольший по включению элемент $S_{*}$,
а $A\cup B$ --- наименьший элемент $S^{*}$.
Очевидно, $A\cap B\in S_{*}$ и $A\cup B\in S^{*}$.

Пусть $Z\in S_{*}$ и $x\in Z$, тогда
\[
	x\in Z\implies x\in A\land x\in B\implies x\in A\cap B,
\]
что справедливо для любого $x$,
значит ${(\forall Z\in S_{*})~Z\subseteq A\cap B}$. Тогда
$A\cap B$ является наибольшим элементом $S_{*}$.
Аналогично доказывается, что $A\cup B$ --- наименьший элемент $S^{*}$.\qed

\newcommand\Q{\mathbb Q}
\vspace{1em}
{\it Упражнения:}
\begin{enumerate}
	\item{}Доказать, что $A\cup B$ --- наименьший элемент $S^{*}$.
	\item{}Доказать, что $\leq$ на $\Z$ и $\subseteq$ на $2^{S}$ --- частичные порядки.
	\item{}Пусть $S$ --- произвольное множество.
	Найти максимальные, минимальные, наибольшие и
	наименьшие по включению элементы множеств
	\[
		2^{S}\qquad 2^{S}\setminus\{\eset\}\qquad
		2^{S}\setminus \{S\}
	\]
	Нарисовать диаграммы для случая $S=\{a,b,c\}$.
	\item{}*Доказать, что если наименьший элемент существует, то
	он единственен.
	\item{}*Пусть $\leq$ --- порядок (не частичный) на $X$, а $x_0\in X$ --- минимальный
	элемент. Доказать, что $x_0$ --- наименьший элемент.
	\item{}*Пусть $\Q$ --- множество рациональных чисел\footnote{
		Множество рациональных чисел\index{множество!рациональных чисел, $\Q$}
		--- множество всех дробей вида $a/b$, где
		\[
			a\in\Z,\quad b\in\N
		\]
		и для любых ${b'\in\N}$, ${a',p\in\Z}$ справедливо
		\[
			a=a'p\land b=b'p\implies p=1,
		\]
		где последнее условие обеспечивает простоту дроби
		и единственность представления $a/b$.}.
	Пусть $A\subseteq \Q$ и множество
	\[
		M:=\left\{x\in \Q\;\big|\;(\forall a\in A)~a\leq x\right\}
	\]
	не пустое. Существует ли в $M$ наименьший по $\leq$ элемент
	в следующих случаях:
	\begin{multicols}{2}
		\begin{enumerate}
			\item{}$A=[a,b]\cap\Q$, где $a,b\in\Q$
			\item{}$A=\{a_1,a_2,...,a_{n}\}$
			\item{}$A=\eset$
			\item{}$A=\left\{x\in\Q\;\big|\; x^{2}<2\right\}$
		\end{enumerate}
	\end{multicols}
\end{enumerate}

\section{Замыкания и разбиения}

Возьмём множество $A$ c произвольным частичным порядком $\leq$. Отображение
$\Gamma:A\to A$ называют {\it замыканием на $A$ с $\leq$}\index{замыкание}, если
\begin{enumerate}
	\item{}$(\forall a\in A)~a\leq\Gamma(a)$
	\item{}$(\forall a,b\in A)~a\leq b\implies \Gamma(a)\leq \Gamma(b)$
	\item{}$(\forall a\in A)~\Gamma(\Gamma(a))=\Gamma(a)$
\end{enumerate}
Элемент ${a\in A}$ называют {\it замкнутым по $\Gamma$}\index{замкнутость},
если ${\Gamma(a)=a}$.

Например, рассмотрим $I$ --- множество всех интервалов $(a,b)$, $[a,b]$, $(a,b]$
и $[a,b)$, где $a,b$ --- конечные числа, с частичным порядком $\subseteq$.
Тогда отображение $\Gamma:I\to I$, определённое как
\[
	\Gamma((a,b)):=[a,b]\qquad \Gamma([a,b]):=[a,b]\qquad
	\Gamma((a,b]):=[a,b]\qquad \Gamma([a,b)):=[a,b]
\]
является замыканием, а $[a,b]$ --- замкнутые по $\Gamma$ элементы $I$.

\vspace{1em}
{\it Теорема:} Пусть $a\in A$. Тогда $\Gamma(a)$ --- наименьший по $\leq$
замкнутый по $\Gamma$ элемент $A$,
больший (содержащий\footnote{В зависимости от отношения $\leq$ формулу
	$a\leq b$ также иногда читают как ``$b$ содержит $a$''.}) $a$.

	{\it Доказательство:}
\[
	\gamma:=\left\{q\in A\;\big|\; \Gamma(q)=q\land a\leq q\right\}
\]
Нужно доказать, что $\Gamma(a)$ --- наименьший элемент $\gamma$.
\[
	\Gamma(\Gamma(a))=\Gamma(a)\land a\leq\Gamma(a)\implies\Gamma(a)\in\gamma
\]

Пусть $q\in \gamma$, тогда
\[
	a\leq q\implies \Gamma(a)\leq\Gamma(q)=q\implies\Gamma(a)\leq q
\]
и $\Gamma(a)$ --- минимальный элемент $\gamma$.\qed

\newcommand\R{\mathbb R}
\vspace{1em}
{\it Теорема:}
Будем говорить, что ${a\in A}$ замкнут, если $P(a)$\footnote{В данном случае
$P(a)$ может быть любой формулой о $a$.

В примере она примет вид
\[
	P(I)\equiv \exists a,b\in\R:I=[a,b],
\]
где $\R$ --- множество действительных чисел.}.
Пусть ${\Gamma':A\to A}$ --- такое отображение, что $\Gamma'(a)$ --- наименьший по
$\leq$ замкнутый элемент $A$, содержащий $a$. Тогда $\Gamma'$ --- замыкание,
причём элемент замкнут ттк он замкнут по $\Gamma'$.

	{\it Доказательство:}
Проверим свойства замыкания.
\begin{enumerate}
	\item{}$(\forall a\in A)~a\leq \Gamma'(a)$ по определению $\Gamma'$.

	\item{}Пусть $a,b\in A$ и $a\leq b$. Элемент $\Gamma'(b)$ замкнут и
	$a\leq b\leq\Gamma'(b)$, значит $a\leq\Gamma'(b)$. Тогда, по наименьшести,
	$\Gamma'(a)\leq\Gamma'(b)$.

	\item{}$a$ --- наименьший элемент, содержащий $a$, поэтому
	если $P(a)$, то $\Gamma'(a)=a$. Если $\Gamma'(a)=a$, то $P(a)$
	по определению $\Gamma'$.

	\item{}$\Gamma'(a)$ замкнут, поэтому $\Gamma'(\Gamma'(a))=\Gamma'(a)$.\qed
\end{enumerate}

\newcommand\B{\mathcal B}
Возьмём множество $S$ и представим его как объединение семейства непустых
попарно непересекающихся множеств $\B:=\{B_{i}\}_{i\in I}$.
\[
	S=\bigcup_{i\in I}B_{i}\qquad
	(\forall i,j\in I:i\neq j)~B_{i}\cap B_{j}=\eset\qquad
	(\forall i\in I)~B_{i}\neq\eset
\]
Такое представление называется {\it разбиением}\index{разбиение}.
Определим следующее множество:
\[
	T:=\left\{\alpha\in 2^{S}\;\big|\; (\forall x,y\in\alpha,i,j\in I)~
	x\neq y\land	x\in B_{i}\land y\in B_{j}\implies i\neq j\right\},
\]
то есть все элементы множества $\alpha\in T$ содержатся в разных $B_{i}$.

\vspace{1em}
{\it Теорема:} Если $\alpha$ --- максимальный по включению элемент $T$, то
\begin{equation}\label{eq:thm_ex_in_class}
	(\forall i\in I)~\exists a\in\alpha:a\in B_{i}
\end{equation}

{\it Доказательство:}
Пусть $\alpha$ --- максимальный по включению элемент $T$ и
$\lnot\eqref{eq:thm_ex_in_class}$, то есть
\[
	\exists i\in I:(\forall a\in\alpha)~a\notin B_{i}
\]

Множество $B_{i}$ не пустое, значит существует элемент $b\in B_{i}$.
Тогда множество $\alpha\cup \{b\}$ лежит в $T$, что противоречит
максимальности $\alpha$\footnote{Вспомним, что максимальный элемент
	нельзя ``увеличить''.}.
Источник противоречия --- отрицание \eqref{eq:thm_ex_in_class}.\qed

\vspace{1em}
{\it Теорема:} Максимальные по включению элементы $T$ определяют
отображения $f:\B\to S$ такие, что
\[
	(\forall b\in\B)~f(b)\in b
\]
Отображения со свойством $f(b)\in b$ называют {\it функциями выбора}.

{\it Доказательство:} Пусть $\alpha$ --- максимальный по включению
элемент $T$.
По определению $T$ (если принять $i=j$)
\[
	(\forall i\in I,a,b\in\alpha)~i=i\implies a=b\lor a\notin B_{i}\lor b\notin B_{i}
\]
\[
	(\forall i\in I,a,b\in\alpha)~a,b\in B_{i}\implies a=b
\]

Используя полученное утверждение и предыдущую теорему, получаем
\[
	(\forall i\in I)~\exists !a\in\alpha:a\in B_{i}
\]

Тогда через следующее бинарное отношение $f$ между $\B$ и $S$
\[
	bfa\iff a\in \alpha\land a\in b,\quad b\in\B,~a\in S
\]
можем определить необходимое отображение $f:\B\to S$.\qed

Утверждение, что для каждого множества $X$ существует функция
выбора $f:2^{X}\setminus\{\eset\}\to X$,
называется {\it Аксиомой Выбора}\index{Аксиома!Выбора}.

Утверждение, что функция выбора существует для каждого разбиения,
эквивалентно Аксиоме Выбора.

Существование максимальных элементов $T$ для любого
разбиения также эквивалентно Аксиоме Выбора.

Например, пусть $S=\{a,b,c\}$, $B_1=\{a,c\}$ и $B_2=\{b\}$, тогда
множества $\{a,b\}$ и $\{b,c\}$ --- максимальные элементы $T$.
Наглядно это можно увидеть на рис. $\ref{fig:inc_t}$.

Они определяют две функции выбора:
\[
	f(B_1)=a\quad f(B_2)=b\qquad g(B_1)=c\quad g(B_2)=b
\]

\begin{marginfigure}
	\center
	\begin{tikzpicture}
		\node (e) at (-2,2) {$\eset$};

		\node (a) at (0,3) {$\{a\}$};
		\node (b) at (0,2) {$\{b\}$};
		\node (c) at (0,1) {$\{c\}$};

		\node (ab) at (2,2.5) {$\{a,b\}$};
		\node (bc) at (2,1.5) {$\{b,c\}$};

		\draw [-Latex] (e) -- (a);
		\draw [-Latex] (e) -- (b);
		\draw [-Latex] (e) -- (c);

		\draw [-Latex] (a) -- (ab);
		\draw [-Latex] (b) -- (ab);
		\draw [-Latex] (b) -- (bc);
		\draw [-Latex] (c) -- (bc);
	\end{tikzpicture}

	\caption{Отношение $\subseteq$ на $T$.}\label{fig:inc_t}
\end{marginfigure}

\vspace{1em}
{\it Упражнения:}
\begin{enumerate}
	\item{}Пусть $\Gamma:A\to A$ --- замыкание на $A$ с $\leq$ и $a$ ---
	максимальный по $\leq$ элемент $A$. Доказать, что $a$ замкнут по $\Gamma$.
	\item{}Пусть $f:X\to Y$. Доказать, что отображение
	\[
		\Gamma:2^{X}\to 2^{X},\quad \Gamma(A):= f^{-1}(f(A))
	\]
	является замыканием на $2^{X}$ с $\subseteq$. Какие множества
	будут замкнутыми, если $X=\Z,f:\Z\to \Z,f(x)=|x|$? Если $f(x)=x^{2}$?
	\item{}Пусть $A$ --- множество с частичным порядком $\leq$.
	Доказать, что $a\in A$ --- наибольший элемент, меньший $a$,
	и наименьший элемент, больший $a$.
	\item{}Доказать, что $\alpha\in T\implies 2^{\alpha}\subseteq T$.
	\item{}*В каких случаях $T$ содержит наибольший по включению элемент?
	Доказать, что если $\alpha\in T$ --- наибольший по включению элемент $T$,
	то $T=2^{\alpha}$.
	\item{}*Доказать, что для максимальности элемента $\alpha\in T$ необходимо
	и достаточно
	\[
		(\forall i\in I)~\exists a\in\alpha:a\in B_{i}
	\]
	\item{}*Возьмём сюръекцию $f:S\to K$. Пусть
	\[
		B_{k}:=f^{-1}(\{k\}),\quad \B:=\{B_{k}\}_{k\in K}
	\]
	\begin{enumerate}
		\item{}Доказать, что $\B$ --- разбиение $S$.
		\item{}Пусть $h:\B\to S$ --- функция выбора и
		\[
			g:K\to S,\qquad g(k):=h(B_{k})
		\]
		Какое отношение между $g$ и $f$?
		\item{}Найти $f(g(k))$ для произвольного $k\in K$.
		\item{}Найти $g(f(s))$ для произвольного $s\in S$ при условии, что
		\[
			(\forall k\in K)(\forall a,b)~a,b\in B_{k}\implies a=b
		\]
	\end{enumerate}
\end{enumerate}

\section{Отношение эквивалентности}

Возьмём бинарное отношение $R$ на множестве $X$. $R$ называют
{\it отношением эквивалентности}\footnote{отношение эквивалентности},
если оно имеет следующие свойства:
\begin{enumerate}
	\item{}$(\forall x\in X)~xRx$ --- {\it рефлексивность}.
	\item{}$(\forall x,y\in X)~xRy\implies yRx$
	--- {\it cимметричность}\index{симметричность}.
	\item{}$(\forall x,y,z\in X)~xRy\land yRz\implies xRz$ --- {\it транзитивность}.
\end{enumerate}
Например, отношение $=$ на любом множестве является отношением эквивалентности.

Пусть $R$ --- отношение эквивалентности на $X$ и $x\in X$.
Назовём множество
\[
	[x]:=\left\{y\in X\;\big|\; xRy\right\}
\]
{\it классом эквивалентности} элемента $x$.\index{класс эквивалентности}
Он содержит все элементы, эквивалентные $x$.
По рефлексивности $x \in [x]$.

\vspace{1em}
{\it Теорема:}
Для элементов $x,y\in X$ справедливо
\[
	[x]\cap [y]\neq\eset\implies [x]=[y]
\]

{\it Доказательство:}
\[
	[x]\cap [y]\neq\eset\implies\exists z:z\in[x]\land z\in[y]
\]
Тогда существует $z\in X$ такой, что
\[
	xRz\land yRz\xRightarrow{\text{симметричность}} xRz\land zRy
	\xRightarrow{\text{транзитивность}} xRy
\]
Тогда для произвольного $z\in X$ справедливо
\[
	z\in [y]\iff yRz \iff xRz \iff z\in [x]
\]
и $[x]=[y]$.\qed

Множество всех классов эквивалентности называют
{\it фактормножеством}\index{фактормножество}.
Его можно определить через естественное
расширение отображения $[\cdot]:X\to 2^{X}$
\[
	X/R:=[X]=\left\{\alpha\in 2^{X}\;\big|\; \exists x\in X:\alpha=[x]\right\}
\]

Пусть $\B=X/R=\{B_{i}\}_{i\in I}$\footnote{Любое множество $A$ можно представить
как индексированное семейство:
\[
	I:= A\qquad a_{i}:=i\qquad
	A=\{a_{i}\}_{i\in I}
\]}, где каждому элементу соответствует единственный индекс, тогда
\[
	X=\bigcup_{i\in I}B_{i}\qquad
	(\forall i,j\in I:i\neq j)~B_{i}\cap B_{j}=\eset\qquad
	(\forall i\in I)~B_{i}\neq\eset,
\]
значит множество классов эквивалентности является разбиением множества $X$.

Произвольное разбиение $\{C_{i}\}_{i\in I}$ множества $X$ также задаёт
отношение эквивалентности. Пусть $f:X\to I$ --- такое отображение,
что $(\forall x\in X)~x\in C_{f(x)}$. Отношение
\[
	xRy\iff f(x)=f(y)
\]
является отношением эквивалентности (см. упр.~\ref{ex:fn_equiv}),
причём
\[
	X/R=\{C_{i}\}_{i\in I}
\]

Пусть $R_{m}:\Z\to\N_0$ --- функция остатка при делении на $m\in\N$,
определим её следующим образом:
\[
	S_{m}(z):=\{r\in\N\;\big|\; \exists q\in\Z:z=qm+r\}\qquad
	R_{m}(z):=\min S_{m}(z),
\]
где $\min A$ --- наименьший элемент $A$. В данном случае он существует, потому
что у любого непустого подмножества $\N$ существует наименьший элемент.

Если бы $R_{m}(z)\geq m$, то $(R_{m}(z)-m)\in S_{m}(z)$,
что противоречит определению
наименьшего элемента, значит $0\leq R_{m}(z)< m$.

Определим отношение эквивалентности (см. упр.~\ref{ex:fn_equiv}) на $\Z$:
\[
	x=_{m}y\iff R_{m}(x)=R_{m}(y)\qquad \Z_{m}:=\Z/=_{m}
\]
Это определение эквивалентно
\[
	x=_{m}y\iff \exists q\in\Z:x-y=qm
\]
Введём операцию сложения на $\Z_{m}$, для которой выполняется
\begin{equation}\label{eq:sum_axiom}
	(\forall x,y\in\Z)~[x]+[y]=[x+y]
\end{equation}
Заметим, что первое использование $+$ --- отображение $\Z_{m}^{2}\to \Z_{m}$,
а второе --- отображение $\Z^{2}\to\Z$.

Пусть $f:Z_{m}\to \Z$ --- функция выбора, тогда сложение можно
определить так:
\[
	x+y:=[f(x)+f(y)]
\]

\vspace{1em}
{\it Теорема:}
Такое сложение удовлетворяет \eqref{eq:sum_axiom}.

{\it Доказательство:}
Пусть $x,y\in \Z$ и
\[
	x':=f([x])\quad x'=_{m}x\qquad y':=f([y])\quad y'=_{m}y
\]
\[
	\begin{aligned}
		 & x'=_{m}x\land y'=_{m}y\implies                                 \\
		 & \implies \exists q_{x}\in\Z:x'-x=q_{x}m\land
		\exists q_{y}\in\Z:y'-y=q_{y}m\implies                            \\
		 & \implies \exists q\in\Z:(x'+y')-(x+y)=qm\implies x'+y'=_{m}x+y
	\end{aligned}
\]
\[
	[x]+[y]=[x'+y']=[x+y]\qed
\]

\vspace{1em}
{\it Теорема:} Пусть $+,+':\Z_{m}^{2}\to\Z_{m}^{2}$ --- отображения,
удовлетворяющие \eqref{eq:sum_axiom}. Тогда
\[
	(\forall x,y\in\Z_{m})~x+y=x+'y,
\]
то есть $+,+'$ {\it равны на} $\Z_{m}^{2}$.\index{равенство!отображений}

{\it Доказательство:} Возьмём $x,y\in\Z_{m}$. Они не пустые, значит
существуют $x'\in x,y'\in y$, тогда
\[
	x+y=[x']+[y']=[x'+y']=[x']+'[y']=x+'y\qed
\]

Осталось только определить функцию выбора. Это можно сделать либо через
Аксиому Выбора, либо $f(\alpha):=\min(\alpha\cap\N)$, причём такая функция
выбора будет обладать свойством
\[
	(\forall a\in\Z)~f([a])=R_{m}(a)
\]

Заметим, что из \eqref{eq:sum_axiom} следует существование и единственность\footnote{
	Отображения $f,g:X\to Y$ называют равными и пишут $f=g$, если
	\[
		(\forall x\in X)~f(x)=g(x)
	\]

	То есть $(+)=(+')$, если
	\[
		(\forall x,y\in \Z_{m}^{2})~x+y=x+'y
	\]}
сложения, поэтому можно сказать, что эта формула является определением
сложения на $\Z_{m}^{2}$.

Также говорят, что мы {\it расширили} функцию $+$\index{расширение}
на $\Z_{m}^{2}$. Аналогично, естественное расширение также является расширением.

\vspace{1em}
{\it Упражнения:}
\begin{enumerate}
	\item{}Пусть $f:X\to T$. Доказать, что отношение\label{ex:fn_equiv}
	\[
		aRb\iff f(a)=f(b)
	\]
	является отношением эквивалентности на $X$.

	\item{}Доказать, что следующие отношения являются отношениями эквивалентности:
	\begin{enumerate}
		\item{}$=$ на $X$, где $X$ --- произвольное.
		\item{}Отношение параллельности прямых на множестве прямых.
		\item{}Отношение подобия треугольников на множестве треугольников
		на плоскости.
		\item{}Отношение равенства длины на множестве отрезков.
	\end{enumerate}

	\item{}Пусть $R$ --- отношение эквивалентности на множестве $X$.
	Пусть $(\circ):X^{2}\to X$, причём выполняется
	\[
		(\forall x,x',y,y'\in X)~xRx'\land yRy'\implies (x\circ y)R(x'\circ y')
	\]

	Доказать, опираясь на Аксиому Выбора,
	что существует единственное расширение $\circ$ на $X/R$, удовлетворяющее
	\[
		(\forall x,y\in X)~[x]\circ[y]=[x\circ y]
	\]

	\item{}*Пусть $X$ --- произвольное.
	Определим отношение $\leq_{\mu}$ на $2^{X}$:
	\[
		\beta\leq_{\mu}\alpha\iff \exists f:(f:\alpha\to\beta)\land f(\alpha)=\beta
	\]
	Доказать, что отношение
	\[
		\alpha R\beta\iff (\alpha\leq_{\mu}\beta)\land (\beta\leq_{\mu}\alpha)
	\]
	является отношением эквивалентности на $2^{X}$.

	\item{}*Пусть $\sigma(x):=x\cup\{x\}$. Будем говорить, что $M$
	{\it индуктивно}\index{индуктивное множество}\index{множество!индуктивное}
	и писать $I(M)$, если
	\[
		I(M)\equiv \eset\in M\land \forall x(x\in M\implies \sigma(x)\in M)
	\]
	Будем говорить, что $a$ {\it достижим}\index{достижимость}
	\index{элемент!достижимый} и писать $S(a)$,
	если он содержится во всех индуктивных множествах, то есть
	\[
		S(a)\equiv \forall M(I(M)\implies a\in M)
	\]

	Пусть $P$ --- произвольное индуктивное множество\footnote{
		Аксиома Бесконечности:\index{Аксиома!Бесконечности}
		\[
			\exists M:I(M)
		\]}.
	Определим множество натуральных чисел с $0$ как
	\[
		\N_0:=\left\{a\in P\;\big|\; S(a)\right\}
	\]
	\begin{enumerate}
		\item{}Пусть $P'$ --- другое индуктивное множество, которое может
		отличаться от $P$ и $\N_0':=\left\{a\in P'\;\big|\; S(a)\right\}$.
		Доказать $\N_0=\N_0'$.
		\item{}Доказать, что $\N_0$ --- наименьшее индуктивное множество.
		В этом случае достаточно доказать утверждение
		\[
			(\forall M)~I(M)\implies\N_0\subseteq M
		\]
		\item{}Доказать принцип математической индукции\index{индукция}
		\footnote{Возьмём формулу $P(x)$,
		для которой доказано
		\[
			P(0)\land (\forall n\in\N_0)~[P(n)\implies P(\sigma(n))]
		\]
		Тогда множество
		\[
			M:=\left\{n\in\N_0\;\big|\; P(n)\right\}
		\]
		удовлетворяет условию упражнения и $\N_0\subseteq M$,
		тогда
		\[
			(\forall n\in\N_0)~P(n)
		\]}:
		\[
			[0\in M\land (\forall n\in\N_0)~(n\in M\implies
					\sigma(n)\in M)]\implies \N_0\subseteq M
		\]
		\item{}Пусть отображения $+,+':\N_0^{2}\to\N_0$ оба удовлетворяют
		\begin{enumerate}
			\item{}$(\forall a\in\N_0)~a+0=a$
			\item{}$(\forall a,b\in\N_0)~a+\sigma(b)=\sigma(a+b)$
		\end{enumerate}
		Доказать, что они равны на $\N_0^{2}$. Подсказка:
		использовать индукцию с формулой
		\[
			P(n)\equiv (\forall a\in\N_0)~a+n=a+'n
		\]
		\item{}Пусть отображения $\cdot,\cdot':\N_0^{2}\to\N_0$ оба удовлетворяют
		\begin{enumerate}
			\item{}$(\forall a\in\N_0)~a\cdot 0=0$
			\item{}$(\forall a,b\in\N_0)~a\cdot\sigma(b)=a\cdot b+a$
		\end{enumerate}
		Доказать, что они равны на $\N_0^{2}$.
	\end{enumerate}

	\item{}Пусть $a\in S$ и $f:S\to S$. Дать определение множеству
	\[
		V=\{a,f(a),f(f(a)),f(f(f(a))),...\}
	\]
	В каких случаях оно бесконечно?
\end{enumerate}
