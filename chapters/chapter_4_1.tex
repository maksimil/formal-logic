\part{*Естественная дедукция}

Данная глава не обязательна к прочтению: уже изложенного материала достаточно
для дальнейшего изучения математики, но очень полезна.
В ней будет описана более строгая формальная система.
В конце на основе ZFC будет доказана следующая теорема:

Пусть $A=\{a\}$ и $B\subseteq A$, тогда $B=\varnothing\lor B=A$.

\section{Диаграммы Фитча}

В рассуждениях мы будем пользоваться {\it диаграммами Фитча}:

\begin{figure*}
	\centering
	\begin{subfigure}{0.3\textwidth}
		\[
			\begin{nd}
				\hypo{1}{A}
				\have{2}{B_1}
				\have[\vdots]{}{\vdots}
				\have[n+1]{3}{B_{n}}
				\have[n+2]{4}{C}
			\end{nd}
		\]
		\caption{Диаграмма фитча}\label{fig:ex_diagram}
	\end{subfigure}
	\hfill
	\begin{subfigure}{0.3\textwidth}
		\[
			\begin{nd}
				\hypo{1}{A}
				\open
				\hypo{2}{A'}
				\have[\vdots]{a}{\vdots}
				\have[n]{3}{C}
				\close
				\have{4}{A'\implies C}
			\end{nd}
		\]
		\caption{Поддоказательство}\label{fig:ex_subproof}
	\end{subfigure}
	\hfill
	\begin{subfigure}{0.3\textwidth}
		\[
			\begin{nd}
				\hypo{1}{A}
				\open[t]
				\hypo{}{}
				\have[\vdots]{a}{\vdots}
				\have[n]{3}{Q(t)}
				\close
				\have{4}{(\forall x)~Q(x)}
			\end{nd}
		\]
		\caption{Введение переменной}\label{fig:ex_var_intro}
	\end{subfigure}
	\caption{Диаграммы фитча}\label{fig:ex_diagrams}
\end{figure*}

На рис.~\ref{fig:ex_diagrams}(a) $A$ называется {\it предположением},
строки $2$-$(n+2)$ --- {\it рассуждением},
а $C$ --- {\it заключением}. Такая диаграмма также называется
{\it диаграммой из $A$ в $C$} или {\it доказательством из $A$ в $C$}.

Диаграммы фитча также допускают {\it поддоказательства}.
На рис.~\ref{fig:ex_diagrams}(b)
во второй строке предполагается $A'$, чтобы вывести $C$ и заключить
${A'\implies C}$\footnote{Правило, по которому это возможно будет введено позже.}.

Для рассуждений с кванторами часто нужно вводить новые переменные.
На рис.~\ref{fig:ex_diagrams}(c) во второй строке вводится произвольная переменная $t$.
Строка с предположением не
обязана быть пустой, но для $(\forall x)Q(x)$ нужно вывести $Q(t)$
для произвольного $t$\footnote{
	Более точно это правило вывода будет сформулировано позже.}.

{\it Область видимости} переменной, формулы или поддоказательства $A$ ---
доказательство\footnote{
	Или поддоказательство. Далее под словом ``доказательство'' будет подразумеваться
	``доказательство или поддоказательство''.}, где $A$ написано,
и поддоказательства этого доказательства. $A$ {\it находится
		в области видимости} $B$, если область видимости $B$
содержится в области видимости $A$.
Совокупность $A_1,A_2,...,A_{n}$ {\it находится в одной области видимости},
если в ней существует такой $A_{i}$, что все члены совокупности находятся в
области видимости $A_{i}$. В таком случае {\it совокупность находится в области
видимости} $A_{i}$.

На рис.~\ref{fig:ex_diagrams}(b) область видимости $A'$ и $C$ --- строки $2$-$n$,
а $A$, (${A'\implies C}$) и доказательства $2$-$n$ --- строки $1$-$(n+1)$.
$A$ и $A'$ находятся в одной области видимости --- области видимости $A'$.

\pagebreak
