\documentclass[titlepage, nofonts, nobib]{tufte-handout}

% language
\newcommand{\textls}[2][5]{%
    \begingroup\addfontfeatures{LetterSpace=#1}#2\endgroup
  }
  \renewcommand{\allcapsspacing}[1]{\textls[15]{#1}}
  \renewcommand{\smallcapsspacing}[1]{\textls[10]{#1}}
  \renewcommand{\allcaps}[1]{\textls[15]{\MakeTextUppercase{#1}}}
  \renewcommand{\smallcaps}[1]{\smallcapsspacing{\scshape\MakeTextLowercase{#1}}}
  \renewcommand{\textsc}[1]{\smallcapsspacing{\textsmallcaps{#1}}}

\defaultfontfeatures{Ligatures = TeX}
\usepackage{fontspec}
\usepackage{polyglossia}
\setmainlanguage[babelshorthands=true]{russian}

\setmainfont{cmun}[
Extension=.otf,
UprightFont=*rm,
ItalicFont=*ti,
  BoldFont=*bx,
  BoldItalicFont=*bi,
  SlantedFont=*sl,
]
\setsansfont{cmun}[
  Extension=.otf,
  UprightFont=*ss,
  ItalicFont=*si,
  BoldFont=*sx,
  BoldItalicFont=*so,
]
% \setmonofont{cmun}[
%   Extension=.otf,
%   UprightFont=*btl,% light version
%   ItalicFont=*bto,%  light version
%   BoldFont=*tb,
%   BoldItalicFont=*tx,
% ]

% page styling
\usepackage{enumitem}
% \usepackage{showframe}
\usepackage{multicol}

% math
\usepackage{amsmath, amsfonts, amssymb}
\usepackage{amsthm}
\usepackage{fitch}
\usepackage{caption,subcaption}

% titling
\title{Математическая\\ логика}
\author{Ксения К.}

% document
\begin{document}

\maketitle

\setcounter{tocdepth}{1}
\tableofcontents

\pagebreak
\part{Предисловие}

Данная книга стремится доступно сформулировать логическую систему,
которая может быть использована для изучения математики, строгого
формулирования доказательств и понимания математических утверждений.
Проблема, с которой сталкиваются некоторые первокурсники это попытки
интуитивно понимать теоремы и определения. С таким подходом невозможно
научиться обращаться с неинтуитивными
понятиями\footnote{Например, бесконечные множества, Аксиома Выбора.}.
Нередко утверждение может показаться верным интуитивно,
но оказаться ложным.

В данной книге достаточно много сносок, что не значит, что их
не нужно читать\footnote{Они содержат важные оговорки,
	улучшающие понимание материала.}.

Надеюсь, что после прочтения данной книги вы сможете работать с логическими
утверждениями и формулировать строгие доказательства\footnote{Под словом ``строгий''
	имеется в виду не имеющий шагов, обоснованных лишь интуитивно.}. В качестве практики
после некоторых глав предложены упражнения. Упражнения с повышенной сложностью
отмечены *.

По всем вопросам обращайтесь по адресу {\sl kksenya758@gmail.com}.

\part{Основные понятия}

\section{Логические утверждения}

Центральным понятием в математике является логическое утверждение,
определим это понятие.

Для начала введём {\it алфавит}, на котором эти утверждения будут записаны.
Он будет состоять из следующих компонентов:
\begin{enumerate}
	\item{}{\it Переменных} --- букв латинского, греческого алфавита,
	иврита, арабских цифр и так далее, принимающих {\it значения}.
	Значениями могут быть множества, числа и прочее\footnote{Обычно значения
		являются неопределимыми понятиями, то есть они ``просто есть''. Их смысл
		проявляется в том, как они относятся друг к другу через операции.}.
	\item{}{\it Специальных символов}, которые связывают переменные:
	логические операции, операции сравнения, скобки и прочее.
\end{enumerate}

Пусть переменная $p$ имеет значение $\xi$, тогда можно записать $p=\xi$. Если переменные
$p$ и $q$ имеют одно значение, то $p=q$.

Введём следующие два значения, которые могут принимать
переменные: правда ($\top$) и ложь ($\bot$), также введём логические
операции\footnote{Обратите внимание на определение импликации, потому что оно
	скорее всего отличается от определения, которое вам давали в школе.}:
\begin{enumerate}
	\item{}И ($\land$): $p\land q$ верно означает ``$p$ и $q$
	верны''\footnote{Слово ``верно'' может быть опущено: $p\land q$
		означает ``$p$ и $q$''.}.

	\begin{tabular}{c|cc}
		$\land$ & $\top$ & $\bot$ \\\hline
		$\top$  & $\top$ & $\bot$ \\
		$\bot$  & $\bot$ & $\bot$
	\end{tabular}

	\pagebreak

	\item{}Или ($\lor$): $p\lor q$ означает ``$p$ или $q$''.

	\begin{tabular}{c|cc}
		$\lor$ & $\top$ & $\bot$ \\\hline
		$\top$ & $\top$ & $\top$ \\
		$\bot$ & $\top$ & $\bot$
	\end{tabular}

	\item{} Отрицание ($\lnot$): $\lnot p$ означает ``не $p$''.

	$\lnot\top:=\bot$, $\lnot\bot:=\top$

	$:=$ означает ``определено как'' или ``равно по определению''.

	\item{}Следствие, импликация (${\implies}$): ${p\implies q}$ означает
	``из $p$ следует $q$'', ``$q$ всякий раз, когда $p$'', ``для $q$ {\it достаточно} $p$``,
	``для $p$ {\it необходимо} $q$''.

	$(p\implies q):=(\lnot p)\lor q$\footnote{см. урпажнение~\ref{ex:imply_def}.}

	\item{}Равносильность, эквивалентность (${\iff}$):
	${p\iff q}$, если ``$q$ тогда и только
	тогда, когда $p$''

	$(p\iff q):=(p\implies q)\land (q\implies p)$
\end{enumerate}

Каждой операции дан её аналог на русском языке, но в доказательствах
должны быть использованы формальные определения.

{\it Выражение} --- всякая последовательность символов алфавита. Существует совокупность
выражений, которые ``имеют смысл'', называемых {\it формулами}.
Со значениями $\top$, $\bot$
и операциями $\land$, $\lor$, $\lnot$, $\implies$, $\iff$ формулу можно
определить следующим образом:
\begin{enumerate}
	\item{}Всякая переменная является формулой.
	\item{}Если выражения $A$ и $B$ --- формулы, то следующие выражения
	тоже являются формулами:
	\[
		A\land B\qquad A\lor B\qquad \lnot A\qquad A\implies B\qquad A\iff B
	\]
\end{enumerate}

При введении новых значений и специальных символов это определение
должно быть уточнено. Так, если мы введём числo $1$
как возможное значение переменной, то $1$ является формулой, потому что символ $1$
можно интерпретировать как переменную, всегда имеющую значение $1$, но выражение
$1\land 1$ не является формулой, потому что оно не имеет смысла\footnote{Заметим,
	что $1$ в данном случае сложно называть целым числом, потому что операция сложения
	не была введена. Смысл целых чисел заключается в операциях над ними, а не в том, что мы
	используем арабские цифры для их записи.}.

Выражение может являться формулой только при некоторых значениях
переменных. Такие значения называются {\it допустимыми}.

Рассмотрим формулу ${A:=(p\implies q)}$. $p$ и $q$ называются {\it свободными
переменными}\footnote{Отличие их от связанных переменных будет объяснено позже.},
их допустимые значения это $\top$ и $\bot$.

	{\it Значением формулы} при определённых значениях свободных переменных называется
результат операций, содержащихся в формуле.
Формулы равны, если они имеют одно значение\footnote{$p\land q=p\lor q$ если $p=q=\top$.}.

{\it Логическое утверждение} --- формула, принимающая значение $\top$ или $\bot$
при любых допустимых значениях свободных переменных.

{\it Тавтология} --- логическое утверждение, принимающее значение $\top$ при
любых допустимых значениях свободных переменных.

{\it Упражнения:}

\begin{enumerate}
	\item{}Показать, что следующие утверждения --- тавтологии
	\begin{enumerate}
		\item{}$\top$\footnote{Символ $\top$ по этой причине называется
			символом элементарной тавтологии.}
		\item{}$p\lor (\lnot p)$ --- закон исключённого третьего
		\item{}$(p\land p)\iff p$
		\item{}$(p\land (p\implies q))\implies q$ --- modus ponens
		\item{}$\lnot(p\lor q)\iff (\lnot p)\land (\lnot q)$ --- закон Де Моргана
		\item{}$(\lnot(p)\land p)\iff\bot$\footnote{Символ $\bot$ по этой причине называется
			символом элементарного противоречия.}
		\item{}$p\implies (q\implies p)$
		\item{}${(p\implies q)\iff ((\lnot q)\implies (\lnot p))}$
		\item{}$(\lnot(p)\implies\bot)\implies p$\footnote{На этой тавтологии основаны
			доказательства от противного. То есть если из $\lnot S$ мы можем прийти
			к противоречию, то $S$.}
	\end{enumerate}
	\item{}Показать, что если $A$ --- тавтология и $(A\implies B)$, то $B$.
	\item{}Объяснить, почему запись $p=q$ для двух логических утверждений $p$ и $q$
	означает то же самое, что и $(p\iff q)=\top$.
	\item{}\label{ex:imply_def}Обосновать определение ${p\implies q}$ и использование
	слов ``необходимость'' и ``достаточность''.
	\item{}*Определить понятие формулы для системы, в которой значения это целые числа,
	а специальные символы это $+$, $-$.
	\item{}*Определить понятие формулы для системы, в которой значения это $\top$, $\bot$
	и натуральные числа, а специальные символы это $\land$,  $+$.
\end{enumerate}

\section{Предикаты и кванторы}

Если $P$ --- логическое утверждение со свободными переменными $a_1,...,a_{n}$,
оно называется {\it предикатом} и обозначается $P(a_1,...,a_{n})$. Переменные
в скобках называются {\it аргументами} предиката, причём не все свободные
переменные должны быть аргументами.

Например, рассмотрим предикат ${E(x):=(x\iff y)}$\footnote{Он принимает
	значение $\top$ тогда и только тогда, когда $x\iff y$.}. В нём
$x$ и $y$ --- свободные переменные, но только
$x$ --- аргумент.

Предикаты можно рассматривать как шаблоны логических утверждений.

Если $P(x)$ верно для всякого допустимого значения $x$\footnote{$P(x)$ не обязательно
	является тавтологией, потому что $x$ может быть не единственной
	свободной переменной $P$.}, пишут
\[
	(\forall x)~P(x)
\]

$\forall$ --- {\it квантор всеобщности}.

\pagebreak

Если существует такое значение $x$, что $P(x)$, пишут
\[
	\exists x:P(x)
\]

$\exists$ --- {\it квантор существования}. Знак ``$:$'' читается как ``такой, что''.

Расширим понятие формулы, чтобы оно включало выражения с кванторами.
Пусть если $A$ --- формула, $\gamma$ --- переменная,
то следующие выражения тоже являются формулами:
$(\forall \gamma)A$, ${\exists \gamma:A}$.

Значит выражение
$(\forall x)P(x,y)$ является формулой, поэтому выражение
$(\forall y)(\forall x)P(x,y)$ тоже является формулой.
То есть в одной формуле мы можем использовать несколько кванторов.

Если ${[\exists x:P(x)]\land[(\forall x)(\forall y)(P(x)
				\land P(y)\implies x=y)]}$\footnote{
	Последнее
	условие гарантирует единственность значения $x$, потому что иначе
	могут существовать такие $x$ и $y$, что $x\neq y\land P(x)\land P(y)$.
	В нём использован оператор ``$=$''.
	Формула $x=y$ принимает значение $\top$, если $x$ и $y$ имеют одно значение,
	и $\bot$ --- иначе.}, то пишут
\[
	\exists! x:P(x)
\]

$\exists!$ --- {\it квантор существования и единственности}. Расширим
понятие формулы: пусть если
$A$ --- формула, $\gamma$ --- переменная, то выражение ${\exists!\gamma: A}$
тоже является формулой.

Часто формулы с кванторами сокращают:
\begin{enumerate}
	\item{}Если ${P(x)=[Q(x)\implies R(x)]}$\footnote{Квадратные скобки означают то же самое,
		что и круглые. Они обычно используются, чтобы не запутаться в круглых скобках.},
	то формулу ${(\forall x)P(x)}$ можно сократить
	до ${(\forall x:Q(x))R(x)}$. Читается как ``для всякого $x$ такого, что
	$Q(x)$, $R(x)$''.
	\item{}Формулу $(\forall x:x\prec a)P(x)$ можно сократить до $(\forall x\prec a)P(x)$,
	где вместо $\prec$ могут быть $<$, $>$, $\leq$, $\in$ и прочие.
	\item{}Выражения с одинаковыми кванторами можно объединить:
	выражение $(\forall x)(\forall y)P(x,y)$ можно записать как $(\forall x,y)P(x,y)$,
	а $\exists x:\exists y:P(x,y)$ --- как $\exists x,y:P(x,y)$.
	\item{}Скобки вокруг $\forall x$ можно опустить, если после него следует $\exists$ или
	выражение в скобках.
	То есть ${(\forall \varepsilon)\exists \delta:P(\varepsilon,\delta)}$
	можно сократить до $\forall \varepsilon~\exists \delta:P(\varepsilon,\delta)$,
	а $(\forall x)(x=x)$ до $\forall x(x=x)$.

	Некоторые авторы опускают ``$:$'' после $\exists$ и ставят выражение $\exists x$ в
	скобки: $(\forall a)(\exists b)P(a,b)$.
\end{enumerate}

Для кванторов выполняются следующие {\it законы отрицания}\footnote{Эти законы являются
	частью определения
	кванторов, но их можно обосновать словесно (см. упражнение~\ref{ex:quantor_neg_def}).}:
\[
	\lnot((\forall x)~P(x)):=(\exists x:\lnot P(x))
\]
\[
	\lnot(\exists x:P(x)):=((\forall x)~\lnot P(x))
\]

Переменная $\gamma$ называется {\it связанной} в формуле $F$, если $F$
содержит выражение $K\gamma$, где $K$ --- квантор.
Переменная $\gamma$ называется {\it свободной}, если она не связанна.
Заметим, что если переменная связанна в формуле, то она не может быть
аргументом предиката.

Если одна переменная связанна в двух различных формулах,
то эти два использования одного символа не связанны, то есть
выражение ${[(\forall x)P(x)]\land[\exists x:Q(x)]}$ означает то же самое, что
и $[(\forall \alpha)P(\alpha)]\land[\exists \beta:Q(\beta)]$.

\pagebreak

Рассмотрим следующую формулу:
\[
	P(q):=[(\forall p)~(p\implies (q\implies p))]
\]

$p$ --- связанная переменная, $q$ --- свободная переменная, а
$P$ является тавтогологией и предикатом.

Символ связанной переменной можно заменить на другой:
\[
	P(q)=[(\forall \chi)~(\chi\implies (q\implies \chi))]
\]

{\it Упражнения:}
\begin{enumerate}
	\item{}Обосновать на основе словесных определений законы отрицания
	кванторов\label{ex:quantor_neg_def}.
	\item{}Записать отрицание утверждения
	\[
		\forall \varepsilon>0~\exists \delta>0:
		(\forall x\in\mathbb{R})~[|x|<\delta\implies |f(x)|<\varepsilon]
	\]

	Где $f$ --- числовая функция, $\mathbb{R}$ --- множество действительных чисел.
	Подразумевается, что $\varepsilon$ и $\delta$ --- действительные числа.
	${(x\in\mathbb{R})}$ --- предикат, принимающий значение $\top$,
	если $x$ --- действительное чило.

	\item{}Показать, что если $P(x)$ --- тавтология, то $(\forall x)P(x)$.

	\item{}В каком случае $(\forall x)P(x)$ означает, что $P(x)$ --- тавтология?

	\item{}*Привести пример выражения без свободных переменных,
	являющегося тавтологией.
\end{enumerate}


\pagebreak
\part{Вывод теорем}

\section{Формальная система}

Совокупность правил называется {\it формальной системой}, если выполняются следующие
условия:
\begin{enumerate}
	\item{}Задан {\it алфавит} для составления {\it выражений}.
	\item{}Определено понятия {\it формулы}.
	\item{}Задана совокупность формул, называемых {\it аксиомами}.
	\item{}Заданы правила вывода {\it теорем}.
\end{enumerate}

Формула, выводимая из аксиом называется {\it теоремой}.

\newcommand\ruleR{\mathbf{R}}
\newcommand\ruleC{\mathbf{C}}

Например, можем составить следующую формальную систему:
\begin{enumerate}
	\item{}Алфавит: $a$, $b$, $c$.
	\item{}Всякое непустое выражение является формулой.
	\item{}Аксиомы: $aab$, $c$.
	\item{}Правила вывода:
	\begin{enumerate}
		\item[($\ruleR$)]{}Из формулы можно убрать один символ.
		\item[($\ruleC$)]{}Если формула $A$ содержит только символы $a$ и $b$,
		а формула $C$ содержит только символы $c$, то можно
		вывести $AC$.
	\end{enumerate}
\end{enumerate}

Если формула $T$ выводима из формул $A_1,A_2,...,A_{n}$, то пишут
$A_1,A_2,...,A_{n}\vdash T$. Если $A_1,A_2,...,A_{n}\vdash T$ по правилу вывода $r$,
то можно написать $A_1,A_2,...,A_{n}\vdash_{r} T$.
\[
	aab\vdash_{\ruleR} ab\qquad
	ab\vdash_{\ruleR} b\qquad
	aab\vdash_{\ruleR,\ruleR} b\qquad
	b,c\vdash_{\ruleC}bc
\]

То есть $ab$, $b$ и $bc$ являются теоремами.

\textsc{формальная система является просто набором правил и её формулы
	могут не иметь смысла. Они и не будут иметь смысла, пока мы не начнём их
	интерпретировать, то есть придавать им этот смысл.}

В общем случае формальные системы не обращаются с понятиями ``истины'' и ``лжи'',
они обращаются с понятиями ``выводимости''. Но для удобства определим формулу
как {\it истинную}, если она является аксиомой или теоремой\footnote{
	Если $F$ --- истинная формула, то её истинность исходит из
	определения формальной системы, поэтому такие $F$ принято называть тавтологиями.
}.

Теперь сформулируем формальную систему, в которой мы будем работать.
Алфавит и понятие формулы уже определены. Пусть $\top$~---~аксиома,
а $S\vdash T$ тогда и только тогда, когда ${S\implies T}$.
Тогда тавтологии из первой главы будут истинными
утверждениями\footnote{Имеется в виду истинными в фомальной системе,
	а не принимающими значение $\top$.}, потому что если $A$ --- тавтология,
то ${\top\implies A}$, а значит $\top\vdash A$ и $A$ --- теорема.

\pagebreak

Для кванторов вводятся свои правила вывода. Такие правила для произвольного
символа $\lambda$ обозначаются так: $\lambda$I --- {\it правило
введения}, $\lambda$E --- {\it правило исключения (использования)}\footnote{
	I --- Introduction, введение.

	E --- Elimination, исключение.
}.

\newcommand\Aii{$\forall$I}
\newcommand\Aee{$\forall$E}
\newcommand\Eii{$\exists$I}
\newcommand\Eee{$\exists$E}
\begin{enumerate}
	\item[(\Aii)]{}Из ``для произвольного $t$ выводимо $P(t)$''
	выводимо $(\forall x)P(x)$.
	\item[(\Aee)]{}$(\forall x)P(x)\vdash P(t)$, где $t$ --- произвольная переменная.
	\item[(\Eii)]{}$P(t)\vdash [\exists x:P(x)]$
	\item[(\Eee)]{}$[\exists x:P(x)], [(\forall x)(P(x)\implies C)]\vdash C$
\end{enumerate}

Теперь мы можем формально доказать законы отрицания кванторов,
которые мы вводили как определения.

\phantomsection\label{wordproof}
{\it Теорема:}
$(\forall x)P(x)\implies \lnot[\exists x:\lnot P(x)]$

{\it Доказательство:}

Предположим $(\forall x)P(x)$ и $\exists x:\lnot P(x)$.

Возьмём произвольное $t$. $P(t)$ по \Aee. Тогда $\lnot P(t)\implies\bot$.

По \Aii{} можно сделать вывод $(\forall x)(\lnot P(x)\implies\bot)$.
Тогда по \Eee{} из предположения ${\exists x:\lnot P(x)}$ можно вывести $\bot$,
значит оно неверно\footnote{
	Доказательство от обратного. Если из $A$ можно прийти к противоречию, то $\lnot A$.
} и из предположения $(\forall x)P(x)$
можно вывести $\lnot[\exists x:\lnot P(x)]$.

Можем сделать вывод, что $(\forall x)P(x)\implies \lnot[\exists x:\lnot P(x)]$.
\qed\footnote{$\qed$ означает ``что и требовалось доказать''.}

Обычно для доказательств достаточно интуитивно понимать кванторы,
но важно знать, откуда эта интуиция берётся.

\textsc{В формальной системе мы можем работать с понятиями выводимости
	и истинности.}

Важно понимать связь между понятием истины в формальных системах
и эмпиричной истины\footnote{Эмпиризм --- метод познания через ощущение (наблюдение).}.
Физики подбирают аксиомы (постулаты),
из которых выводимы теоремы (законы), соответствующие наблюдениям.
Они подгоняют истину формальной системы к эмпиричной истине.

Математики же
в большинстве областей подгоняют формальную систему так,
чтобы она соответствовала интуитивной логике,
что не всегда получается: понятие ``импликации'' в математике не совпадает
с понятием ``следствия'' интуитивной логики. Следствие
нередко подразумевает причинно-следственную
связь, импликация же такой связи не подразумевает (например, ${p\implies(q\implies p)}$).

	{\it Упражнения:}

\begin{enumerate}
	\item{}Обосновать правила вывода кванторов.

	\item{}*Сформулировать правила $\land$I, $\land$E.
\end{enumerate}

\pagebreak

\section{Элементы теории множеств}

Рассмотрим примеры доказательств в теории множеств.

{\it Множество} --- совокупность значений, называемых {\it элементами множества}.
Логическое утверждение $x\in S$ означает ``$x$ --- элемент множества $S$''.

Введём новые значения переменных: все множества $S$, для которых для
всякого значения $x$ логическое утверждение ${x\in S}$ имеет определённое
значение\footnote{Существуют и другие способы определить
	понятие множества, например аксиомы ZFC. Я сильно реккомендую
	хотя бы ознакомиться с ними после прочтения книги.}.
Если $S$ содержит только элементы $a_1,a_2,...$, то пишут $S=\{a_1,a_2,...\}$.
$\varnothing:=\{\}$.

Определим следующие понятия для множеств\footnote{Заметьте схожесть между операциями
	над множествами и логическими операциями.}:
\begin{enumerate}
	\item{}$A$ --- подмножество $B$ (${A\subseteq B}$ или $B\supseteq A$) означает
	\[
		(\forall x)~x\in A\implies x\in B
	\]

	\item{}Множества $A$ и $B$ равны ($A=B$) тогда и только тогда, когда
	$A\subseteq B\land B\subseteq A$.
	То есть \textsc{множество определяется его элементами и ничем более}\footnote{
		В целом, если для двух значений определено понятие равненства,
		то имеется в виду, что значение такого рода определяется только этими
		свойствами.

		Например, следующее утверждение для векторов:
		\[
			\vec{a}=\vec{b}\iff\left[|\vec{a}|=|\vec{b}|\land
				\vec{a}\upuparrows\vec{b}\right]
		\]

		Означает, что вектор определяется длиной (модулем) и направлением.
	}.

	\item{}Пересечение множеств $A$ и $B$ --- такое множество $A\cap B$, что
	\[
		(\forall x)~x\in A\cap B\iff(x\in A\land x\in B)
	\]

	\item{}Объединение множеств $A$ и $B$ --- такое множество $A\cup B$, что
	\[
		(\forall x)~x\in A\cup B\iff (x\in A\lor x\in B)
	\]

	\item{}Разность множеств $A$ и $B$ --- такое множество $A\setminus B$, что
	\[
		(\forall x)~x\in A\setminus B\iff (x\in A\land \lnot(x\in B))
	\]
\end{enumerate}

{\it Теорема:} если $S$ --- множество, то $S\cap S=S$\footnote{
	Технически, ``если $S$ --- множество'' можно опустить, но если
	мы введём $\cap$ например, для чисел, то это уточнение станет необходимым.
}.

{\it Доказательство:}

$S\cap S=S$ тогда и только
тогда, когда ${(\forall x)(x\in S\cap S\iff x\in S)}$,
значит это утверждение и нужно доказать.

Возьмём произвольное $t$. Предположим $t\in S\cap S$.
\[
	t\in S\cap S\implies (t\in S\land t\in S)\implies t\in S
\]

То есть $t\in S\cap S\implies t\in S$\footnote{Используется
	тавтология
	\[
		[p\implies q]\land[q\implies t]\implies [p\implies t]
	\]

	Думаю вы уже со школы знакомы с цепочками импликаций.
}.

Предположим $t\in S$.
\[
	t\in S\implies (t\in S\land t\in S)\implies t\in S\cap S
\]

То есть $t\in S\implies t\in S\cap S$ и $t\in S\iff t\in S\cap S$.

По \Aii{} можем вывести $(\forall x)(x\in S\iff x\in S\cap S)$.\qed

\pagebreak

У большинства доказательств следующие шаги:
\begin{enumerate}
	\item{}Раскрыть определения, чтобы выяснить, какое утверждение нужно доказать.
	\item{}Применить логические операции и преобразования, чтобы вывести это утверждение.
	\item{}Если застряли, попробовать доказать от обратного.
\end{enumerate}

{\it Упражнения:}
\begin{enumerate}
	\item{}Доказать следующие утверждения для всяких множеств $R,S,T$
	\begin{enumerate}
		\item{}$S\cup S=S$
		\item{}$R\cap (S\cup T)=(R\cap S)\cup (R\cap T)$
		\item{}$S\cap (S\cup T)=S\cup (S\cap T)=S$
		\item{}$R\subseteq T\implies R\cup (S\cap T)=(R\cup S)\cap T$
	\end{enumerate}

	\item{}Доказать, что для множеств $A$ и $B$ следующие утверждения эквивалентны:
	\[
		A\subseteq B\qquad A\cup B=B\qquad A\cap B=A
		\qquad (\forall x)~\lnot(x\in A\setminus B)
	\]
	\item{}Обосновать альтернативное определение пустого множества:
	\[
		(\forall x)\lnot(x\in \varnothing)
	\]

	Как можно схожим образом определить множества $\{a\}$, $\{a,b\}$,
	$\{a_1,a_2,...,a_{n}\}$?
	\item{}*Возьмём совокупность множеств $U=\{U_{1},U_{2},...\}$.
	Как можно определить объединение, пересечение всех множеств в совокупности?
	Как это сделать, если $U$ бесконечно?
\end{enumerate}

\section{Аксиомы Пеано}

Рассмотрим пример системы\footnote{
	Система, совокупность и набор --- синонимы, а множество --- более формальное
	понятие, обычно определённое аксиомами ZFC.
} аксиом, определяющей натуральные числа: {\it аксиомы Пеано}.

\newcommand\N{\mathbb{N}}
Множество $\N$ с функцией следования $\sigma$\footnote{$\sigma(x)$ нельзя определить
	как $x+1$, потому что операция сложения не была введена.},
определённой для всякого $x\in\N$
называется {\it множеством натуральных чисел}, если для него
выполняются {\it Аксиомы Пеано}:
\begin{enumerate}
	\item{}$1\in \N$
	\item{}$x\in\N\implies \sigma(x)\in\N$
	\item{}$(\forall x\in\N)~\sigma(x)\neq 1$
	\item{}$(\forall a,b\in\N)(\sigma(a)=\sigma(b)\implies a=b)$
	\item{}$P(1)\land [(\forall n\in\N)~(P(n)\implies P(\sigma(n)))]
		\implies(\forall n\in\N)~P(n)$\footnote{На этой аксиоме основаны
		доказательства по индукции.}
\end{enumerate}

\pagebreak

Введём существование такого множества как аксиому, а
все его элементы как
возможные значения переменных\footnote{Множество натуральных чисел можно
	получить и из множеств:
	\[
		1:=\{\varnothing\}\qquad\sigma(x):=x\cup \{x\}
	\]
	\[
		\N:=\{1,\sigma(1),...\}
	\]

	Существование такого множества следует из
	аксиомы бесконечности ZFC (см. в интернете).}.

Обозначим за $F(n)$ следующее утверждение: ${n\in\N}$ можно представить как
$n=\sigma(\sigma(...\sigma(1)))$, причём это выражение конечно.

{\it Теорема:} $(\forall n\in\N)~F(n)$.

	{\it Доказательство:}

$1\in\N$ можно представить как $1$, значит $F(1)$.

Возьмём произвольное $n$ и предположим $F(n)$, тогда $\sigma(n)$ тоже
представимо в виде $\sigma(\sigma(...\sigma(1)))$, то есть $F(\sigma(n))$.

По \Aii{} можем вывести $(\forall n\in\N)(F(n)\implies F(\sigma(n)))$.

Применим аксиому $4$ и выведем $(\forall n\in\N)~F(n)$.\qed

{\it Упражнения:}
\begin{enumerate}
	\item{}Доказать $(\forall n\in\N)~1<n\lor n=1$, если $<$ определено как
	\begin{enumerate}
		\item{}$(\forall a\in\N)~a<\sigma(a)$
		\item{}$(\forall a,b,c\in\N)~(a<b\land b<c)\implies a<c$
	\end{enumerate}
	\item{}Доказать $a<b\land b<a\implies a<a$\footnote{
		Заметьте, что в данном случае $<$ не соответствует вам известному понятию $<$,
		введённому в школе.
	}.
	\item{}*Определить операцию сложения двух натуральных чисел.
	\item{}*Определить операцию умножения двух натуральных чисел.
\end{enumerate}

\part{Элементы теории множеств}

{\it Множество} --- неопределимое понятие в рамках формальной системы,
обычно оно интерпретируется как совокупность объектов,
называемых {\it элементами множества}.
В теории множеств каждый терм --- множество.

Введём в алфавит бинарный предикат $\in$. Формула ${x\in S}$ читается
как ``$x$ --- элемент $S$'' или ``$x$ {\it лежит} в $S$''.

Пусть $P$ --- утверждение о $x$. Можно сказать, что $P$ описывает какое-то
свойство $x$.
Формулы, описывающие свойство каждого элемента множества или
существование элемента с каким-то свойством обычно записывают в сокращённом виде:
\[
	(\forall x\in S)~P\qquad \exists x\in S:P\qquad \exists!x\in S:P
\]
Они читаются как ``$P$ для произвольного $x$ из $S$'' и
``существует (единственный) $x$ в $S$ такой, что $P$'' соответственно.

Например,
\[
	(\forall n\in\N)~\exists m\in\N:n+1=m
\]
читается как ``для любого натурального числа $n$ существует натуральное число
$m$ такое, что $n+1=m$''.

Важно понимать, что единственный новый знак, который вводится в теории множеств
это $\in$. Операции пересечения и объединения, понятие подмножества, пустое множество,
множество $\{a\}$ нужно будет сформулировать через него.

ZF (Zermelo-Fraenkel) --- набор аксиом Цермело-Френкеля, обычно с ними также
используется Аксиома Выбора. Аксиомы ZF с Аксиомой Выбора
обозначаются как ZFC (ZF, axiom of Choice).
ZFC --- одна из наиболее широко используемых систем аксиом теории множеств.

\section{Равенство множеств и операции на множествах}

Введём новый бинарный предикатный знак $\subseteq$.
\[
	a\subseteq b\equiv \forall x(x\in a\implies x\in b)
\]
${a\subseteq b}$ читается как $a$ --- подмножество $b$, то есть все
элементы множества $a$ лежат в $b$.

Введём первую аксиому ZFC:
\[
	(\forall x)(\forall y)[(\forall z)(z\in x\iff z\in y)\implies x=y]
\]
или другими словами
\[
	\forall x\forall y[x\subseteq y\land y\subseteq x\implies x=y]
\]

\textsc{Такие аксиомы формализуют, что мы понимаем под равенством объектов.}
Например, для векторов равенство можно ввести аксиомой
$(\forall \vec{a},\vec{b})~
	\big[\vec{a}\upuparrows\vec{b}\land|\vec{a}|=|\vec{b}|\big]
	\implies \vec{a}=\vec{b}$.

Из аксиом ZFC следуют\footnote{
	Данная книга не будет углубляться в сами аксиомы ZFC, но я сильно реккомендую
	ознакомиться с ними самостоятельно.
} формулы
\[
	\forall x\forall y\exists z:\forall w(w\in z\iff w\in x\land w\in y)
\]
\[
	\forall x\forall y\exists z:\forall w(w\in z\iff w\in x\lor w\in y)
\]

Единственность можно доказать следующим образом:
\[
	w\in z_1\iff w\in x\land w\in y\iff w\in z_2,
\]
тогда $z_1=z_2$ по первой аксиоме.

Можем ввести бинарные функциональные знаки $\cap$, $\cup$ и аксиомы
\[
	\forall x\forall y\forall w(w\in x\cap y\iff w\in x\land w\in y)
\]
\[
	\forall x\forall y\forall w(w\in x\cup y\iff w\in x\lor w\in y)
\]
$A\cap B$ называется {\it пересечением} $A$ и $B$,
$A\cup B$ --- их {\it объединением}.

\vspace{1em}
{\it Теорема:} ${\forall R\forall S(R\subseteq S\implies R\cap S=R\land R\cup S=S)}$.
Обычно формулировки таких теорем расписывают более подробно:
``Пусть $R,S$ --- множества и $R\subseteq S$, тогда $R\cap S=R$ и $R\cup S=S$''.

{\it Доказательство:}

Пусть $R\subseteq S$.
\[
	t\in R\cap S\iff t\in R\land t\in S\xLeftrightarrow{t\in R\implies t\in S} t\in R
\]

Тогда $(\forall x)(x\in R\cap S\iff x\in R)$ и $R\cap S=R$.
\[
	t\in R\cup S\iff t\in R\lor t\in S\iff t\in S
\]

Тогда $R\cup S=S$.\qed

% \pagebreak

\newcommand\eset{\varnothing}
\vspace{1em}
{\it Упражнения:}
\begin{enumerate}
	\item{}Доказать эквивалентность двух формулировок первой аксиомы:
	\[
		\forall x\forall y(\forall z(z\in x\iff z\in y)\implies x=y)
	\]
	\[
		\forall x\forall y(x\subseteq y\land y\subseteq x\implies x=y)
	\]

	\item{}Доказать теоремы
	\begin{fullwidth}
		\begin{multicols}{2}
			\begin{enumerate}
				\item{}$\forall S(S\cup S=S)$
				\item{}$\forall R\forall S\forall T[R\cap (S\cup T)=(R\cap S)\cup (R\cap T)]$
				\item{}$\forall S\forall T(S\cap (S\cup T)=S\cup (S\cap T)=S)$
				\item{}$\forall R\forall S\forall T[R\subseteq T
							\implies R\cup (S\cap T)=(R\cup S)\cap T]$
			\end{enumerate}
		\end{multicols}
	\end{fullwidth}

	\item{}Доказать, что для множеств $A$ и $B$ следующие утверждения эквивалентны
	(любые два утверждения эквивалентны).
	\[
		A\subseteq B\qquad A\cup B=B\qquad A\cap B=A
		\qquad (\forall x)~\lnot(x\in A\setminus B)
	\]
\end{enumerate}

\section{Конструкции из множеств}

Из аксиом ZFC следует, что для $n>0$ справедливо
\[
	\forall a_1...\forall a_{n}\exists !A:\forall w
		[w\in A\iff (w=a_1\lor...\lor w=a_{n})],
\]
тогда введём $n$-арный функциональный знак\footnote{
	То есть мы вводим функциональный знак для каждого натурального числа.
} $\{,\}$ и аксиому
\[
	\forall a_1...\forall a_{n}\forall w
	[w\in \{a_1,a_2,...,a_{n}\}\iff (w=a_1\lor ...\lor w=a_{n})]
\]

Из ZFC следует ${\exists !e:\forall x(x\notin e)}$\footnote{
	см. упражнение~\ref{ex:eset_only}.
}, можем ввести константу $\eset$ и аксиому $\forall x(x\notin \eset)$.
$\eset$ называется {\it пустым множеством}.

На основе теории множеств можно построить многие другие области математики.
Например, натуральные числа можно определить следующим образом:
\[ \begin{array}{llll}
		0 :=\eset\qquad & 1  :=\{\eset\}\qquad & 2  :=\{\eset,\{\eset\}\}\qquad & ... \\
		                & 1:=\{0\}\qquad       & 2:=\{0,1\}
		                & n:=\{0,1,...,n-1\}
	\end{array}
\]

Другими словами, ${0:=\eset}$ и $\sigma(x):=x\cup \{x\}$ --- функция следования,
возвращающая следующее натуральное число после $x$. Заметим, что количество элементов
в множестве $n$ равно $n$. Из аксиом ZFC следует как существование множества таких
натуральных чисел, так и принцип индукции.

Часто в математике нужно понятие {\it упорядоченного множества}~---~множества,
в котором определён порядок элементов. Порядок элементов
конечного упорядоченного множества можно
определить, просто записав его элементы в каком-то порядке.
Упорядоченное множество из двух элементов (упорядоченную пару)
можно определить так:
\[
	(x,y):=\{x,\{x,y\}\}
\]

Рассмотрим задачу по-интереснее: как определить пересечение и объединение
всех множеств в множестве $U=\{u_1,u_2,...\}$, где $U$ может быть бесконечно?
Введём (из аксиом ZFC следуют необходимые формулы) унарные $\cup$, $\cap$
и аксиомы
\[
	\forall U\forall x(x\in \cap U\iff \forall u(u\in U\implies x\in u))
\]
\[
	\forall U\forall x(x\in \cup U\iff \exists u(u\in U\land x\in u))
\]

\vspace{1em}
{\it Упражнения:}
\begin{enumerate}
	\item{}Как можно определить упорядоченные множества из трёх элементов?
	Из одного элемента? Из $n$ элементов?

	\item{}Доказать теоремы
	\begin{enumerate}
		\item{}${\exists e:\forall x(x\notin e)\implies
					\exists!e:\forall x(x\notin e)}$ \label{ex:eset_only}
		\item{}$\forall x\forall y[(x,y)\neq (y,x)]$
	\end{enumerate}
	\item{}Объяснить определения унарных $\cup$ и $\cap$.
\end{enumerate}

\pagebreak


\pagebreak
\part{Заключение}

\section{Истина в математике}

Истинность любого утверждения в математике исходит только из определения правил
вывода и определения аксиом как истинных утверждений.
То есть $1+1=2$ --- истина по определению формальной системы\footnote{
	Поэтому истинные утверждения в математике часто называют тавтологиями.
}.
С помощью натуральных чисел мы формализуем понятие ``если взять одно яблоко
и ещё одно яблоко, то будет два яблока''.

Важно понимать связь между понятием истины в формальных системах
и эмпиричной истины\footnote{Эмпиризм --- метод познания через ощущение (наблюдение).}.
Физики подбирают аксиомы (постулаты),
из которых выводимы теоремы (законы), соответствующие наблюдениям.
Они подгоняют истину формальной системы к эмпиричной истине.
Они формализуют наблюдения в постулатах, чтобы использовать достаточно мощные
инструменты матанализа для предсказания движения тел.

\section{В защиту интуиции}

Ценность формальных доказательств в их точности. Каждый шаг полностью обоснован и,
поэтому, хорошее формальное доказательство неоспоримо.

Но нельзя забывать и о ценности интуитивного понимания формул.
Оно легче укладывается в голове\footnote{Например, диаграммы
	Эйлера-Венна помогают понять и запомнить операции и теоремы, связанные с
	множествами, но их нельзя приводить как доказательства.},
а интуитивное доказательство может натолкнуть на формальное. Поэтому я привожу пример
некоторых словесных интерпретаций формул:
\begin{table}
	\centering
	\begin{tabular}{r|l}
		$\exists k:(\forall n>k)~P$     & $P$ начиная с какого-то $n$       \\
		                                & $P$ для достаточно больших $n$    \\[1em]
		$\exists \delta:\forall x~
		\big(|x|<\delta\implies P\big)$ & $P$ для достаточно малых $x$      \\[1em]
		$\forall \varepsilon~
		\exists \delta:P$               & Для всякого $\varepsilon$ можно   \\
		                                & подобрать такой $\delta$, что $P$
	\end{tabular}
	\caption{Интерпретации формул}\label{table:formula_interp}
\end{table}

{\it Упражнения:}
\begin{enumerate}
	\item{}Обосновать интерпретации в таблице~\ref{table:formula_interp}.
	\item{}Взять лист бумаги и записать все правила и определения, составляющие
	формальную систему, в которой мы работали, то есть построить её с нуля.
\end{enumerate}




\end{document}
